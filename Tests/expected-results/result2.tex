\documentclass[11pt,twoside]{article}\makeatletter

\IfFileExists{xcolor.sty}%
  {\RequirePackage{xcolor}}%
  {\RequirePackage{color}}
\usepackage{colortbl}
\usepackage{wrapfig}
\usepackage{ifxetex}
\ifxetex
  \usepackage{fontspec}
  \usepackage{xunicode}
  \catcode`⃥=\active \def⃥{\textbackslash}
  \catcode`❴=\active \def❴{\{}
  \catcode`❵=\active \def❵{\}}
  \def\textJapanese{\fontspec{IPAMincho}}
  \def\textChinese{\fontspec{HAN NOM A}\XeTeXlinebreaklocale "zh"\XeTeXlinebreakskip = 0pt plus 1pt }
  \def\textKorean{\fontspec{Baekmuk Gulim} }
  \setmonofont{DejaVu Sans Mono}
  
\else
  \IfFileExists{utf8x.def}%
   {\usepackage[utf8x]{inputenc}
      \PrerenderUnicode{–}
    }%
   {\usepackage[utf8]{inputenc}}
  \usepackage[english]{babel}
  \usepackage[T1]{fontenc}
  \usepackage{float}
  \usepackage[]{ucs}
  \uc@dclc{8421}{default}{\textbackslash }
  \uc@dclc{10100}{default}{\{}
  \uc@dclc{10101}{default}{\}}
  \uc@dclc{8491}{default}{\AA{}}
  \uc@dclc{8239}{default}{\,}
  \uc@dclc{20154}{default}{ }
  \uc@dclc{10148}{default}{>}
  \def\textschwa{\rotatebox{-90}{e}}
  \def\textJapanese{}
  \def\textChinese{}
  \IfFileExists{tipa.sty}{\usepackage{tipa}}{}
  \usepackage{times}
\fi
\def\exampleFont{\ttfamily\small}
\DeclareTextSymbol{\textpi}{OML}{25}
\usepackage{relsize}
\RequirePackage{array}
\def\@testpach{\@chclass
 \ifnum \@lastchclass=6 \@ne \@chnum \@ne \else
  \ifnum \@lastchclass=7 5 \else
   \ifnum \@lastchclass=8 \tw@ \else
    \ifnum \@lastchclass=9 \thr@@
   \else \z@
   \ifnum \@lastchclass = 10 \else
   \edef\@nextchar{\expandafter\string\@nextchar}%
   \@chnum
   \if \@nextchar c\z@ \else
    \if \@nextchar l\@ne \else
     \if \@nextchar r\tw@ \else
   \z@ \@chclass
   \if\@nextchar |\@ne \else
    \if \@nextchar !6 \else
     \if \@nextchar @7 \else
      \if \@nextchar (8 \else
       \if \@nextchar )9 \else
  10
  \@chnum
  \if \@nextchar m\thr@@\else
   \if \@nextchar p4 \else
    \if \@nextchar b5 \else
   \z@ \@chclass \z@ \@preamerr \z@ \fi \fi \fi \fi
   \fi \fi  \fi  \fi  \fi  \fi  \fi \fi \fi \fi \fi \fi}
\gdef\arraybackslash{\let\\=\@arraycr}
\def\@textsubscript#1{{\m@th\ensuremath{_{\mbox{\fontsize\sf@size\z@#1}}}}}
\def\Panel#1#2#3#4{\multicolumn{#3}{){\columncolor{#2}}#4}{#1}}
\def\abbr{}
\def\corr{}
\def\expan{}
\def\gap{}
\def\orig{}
\def\reg{}
\def\ref{}
\def\sic{}
\def\persName{}\def\name{}
\def\placeName{}
\def\orgName{}
\def\textcal#1{{\fontspec{Lucida Calligraphy}#1}}
\def\textgothic#1{{\fontspec{Lucida Blackletter}#1}}
\def\textlarge#1{{\large #1}}
\def\textoverbar#1{\ensuremath{\overline{#1}}}
\def\textquoted#1{‘#1’}
\def\textsmall#1{{\small #1}}
\def\textsubscript#1{\@textsubscript{\selectfont#1}}
\def\textxi{\ensuremath{\xi}}
\def\titlem{\itshape}
\newenvironment{biblfree}{}{\ifvmode\par\fi }
\newenvironment{bibl}{}{}
\newenvironment{byline}{\vskip6pt\itshape\fontsize{16pt}{18pt}\selectfont}{\par }
\newenvironment{citbibl}{}{\ifvmode\par\fi }
\newenvironment{docAuthor}{\ifvmode\vskip4pt\fontsize{16pt}{18pt}\selectfont\fi\itshape}{\ifvmode\par\fi }
\newenvironment{docDate}{}{\ifvmode\par\fi }
\newenvironment{docImprint}{\vskip 6pt}{\ifvmode\par\fi }
\newenvironment{docTitle}{\vskip6pt\bfseries\fontsize{18pt}{22pt}\selectfont}{\par }
\newenvironment{msHead}{\vskip 6pt}{\par}
\newenvironment{msItem}{\vskip 6pt}{\par}
\newenvironment{rubric}{}{}
\newenvironment{titlePart}{}{\par }

\newcolumntype{L}[1]{){\raggedright\arraybackslash}p{#1}}
\newcolumntype{C}[1]{){\centering\arraybackslash}p{#1}}
\newcolumntype{R}[1]{){\raggedleft\arraybackslash}p{#1}}
\newcolumntype{P}[1]{){\arraybackslash}p{#1}}
\newcolumntype{B}[1]{){\arraybackslash}b{#1}}
\newcolumntype{M}[1]{){\arraybackslash}m{#1}}
\definecolor{label}{gray}{0.75}
\def\unusedattribute#1{\sout{\textcolor{label}{#1}}}
\DeclareRobustCommand*{\xref}{\hyper@normalise\xref@}
\def\xref@#1#2{\hyper@linkurl{#2}{#1}}
\begingroup
\catcode`\_=\active
\gdef_#1{\ensuremath{\sb{\mathrm{#1}}}}
\endgroup
\mathcode`\_=\string"8000
\catcode`\_=12\relax

\usepackage[a4paper,twoside,lmargin=1in,rmargin=1in,tmargin=1in,bmargin=1in,marginparwidth=0.75in]{geometry}
\usepackage{framed}

\definecolor{shadecolor}{gray}{0.95}
\usepackage{longtable}
\usepackage[normalem]{ulem}
\usepackage{fancyvrb}
\usepackage{fancyhdr}
\usepackage{graphicx}
\usepackage{marginnote}


\renewcommand*{\marginfont}{\itshape\footnotesize}

\def\Gin@extensions{.pdf,.png,.jpg,.mps,.tif}

  \pagestyle{fancy}

\usepackage[pdftitle={Encoding for Interchange: an introduction to the TEI},
 pdfauthor={C. M. Sperberg-McQueen}]{hyperref}
\hyperbaseurl{}

	 \paperwidth210mm
	 \paperheight297mm
              
\def\@pnumwidth{1.55em}
\def\@tocrmarg {2.55em}
\def\@dotsep{4.5}
\setcounter{tocdepth}{3}
\clubpenalty=8000
\emergencystretch 3em
\hbadness=4000
\hyphenpenalty=400
\pretolerance=750
\tolerance=2000
\vbadness=4000
\widowpenalty=10000

\renewcommand\section{\@startsection {section}{1}{\z@}%
     {-1.75ex \@plus -0.5ex \@minus -.2ex}%
     {0.5ex \@plus .2ex}%
     {\reset@font\Large\bfseries\sffamily}}
\renewcommand\subsection{\@startsection{subsection}{2}{\z@}%
     {-1.75ex\@plus -0.5ex \@minus- .2ex}%
     {0.5ex \@plus .2ex}%
     {\reset@font\Large\sffamily}}
\renewcommand\subsubsection{\@startsection{subsubsection}{3}{\z@}%
     {-1.5ex\@plus -0.35ex \@minus -.2ex}%
     {0.5ex \@plus .2ex}%
     {\reset@font\large\sffamily}}
\renewcommand\paragraph{\@startsection{paragraph}{4}{\z@}%
     {-1ex \@plus-0.35ex \@minus -0.2ex}%
     {0.5ex \@plus .2ex}%
     {\reset@font\normalsize\sffamily}}
\renewcommand\subparagraph{\@startsection{subparagraph}{5}{\parindent}%
     {1.5ex \@plus1ex \@minus .2ex}%
     {-1em}%
     {\reset@font\normalsize\bfseries}}


\def\l@section#1#2{\addpenalty{\@secpenalty} \addvspace{1.0em plus 1pt}
 \@tempdima 1.5em \begingroup
 \parindent \z@ \rightskip \@pnumwidth 
 \parfillskip -\@pnumwidth 
 \bfseries \leavevmode #1\hfil \hbox to\@pnumwidth{\hss #2}\par
 \endgroup}
\def\l@subsection{\@dottedtocline{2}{1.5em}{2.3em}}
\def\l@subsubsection{\@dottedtocline{3}{3.8em}{3.2em}}
\def\l@paragraph{\@dottedtocline{4}{7.0em}{4.1em}}
\def\l@subparagraph{\@dottedtocline{5}{10em}{5em}}
\@ifundefined{c@section}{\newcounter{section}}{}
\@ifundefined{c@chapter}{\newcounter{chapter}}{}
\newif\if@mainmatter 
\@mainmattertrue
\def\chaptername{Chapter}
\def\frontmatter{%
  \pagenumbering{roman}
  \def\thechapter{\@roman\c@chapter}
  \def\theHchapter{\roman{chapter}}
  \def\thesection{\@roman\c@section}
  \def\theHsection{\roman{section}}
  \def\@chapapp{}%
}
\def\mainmatter{%
  \cleardoublepage
  \def\thechapter{\@arabic\c@chapter}
  \setcounter{chapter}{0}
  \setcounter{section}{0}
  \pagenumbering{arabic}
  \setcounter{secnumdepth}{6}
  \def\@chapapp{\chaptername}%
  \def\theHchapter{\arabic{chapter}}
  \def\thesection{\@arabic\c@section}
  \def\theHsection{\arabic{section}}
}
\def\backmatter{%
  \cleardoublepage
  \setcounter{chapter}{0}
  \setcounter{section}{0}
  \setcounter{secnumdepth}{2}
  \def\@chapapp{\appendixname}%
  \def\thechapter{\@Alph\c@chapter}
  \def\theHchapter{\Alph{chapter}}
  \appendix
}
\newenvironment{bibitemlist}[1]{%
   \list{\@biblabel{\@arabic\c@enumiv}}%
       {\settowidth\labelwidth{\@biblabel{#1}}%
        \leftmargin\labelwidth
        \advance\leftmargin\labelsep
        \@openbib@code
        \usecounter{enumiv}%
        \let\p@enumiv\@empty
        \renewcommand\theenumiv{\@arabic\c@enumiv}%
	}%
  \sloppy
  \clubpenalty4000
  \@clubpenalty \clubpenalty
  \widowpenalty4000%
  \sfcode`\.\@m}%
  {\def\@noitemerr
    {\@latex@warning{Empty `bibitemlist' environment}}%
    \endlist}

\def\tableofcontents{\section*{\contentsname}\@starttoc{toc}}
\parskip0pt
\parindent1em
\def\Panel#1#2#3#4{\multicolumn{#3}{){\columncolor{#2}}#4}{#1}}
\newenvironment{reflist}{%
  \begin{raggedright}\begin{list}{}
  {%
   \setlength{\topsep}{0pt}%
   \setlength{\rightmargin}{0.25in}%
   \setlength{\itemsep}{0pt}%
   \setlength{\itemindent}{0pt}%
   \setlength{\parskip}{0pt}%
   \setlength{\parsep}{2pt}%
   \def\makelabel##1{\itshape ##1}}%
  }
  {\end{list}\end{raggedright}}
\newenvironment{sansreflist}{%
  \begin{raggedright}\begin{list}{}
  {%
   \setlength{\topsep}{0pt}%
   \setlength{\rightmargin}{0.25in}%
   \setlength{\itemindent}{0pt}%
   \setlength{\parskip}{0pt}%
   \setlength{\itemsep}{0pt}%
   \setlength{\parsep}{2pt}%
   \def\makelabel##1{\upshape\sffamily ##1}}%
  }
  {\end{list}\end{raggedright}}
\newenvironment{specHead}[2]%
 {\vspace{20pt}\hrule\vspace{10pt}%
  \label{#1}\markright{#2}%

  \pdfbookmark[2]{#2}{#1}%
  \hspace{-0.75in}{\bfseries\fontsize{16pt}{18pt}\selectfont#2}%
  }{}
      \def\TheFullDate{}
\def\TheID{\makeatother }
\def\TheDate{2008-02-01}
\title{Encoding for Interchange: an introduction to the TEI}
\author{C. M. Sperberg-McQueen}\makeatletter 
\makeatletter
\newcommand*{\cleartoleftpage}{%
  \clearpage
    \if@twoside
    \ifodd\c@page
      \hbox{}\newpage
      \if@twocolumn
        \hbox{}\newpage
      \fi
    \fi
  \fi
}
\makeatother
\makeatletter
\thispagestyle{empty}
\markright{\@title}\markboth{\@title}{\@author}
\renewcommand\small{\@setfontsize\small{9pt}{11pt}\abovedisplayskip 8.5\p@ plus3\p@ minus4\p@
\belowdisplayskip \abovedisplayskip
\abovedisplayshortskip \z@ plus2\p@
\belowdisplayshortskip 4\p@ plus2\p@ minus2\p@
\def\@listi{\leftmargin\leftmargini
               \topsep 2\p@ plus1\p@ minus1\p@
               \parsep 2\p@ plus\p@ minus\p@
               \itemsep 1pt}
}
\makeatother
\fvset{frame=single,numberblanklines=false,xleftmargin=5mm,xrightmargin=5mm}
\fancyhf{} 
\setlength{\headheight}{14pt}
\fancyhead[LE]{\bfseries\leftmark} 
\fancyhead[RO]{\bfseries\rightmark} 
\fancyfoot[RO]{}
\fancyfoot[CO]{\thepage}
\fancyfoot[LO]{\TheID}
\fancyfoot[LE]{}
\fancyfoot[CE]{\thepage}
\fancyfoot[RE]{\TheID}
\hypersetup{linkbordercolor=0.75 0.75 0.75,urlbordercolor=0.75 0.75 0.75,bookmarksnumbered=true}
\fancypagestyle{plain}{\fancyhead{}\renewcommand{\headrulewidth}{0pt}}\makeatother 
\begin{document}
\let\tabcellsep& \frontmatter 
  \begin{titlepage}
  \begin{docTitle}  \begin{titlePart} TEI Lite: Encoding for Interchange: an introduction to the TEI \end{titlePart} \begin{titlePart} Revised for TEI P5 release\end{titlePart} \end{docTitle} \begin{docAuthor} C. M. Sperberg-McQueen\end{docAuthor} \begin{docDate} February 2006\end{docDate}
  \end{titlepage}
  \cleardoublepage

\section*{Prefatory note}\label{U5-pref}\par
TEI Lite was the name adopted for what the TEI editors originally conceived of as a simple demonstration of how the TEI encoding scheme might be adopted to meet 90\% of the needs of 90\% of the TEI user community. In retrospect, it was predictable that many people should imagine TEI Lite to be all there is to TEI, or find TEI Lite to be far too heavy for their needs.\par
The original TEI Lite was based largely on observations of existing and previous practice in the encoding of texts, particularly as manifest in the collections of the \xref{http://ota.ahds.ac.uk}{Oxford Text Archive} and in our own experience. It is therefore unsurprising that it seems to have become, if not a de facto standard, at least a common point of departure for electronic text centres and encoding projects world wide. Maybe the fact that we actually produced this shortish, readable, manual for it also helped.\par
Early adopters of TEI Lite included a number of ‘Electronic Text Centers’, many of whom produced their own documentation and tutorial materials (some examples are listed in \xref{http://www.tei-c.org/Tutorials}{the TEI Tutorials pages}). It was also widely adopted as the basis for TEI-conformant authoring systems. Documentation introducing TEI Lite has been widely used for tutorial purposes and has been widely translated (see further the list of versions at \url{http://www.tei-c.org/Lite/}).\par
With the publication of TEI P4, the XML version of the TEI Guidelines, which uses the generation of TEI Lite as an example of the modification mechanism built into the TEI Guidelines, the opportunity was taken to produce a lightly revised XML-conformant version, but the present revision is the first substantively changed version since its first appearance in 1997. This revision takes advantage of the many new features introduced into the TEI Guidelines at release P5. A brief list of those changes likely to affect users of previous versions of this document is given below (changes).

\begin{raggedleft}Lou Burnard, February 2006\end{raggedleft}
\mainmatter \par
This document provides an introduction to the recommendations of the Text Encoding Initiative (TEI), by describing a specific subset of the full TEI encoding scheme. The scheme documented here can be used to encode a wide variety of commonly encountered textual features, in such a way as to maximize the usability of electronic transcriptions and to facilitate their interchange among scholars using different computer systems. It is fully compatible with the full TEI scheme, as defined by TEI document P5, \textit{Guidelines for Electronic Text Encoding and Interchange}, as of February 2006, and available from the TEI Consortium website at \url{http://www.tei-c.org}.
\section[{Introduction}]{Introduction}\label{U5-Intro}\par
The Text Encoding Initiative (TEI) Guidelines are addressed to anyone who wants to interchange information stored in an electronic form. They emphasize the interchange of textual information, but other forms of information such as images and sound are also addressed. The Guidelines are equally applicable in the creation of new resources and in the interchange of existing ones.\par
The Guidelines provide a means of making explicit certain features of a text in such a way as to aid the processing of that text by computer programs running on different machines. This process of making explicit we call \textit{markup} or \textit{encoding}. Any textual representation on a computer uses some form of markup; the TEI came into being partly because of the enormous variety of mutually incomprehensible encoding schemes currently besetting scholarship, and partly because of the expanding range of scholarly uses now being identified for texts in electronic form.\par
The TEI Guidelines describe an encoding scheme which can be expressed using a number of different formal languages. The first editions of the Guidelines used the \textit{Standard Generalized Markup Language} (SGML); since 2002, this has been replaced by the use of the Extensible Markup Language (XML). These markup languages have in common the definition of text in terms of \textit{elements} and \textit{attributes}, and rules governing their appearance within a text. The TEI's use of XML is ambitious in its complexity and generality, but it is fundamentally no different from that of any other XML markup scheme, and so any general-purpose XML-aware software is able to process TEI-conformant texts.\par
The TEI was sponsored by the Association for Computers and the Humanities, the Association for Computational Linguistics, and the Association for Literary and Linguistic Computing, and is now maintained and developed by an independent membership consortium, hosted by four major Universities. Funding has been provided in part from the U.S. National Endowment for the Humanities, Directorate General XIII of the Commission of the European Communities, the Andrew W. Mellon Foundation, and the Social Science and Humanities Research Council of Canada. The Guidelines were first published in May 1994, after six years of development involving many hundreds of scholars from different academic disciplines worldwide. During the years that followed, the Guidelines were increasingly influential in the development of the digital library, in the language industries, and even in the development of the World Wide Web itself. The TEI consortium was set up in January 2001, and a year later produced an edition of the Guidelines entirely revised for XML compatibility. In 2004, it set about a major revision of the Guidelines to take full advantage of new schema languages, the first release of which appeared in 2005. This revision of the TEI Lite manual conforms to version 0.3 of this most recent edition of the Guidelines, TEI P5.\par
At the outset of its work, the overall goals of the TEI were defined by the closing statement of a planning conference held at Vassar College, N.Y., in November, 1987; these ‘Poughkeepsie Principles’ were further elaborated in a series of design documents. The Guidelines, say these design documents, should: \begin{itemize}
\item suffice to represent the textual features needed for research;
\item be simple, clear, and concrete;
\item be easy for researchers to use without special-purpose software;
\item allow the rigorous definition and efficient processing of texts;
\item provide for user-defined extensions;
\item conform to existing and emergent standards.
\end{itemize} \par
The world of scholarship is large and diverse. For the Guidelines to have wide acceptability, it was important to ensure that: \begin{enumerate}
\item the common core of textual features be easily shared;
\item additional specialist features be easy to add to (or remove from) a text;
\item multiple parallel encodings of the same feature should be possible;
\item the richness of markup should be user-defined, with a very small minimal requirement;
\item adequate documentation of the text and its encoding should be provided.
\end{enumerate}\par
The present document describes a manageable selection from the extensive set of elements and recommendations resulting from those design goals, which is called \textit{TEI Lite}.\par
In selecting from the several hundred elements defined by the full TEI scheme, we have tried to identify a useful ‘starter set’, comprising the elements which almost every user should know about. Experience working with TEI Lite will be invaluable in understanding the full TEI scheme and in knowing how to integrate specialized parts of it into the general TEI framework.\par
Our goals in defining this subset may be summarized as follows: \begin{itemize}
\item it should be able to handle adequately a reasonably wide variety of texts, at the level of detail found in existing practice (as demonstrated in, for example, the holdings of the Oxford Text Archive);
\item it should be useful for the production of new documents (such as this one) as well as the encoding of existing texts;
\item it should be usable with a wide range of existing XML software;
\item it should be derivable from the full TEI scheme using the extension mechanisms described in the TEI Guidelines;
\item it should be as small and simple as is consistent with the other goals.
\end{itemize} \par
The reader may judge our success in meeting these goals for him or herself. At the time of first writing (1995), our confidence that we have at least partially done so is borne out by its use in practice for the encoding of real texts. The Oxford Text Archive uses TEI Lite when it translates texts from its holdings from their original markup schemes into SGML; the Electronic Text Centers at the University of Virginia and the University of Michigan have used TEI Lite to encode their holdings. And the Text Encoding Initiative itself uses TEI Lite, in its current technical documentation — including this document. \par
Although we have tried to make this document self-contained, as suits a tutorial text, the reader should be aware that it does not cover every detail of the TEI encoding scheme. All of the elements described here are fully documented in the TEI Guidelines themselves, which should be consulted for authoritative reference information on these, and on the many others which are not described here. Some basic knowledge of XML is assumed.
\section[{A Short Example}]{A Short Example}\label{U5-eg}\par
We begin with a short example, intended to show what happens when a passage of prose is typed into a computer by someone with little sense of the purpose of mark-up, or the potential of electronic texts. In an ideal world, such output might be generated by a very accurate optical scanner. It attempts to be faithful to the appearance of the printed text, by retaining the original line breaks, by introducing blanks to represent the layout of the original headings and page breaks, and so forth. Where characters not available on the keyboard are needed (such as the accented letter \textit{a} in \textit{faàl} or the long dash), it attempts to mimic their appearance.\par
\par\hfill\bgroup\exampleFont\vskip 10pt\begin{shaded}
\obeyspaces                           CHAPTER 38\newline
\newline
READER, I married him. A quiet wedding we had: he and I, the par-\newline
son and clerk, were alone present. When we got back from church, I\newline
went into the kitchen of the manor-house, where Mary was cooking\newline
the dinner, and John cleaning the knives, and I said --\newline
  'Mary, I have been married to Mr Rochester this morning.' The\newline
housekeeper and her husband were of that decent, phlegmatic\newline
order of people, to whom one may at any time safely communicate a\newline
remarkable piece of news without incurring the danger of having\newline
one's ears pierced by some shrill ejaculation and subsequently stunned\newline
by a torrent of wordy wonderment. Mary did look up, and she did\newline
stare at me; the ladle with which she was basting a pair of chickens\newline
roasting at the fire, did for some three minutes hang suspended in air,\newline
and for the same space of time John's knives also had rest from the\newline
polishing process; but Mary, bending again over the roast, said only --\newline
   'Have you, miss? Well, for sure!'\newline
   A short time after she pursued, 'I seed you go out with the master,\newline
but I didn't know you were gone to church to be wed'; and she\newline
basted away. John, when I turned to him, was grinning from ear to\newline
ear.\newline
   'I telled Mary how it would be,' he said: 'I knew what Mr Ed-\newline
ward' (John was an old servant, and had known his master when he\newline
was the cadet of the house, therefore he often gave him his Christian\newline
name) -- 'I knew what Mr Edward would do; and I was certain he\newline
would not wait long either: and he's done right, for aught I know. I\newline
wish you joy, miss!' and he politely pulled his forelock.\newline
   'Thank you, John. Mr Rochester told me to give you and Mary\newline
this.'\newline
   I put into his hand a five-pound note.  Without waiting to hear\newline
more, I left the kitchen. In passing the door of that sanctum some time\newline
after, I caught the words --\newline
   'She'll happen do better for him nor ony o' t' grand ladies.' And\newline
again, 'If she ben't one o' th' handsomest, she's noan faa⃥l, and varry\newline
good-natured; and i' his een she's fair beautiful, onybody may see\newline
that.'\newline
   I wrote to Moor House and to Cambridge immediately, to say what\newline
I had done: fully explaining also why I had thus acted. Diana and\newline
\newline
                            474\newline
\newline
                 JANE EYRE                      475\newline
\newline
Mary approved the step unreservedly. Diana announced that she\newline
would just give me time to get over the honeymoon, and then she\newline
would come and see me.\newline
   'She had better not wait till then, Jane,' said Mr Rochester, when I\newline
read her letter to him; 'if she does, she will be too late, for our honey-\newline
moon will shine our life long: its beams will only fade over your\newline
grave or mine.'\newline
   How St John received the news I don't know: he never answered\newline
the letter in which I communicated it: yet six months after he wrote\newline
to me, without, however, mentioning Mr Rochester's name or allud-\newline
ing to my marriage. His letter was then calm, and though very serious,\newline
kind. He has maintained a regular, though not very frequent correspond-\newline
ence ever since: he hopes I am happy, and trusts I am not of those who\newline
live without God in the world, and only mind earthly\newline
things.\end{shaded}
\par\egroup 
\par
This transcription suffers from a number of shortcomings: \begin{itemize}
\item the page numbers and running titles are intermingled with the text in a way which makes it difficult for software to disentangle them;
\item no distinction is made between single quotation marks and apostrophe, so it is difficult to know exactly which passages are in direct speech;
\item the preservation of the copy text's hyphenation means that simple-minded search programs will not find the broken words;
\item the accented letter in \textit{faàl} and the long dash have been rendered by ad hoc keying conventions which follow no standard pattern and will be processed correctly only if the transcriber remembers to mention them in the documentation;
\item paragraph divisions are marked only by the use of white space, and hard carriage returns have been introduced at the end of each line. Consequently, if the size of type used to print the text changes, reformatting will be problematic.
\end{itemize} \par
We now present the same passage, as it might be encoded using the TEI Guidelines. As we shall see, there are many ways in which this encoding could be extended, but as a minimum, the TEI approach allows us to represent the following distinctions: \begin{itemize}
\item Paragraph and chapter divisions are now marked explicitly.
\item Apostrophes are distinguished from quotation marks; direct speech is explicitly marked.
\item The accented letter and the long dash are correctly represented.
\item Page divisions have been marked with an empty \texttt{<pb>} element alone.
\item The lineation of the original has not been retained and words broken by typographic accident at the end of a line have been re-assembled without comment.
\item For convenience of proof reading, a new line has been introduced at the start of each paragraph, but the indentation is removed.
\end{itemize}  \par\bgroup\exampleFont \begin{shaded}\noindent\mbox{}{<\textbf{pb}\hspace*{6pt}{n}="{474}"/>}\mbox{}\newline 
{<\textbf{div}\hspace*{6pt}{n}="{38}"\hspace*{6pt}{type}="{chapter}">}\mbox{}\newline 
\hspace*{6pt}{<\textbf{p}>}Reader, I married him. A quiet wedding we had: he and I,\mbox{}\newline 
\hspace*{6pt}\hspace*{6pt} the parson and clerk, were alone present. When we got back\mbox{}\newline 
\hspace*{6pt}\hspace*{6pt} from church, I went into the kitchen of the manor-house,\mbox{}\newline 
\hspace*{6pt}\hspace*{6pt} where Mary was cooking the dinner, and John cleaning the\mbox{}\newline 
\hspace*{6pt}\hspace*{6pt} knives, and I said —{</\textbf{p}>}\mbox{}\newline 
\hspace*{6pt}{<\textbf{p}>}\mbox{}\newline 
\hspace*{6pt}\hspace*{6pt}{<\textbf{q}>}Mary, I have been married to Mr Rochester this\mbox{}\newline 
\hspace*{6pt}\hspace*{6pt}\hspace*{6pt}\hspace*{6pt} morning.{</\textbf{q}>} The housekeeper and her husband were of that\mbox{}\newline 
\hspace*{6pt}\hspace*{6pt} decent, phlegmatic order of people, to whom one may at any\mbox{}\newline 
\hspace*{6pt}\hspace*{6pt} time safely communicate a remarkable piece of news without\mbox{}\newline 
\hspace*{6pt}\hspace*{6pt} incurring the danger of having one's ears pierced by some\mbox{}\newline 
\hspace*{6pt}\hspace*{6pt} shrill ejaculation and subsequently stunned by a torrent of\mbox{}\newline 
\hspace*{6pt}\hspace*{6pt} wordy wonderment. Mary did look up, and she did stare at\mbox{}\newline 
\hspace*{6pt}\hspace*{6pt} me; the ladle with which she was basting a pair of chickens\mbox{}\newline 
\hspace*{6pt}\hspace*{6pt} roasting at the fire, did for some three minutes hang\mbox{}\newline 
\hspace*{6pt}\hspace*{6pt} suspended in air, and for the same space of time John's\mbox{}\newline 
\hspace*{6pt}\hspace*{6pt} knives also had rest from the polishing process; but Mary,\mbox{}\newline 
\hspace*{6pt}\hspace*{6pt} bending again over the roast, said only —{</\textbf{p}>}\mbox{}\newline 
\hspace*{6pt}{<\textbf{p}>}\mbox{}\newline 
\hspace*{6pt}\hspace*{6pt}{<\textbf{q}>}Have you, miss? Well, for sure!{</\textbf{q}>}\mbox{}\newline 
\hspace*{6pt}{</\textbf{p}>}\mbox{}\newline 
\hspace*{6pt}{<\textbf{p}>}A short time after she pursued, {<\textbf{q}>}I seed you go out with\mbox{}\newline 
\hspace*{6pt}\hspace*{6pt}\hspace*{6pt}\hspace*{6pt} the master, but I didn't know you were gone to church to be\mbox{}\newline 
\hspace*{6pt}\hspace*{6pt}\hspace*{6pt}\hspace*{6pt} wed{</\textbf{q}>}; and she basted away. John, when I turned to him,\mbox{}\newline 
\hspace*{6pt}\hspace*{6pt} was grinning from ear to ear. {<\textbf{q}>}I telled Mary how it would\mbox{}\newline 
\hspace*{6pt}\hspace*{6pt}\hspace*{6pt}\hspace*{6pt} be,{</\textbf{q}>} he said: {<\textbf{q}>}I knew what Mr Edward{</\textbf{q}>} (John was an\mbox{}\newline 
\hspace*{6pt}\hspace*{6pt} old servant, and had known his master when he was the cadet\mbox{}\newline 
\hspace*{6pt}\hspace*{6pt} of the house, therefore he often gave him his Christian\mbox{}\newline 
\hspace*{6pt}\hspace*{6pt} name) — {<\textbf{q}>}I knew what Mr Edward would do; and I was\mbox{}\newline 
\hspace*{6pt}\hspace*{6pt}\hspace*{6pt}\hspace*{6pt} certain he would not wait long either: and he's done right,\mbox{}\newline 
\hspace*{6pt}\hspace*{6pt}\hspace*{6pt}\hspace*{6pt} for aught I know. I wish you joy, miss!{</\textbf{q}>} and he politely\mbox{}\newline 
\hspace*{6pt}\hspace*{6pt} pulled his forelock.{</\textbf{p}>}\mbox{}\newline 
\hspace*{6pt}{<\textbf{p}>}\mbox{}\newline 
\hspace*{6pt}\hspace*{6pt}{<\textbf{q}>}Thank you, John. Mr Rochester told me to give you and\mbox{}\newline 
\hspace*{6pt}\hspace*{6pt}\hspace*{6pt}\hspace*{6pt} Mary this.{</\textbf{q}>}\mbox{}\newline 
\hspace*{6pt}{</\textbf{p}>}\mbox{}\newline 
\hspace*{6pt}{<\textbf{p}>}I put into his hand a five-pound note. Without waiting\mbox{}\newline 
\hspace*{6pt}\hspace*{6pt} to hear more, I left the kitchen. In passing the door of\mbox{}\newline 
\hspace*{6pt}\hspace*{6pt} that sanctum some time after, I caught the words —{</\textbf{p}>}\mbox{}\newline 
\hspace*{6pt}{<\textbf{p}>}\mbox{}\newline 
\hspace*{6pt}\hspace*{6pt}{<\textbf{q}>}She'll happen do better for him nor ony o' t' grand\mbox{}\newline 
\hspace*{6pt}\hspace*{6pt}\hspace*{6pt}\hspace*{6pt} ladies.{</\textbf{q}>} And again, {<\textbf{q}>}If she ben't one o' th'\mbox{}\newline 
\hspace*{6pt}\hspace*{6pt}\hspace*{6pt}\hspace*{6pt} handsomest, she's noan faàl, and varry good-natured;\mbox{}\newline 
\hspace*{6pt}\hspace*{6pt}\hspace*{6pt}\hspace*{6pt} and i' his een she's fair beautiful, onybody may see\mbox{}\newline 
\hspace*{6pt}\hspace*{6pt}\hspace*{6pt}\hspace*{6pt} that.{</\textbf{q}>}\mbox{}\newline 
\hspace*{6pt}{</\textbf{p}>}\mbox{}\newline 
\hspace*{6pt}{<\textbf{p}>}I wrote to Moor House and to Cambridge immediately, to\mbox{}\newline 
\hspace*{6pt}\hspace*{6pt} say what I had done: fully explaining also why I had thus\mbox{}\newline 
\hspace*{6pt}\hspace*{6pt} acted. Diana and {<\textbf{pb}\hspace*{6pt}{n}="{475}"/>} Mary approved the step\mbox{}\newline 
\hspace*{6pt}\hspace*{6pt} unreservedly. Diana announced that she would just give me\mbox{}\newline 
\hspace*{6pt}\hspace*{6pt} time to get over the honeymoon, and then she would come and\mbox{}\newline 
\hspace*{6pt}\hspace*{6pt} see me.{</\textbf{p}>}\mbox{}\newline 
\hspace*{6pt}{<\textbf{p}>}\mbox{}\newline 
\hspace*{6pt}\hspace*{6pt}{<\textbf{q}>}She had better not wait till then, Jane,{</\textbf{q}>} said Mr\mbox{}\newline 
\hspace*{6pt}\hspace*{6pt} Rochester, when I read her letter to him; {<\textbf{q}>}if she does,\mbox{}\newline 
\hspace*{6pt}\hspace*{6pt}\hspace*{6pt}\hspace*{6pt} she will be too late, for our honeymoon will shine our life\mbox{}\newline 
\hspace*{6pt}\hspace*{6pt}\hspace*{6pt}\hspace*{6pt} long: its beams will only fade over your grave or mine.{</\textbf{q}>}\mbox{}\newline 
\hspace*{6pt}{</\textbf{p}>}\mbox{}\newline 
\hspace*{6pt}{<\textbf{p}>}How St John received the news I don't know: he never\mbox{}\newline 
\hspace*{6pt}\hspace*{6pt} answered the letter in which I communicated it: yet six\mbox{}\newline 
\hspace*{6pt}\hspace*{6pt} months after he wrote to me, without, however, mentioning Mr\mbox{}\newline 
\hspace*{6pt}\hspace*{6pt} Rochester's name or alluding to my marriage. His letter was\mbox{}\newline 
\hspace*{6pt}\hspace*{6pt} then calm, and though very serious, kind. He has maintained\mbox{}\newline 
\hspace*{6pt}\hspace*{6pt} a regular, though not very frequent correspondence ever\mbox{}\newline 
\hspace*{6pt}\hspace*{6pt} since: he hopes I am happy, and trusts I am not of those who\mbox{}\newline 
\hspace*{6pt}\hspace*{6pt} live without God in the world, and only mind earthly things.{</\textbf{p}>}\mbox{}\newline 
{</\textbf{div}>}\end{shaded}\egroup\par \par
This particular encoding represents a set of choices or priorities. The decision to focus on Brontë's text, rather than on the printing of it in this particular edition, is an instance of the fundamental \textit{selectivity} of any encoding. An encoding makes explicit only those textual features of importance to the encoder. It is not difficult to think of ways in which the encoding of even this short passage might readily be extended. For example: \begin{itemize}
\item a regularized form of the passages in dialect could be provided;
\item footnotes glossing or commenting on any passage could be added;
\item pointers linking parts of this text to others could be added;
\item proper names of various kinds could be distinguished from the surrounding text;
\item detailed bibliographic information about the text's provenance and context could be prefixed to it;
\item a linguistic analysis of the passage into sentences, clauses, words, etc., could be provided, each unit being associated with appropriate category codes;
\item the text could be segmented into narrative or discourse units;
\item systematic analysis or interpretation of the text could be included in the encoding, with potentially complex alignment or linkage between the text and the analysis, or between the text and one or more translations of it;
\item passages in the text could be linked to images or sound held on other media.
\end{itemize} \par
A TEI-recommended way of carrying out most of these is described in the remainder of this document. The TEI scheme as a whole also provides for an enormous range of other possibilities, of which we cite only a few: \begin{itemize}
\item detailed analysis of the components of names;
\item detailed meta-information providing thesaurus-style information about the text's origins or topics;
\item information about the printing history or manuscript variations exhibited by a particular series of versions of the text.
\end{itemize}  For recommendations on these and many other possibilities, the full Guidelines should be consulted.
\section[{The Structure of a TEI Text}]{The Structure of a TEI Text}\label{U5-struc}\par
All TEI-conformant texts contain (a) a \textit{TEI header} (marked up as a \texttt{<teiHeader>} element) and (b) the transcription of the text proper (marked up as a \texttt{<text>} element). These two elements are combined together to form a single \texttt{<TEI>} element.\par
The TEI header provides information analogous to that provided by the title page of a printed text. It has up to four parts: a bibliographic description of the machine-readable text, a description of the way it has been encoded, a non-bibliographic description of the text (a \textit{text profile}), and a revision history. The header is described in more detail in section \textit{\hyperref[U5-header]{19. The Electronic Title Page}}.\par
A TEI text may be \textit{unitary} (a single work) or \textit{composite} (a collection of single works, such as an anthology). In either case, the text may have an optional \textit{front} or \textit{back}. In between is the \textit{body} of the text, which, in the case of a composite text, may consist of \textit{group}s, each containing more groups or texts.\par
A unitary text will be encoded using an overall structure like this: \par\bgroup\exampleFont \begin{shaded}\noindent\mbox{}{<\textbf{TEI} xmlns="http://www.tei-c.org/ns/1.0">}\mbox{}\newline 
\hspace*{6pt}{<\textbf{teiHeader}>}\mbox{}\newline 
\textit{<!-- [ TEI Header information ]  -->}\mbox{}\newline 
\hspace*{6pt}{</\textbf{teiHeader}>}\mbox{}\newline 
\hspace*{6pt}{<\textbf{text}>}\mbox{}\newline 
\hspace*{6pt}\hspace*{6pt}{<\textbf{front}>}\mbox{}\newline 
\textit{<!-- [ front matter ... ] -->}\mbox{}\newline 
\hspace*{6pt}\hspace*{6pt}{</\textbf{front}>}\mbox{}\newline 
\hspace*{6pt}\hspace*{6pt}{<\textbf{body}>}\mbox{}\newline 
\textit{<!-- [ body of text ... ]  -->}\mbox{}\newline 
\hspace*{6pt}\hspace*{6pt}{</\textbf{body}>}\mbox{}\newline 
\hspace*{6pt}\hspace*{6pt}{<\textbf{back}>}\mbox{}\newline 
\textit{<!--  [ back matter ...  ] -->}\mbox{}\newline 
\hspace*{6pt}\hspace*{6pt}{</\textbf{back}>}\mbox{}\newline 
\hspace*{6pt}{</\textbf{text}>}\mbox{}\newline 
{</\textbf{TEI}>}\end{shaded}\egroup\par \par
A composite text also has an optional front and back. In between occur one or more groups of texts, each with its own optional front and back matter. A composite text will thus be encoded using an overall structure like this: \par\bgroup\exampleFont \begin{shaded}\noindent\mbox{}{<\textbf{TEI} xmlns="http://www.tei-c.org/ns/1.0">}\mbox{}\newline 
\hspace*{6pt}{<\textbf{teiHeader}>}\mbox{}\newline 
\textit{<!--[ header information for the composite ]-->}\mbox{}\newline 
\hspace*{6pt}{</\textbf{teiHeader}>}\mbox{}\newline 
\hspace*{6pt}{<\textbf{text}>}\mbox{}\newline 
\hspace*{6pt}\hspace*{6pt}{<\textbf{front}>}\mbox{}\newline 
\textit{<!--[ front matter for the composite  ]-->}\mbox{}\newline 
\hspace*{6pt}\hspace*{6pt}{</\textbf{front}>}\mbox{}\newline 
\hspace*{6pt}\hspace*{6pt}{<\textbf{group}>}\mbox{}\newline 
\hspace*{6pt}\hspace*{6pt}\hspace*{6pt}{<\textbf{text}>}\mbox{}\newline 
\hspace*{6pt}\hspace*{6pt}\hspace*{6pt}\hspace*{6pt}{<\textbf{front}>}\mbox{}\newline 
\textit{<!--[ front matter of first text ]-->}\mbox{}\newline 
\hspace*{6pt}\hspace*{6pt}\hspace*{6pt}\hspace*{6pt}{</\textbf{front}>}\mbox{}\newline 
\hspace*{6pt}\hspace*{6pt}\hspace*{6pt}\hspace*{6pt}{<\textbf{body}>}\mbox{}\newline 
\textit{<!--[ body of first text  ]-->}\mbox{}\newline 
\hspace*{6pt}\hspace*{6pt}\hspace*{6pt}\hspace*{6pt}{</\textbf{body}>}\mbox{}\newline 
\hspace*{6pt}\hspace*{6pt}\hspace*{6pt}\hspace*{6pt}{<\textbf{back}>}\mbox{}\newline 
\textit{<!--[ back matter of first text ]-->}\mbox{}\newline 
\hspace*{6pt}\hspace*{6pt}\hspace*{6pt}\hspace*{6pt}{</\textbf{back}>}\mbox{}\newline 
\hspace*{6pt}\hspace*{6pt}\hspace*{6pt}{</\textbf{text}>}\mbox{}\newline 
\hspace*{6pt}\hspace*{6pt}\hspace*{6pt}{<\textbf{text}>}\mbox{}\newline 
\hspace*{6pt}\hspace*{6pt}\hspace*{6pt}\hspace*{6pt}{<\textbf{front}>}\mbox{}\newline 
\textit{<!--[ front matter of second text]-->}\mbox{}\newline 
\hspace*{6pt}\hspace*{6pt}\hspace*{6pt}\hspace*{6pt}{</\textbf{front}>}\mbox{}\newline 
\hspace*{6pt}\hspace*{6pt}\hspace*{6pt}\hspace*{6pt}{<\textbf{body}>}\mbox{}\newline 
\textit{<!--[ body of second text  ]-->}\mbox{}\newline 
\hspace*{6pt}\hspace*{6pt}\hspace*{6pt}\hspace*{6pt}{</\textbf{body}>}\mbox{}\newline 
\hspace*{6pt}\hspace*{6pt}\hspace*{6pt}\hspace*{6pt}{<\textbf{back}>}\mbox{}\newline 
\textit{<!--[ back matter of second text ]-->}\mbox{}\newline 
\hspace*{6pt}\hspace*{6pt}\hspace*{6pt}\hspace*{6pt}{</\textbf{back}>}\mbox{}\newline 
\hspace*{6pt}\hspace*{6pt}\hspace*{6pt}{</\textbf{text}>}\mbox{}\newline 
\textit{<!--[ more texts or groups of texts here ]-->}\mbox{}\newline 
\hspace*{6pt}\hspace*{6pt}{</\textbf{group}>}\mbox{}\newline 
\hspace*{6pt}\hspace*{6pt}{<\textbf{back}>}\mbox{}\newline 
\textit{<!--[ back matter for the composite  ]-->}\mbox{}\newline 
\hspace*{6pt}\hspace*{6pt}{</\textbf{back}>}\mbox{}\newline 
\hspace*{6pt}{</\textbf{text}>}\mbox{}\newline 
{</\textbf{TEI}>}\end{shaded}\egroup\par \par
It is also possible to define a composite of TEI texts, each with its own header. Such a collection is known as a \textit{TEI corpus}, and may itself have a header: \par\bgroup\exampleFont \begin{shaded}\noindent\mbox{}{<\textbf{teiCorpus} xmlns="http://www.tei-c.org/ns/1.0">}\mbox{}\newline 
\hspace*{6pt}{<\textbf{teiHeader}>}\mbox{}\newline 
\textit{<!--[header information for the corpus]-->}\mbox{}\newline 
\hspace*{6pt}{</\textbf{teiHeader}>}\mbox{}\newline 
\hspace*{6pt}{<\textbf{TEI}>}\mbox{}\newline 
\hspace*{6pt}\hspace*{6pt}{<\textbf{teiHeader}>}\mbox{}\newline 
\textit{<!--[header information for first text]-->}\mbox{}\newline 
\hspace*{6pt}\hspace*{6pt}{</\textbf{teiHeader}>}\mbox{}\newline 
\hspace*{6pt}\hspace*{6pt}{<\textbf{text}>}\mbox{}\newline 
\textit{<!--[first text in corpus]-->}\mbox{}\newline 
\hspace*{6pt}\hspace*{6pt}{</\textbf{text}>}\mbox{}\newline 
\hspace*{6pt}{</\textbf{TEI}>}\mbox{}\newline 
\hspace*{6pt}{<\textbf{TEI}>}\mbox{}\newline 
\hspace*{6pt}\hspace*{6pt}{<\textbf{teiHeader}>}\mbox{}\newline 
\textit{<!--[header information for second text]-->}\mbox{}\newline 
\hspace*{6pt}\hspace*{6pt}{</\textbf{teiHeader}>}\mbox{}\newline 
\hspace*{6pt}\hspace*{6pt}{<\textbf{text}>}\mbox{}\newline 
\textit{<!--[second text in corpus]-->}\mbox{}\newline 
\hspace*{6pt}\hspace*{6pt}{</\textbf{text}>}\mbox{}\newline 
\hspace*{6pt}{</\textbf{TEI}>}\mbox{}\newline 
{</\textbf{teiCorpus}>}\end{shaded}\egroup\par \noindent  It is not however possible to create a composite of corpora -- that is, a number of \texttt{<teiCorpus>} elements combined together and treated as a single object. This is a restriction of the current version of the TEI Guidelines.\par
In the remainder of this document, we discuss chiefly simple text structures. The discussion in each case consists of a short list of relevant TEI \textit{elements} with a brief definition of each, followed by definitions for any \textit{attributes} specific to that element, and a reference to any \textit{classes} of which the element is a member. These references are linked to full specifications for each object, as given in the TEI \textit{Guidelines}. In most cases, short examples are also given.\par
For example, here are the elements discussed so far: 
\section[{Encoding the Body}]{Encoding the Body}\label{U5-body}\par
As indicated above, a simple TEI document at the textual level consists of the following elements:  Elements specific to front and back matter are described below in section \textit{\hyperref[U5-fronbac]{18. Front and Back Matter}}. In this section we discuss the elements making up the body of a text.
\subsection[{Text Division Elements}]{Text Division Elements}\label{divs}\par
The body of a prose text may be just a series of paragraphs, or these paragraphs may be grouped together into chapters, sections, subsections, etc. Each paragraph is tagged using the \texttt{<p>} tag. The \texttt{<div>} element is used to represent any such grouping of paragraphs. \par
The \textit{@type} attribute on the \texttt{<div>} element may be used to supply a conventional name for this category of text division, or otherwise distinguish them. Typical values might be ‘book’, ‘chapter’, ‘section’, ‘part’, ‘poem’, ‘song’, etc. For a given project, it will usually be advisable to define and adhere to a specific list of such values.\par
A \texttt{<div>} element may itself contain further, nested, \texttt{<div>}s, thus mimicking the traditional structure of a book, which can be decomposed hierarchically into units such as parts, containing chapters, containing sections, and so on. TEI texts in general conform to this simple hierarchic model.\par
The \textit{@xml:id} attribute may be used to supply a unique identifier for the division, which may be used for cross references or other links to it, such as a commentary, as further discussed in section \textit{\hyperref[U5-ptrs]{8. Cross References and Links}}. It is often useful to provide an \textit{@xml:id} attribute for every major structural unit in a text, and to derive its values in some systematic way, for example by appending a section number to a short code for the title of the work in question, as in the examples below.\par
The \textit{@n} attribute may be used to supply (additionally or alternatively) a short mnemonic name or number for the division. If a conventional form of reference or abbreviation for the parts of a work already exists (such as the book/chapter/verse pattern of Biblical citations), the \textit{@n} attribute is the place to record it.\par
The \textit{@xml:lang} attribute may be used to specify the language of the division. Languages are identified by an internationally defined code, as further discussed in section \textit{\hyperref[z636]{6.3. Foreign Words or Expressions}} below.\par
The \textit{@rend} attribute may be used to supply information about the rendition (appearance) of a division, or any other element, as further discussed in section \textit{\hyperref[U5-hilites]{6. Marking Highlighted Phrases}} below. As with the \textit{@type} attribute, a project will often find it useful to predefine the possible values for this attribute, but TEI Lite does not constrain it in anyway.\par
These four attributes, \textit{@xml:id}, \textit{@n}, \textit{@xml:lang}, and \textit{@rend} are so widely useful that they are allowed on any element in any TEI schema: they are \textit{global attributes}. Other global attributes defined in the TEI Lite scheme are discussed in section \textit{\hyperref[xatts]{8.3. Special kinds of Linking}}.\par
The value of every \textit{@xml:id} attribute should be unique within a document. One simple way of ensuring that this is so is to make it reflect the hierarchic structure of the document. For example, Smith's \textit{Wealth of Nations} as first published consists of five books, each of which is divided into chapters, while some chapters are further subdivided into parts. We might define \textit{@xml:id} values for this structure as follows: \par\bgroup\exampleFont \begin{shaded}\noindent\mbox{}{<\textbf{body}>}\mbox{}\newline 
\hspace*{6pt}{<\textbf{div}\hspace*{6pt}{n}="{I}"\hspace*{6pt}{type}="{book}"\hspace*{6pt}{xml:id}="{WN1}">}\mbox{}\newline 
\hspace*{6pt}\hspace*{6pt}{<\textbf{div}\hspace*{6pt}{n}="{I.1}"\hspace*{6pt}{type}="{chapter}"\hspace*{6pt}{xml:id}="{WN101}">}\mbox{}\newline 
\textit{<!-- ... -->}\mbox{}\newline 
\hspace*{6pt}\hspace*{6pt}{</\textbf{div}>}\mbox{}\newline 
\hspace*{6pt}\hspace*{6pt}{<\textbf{div}\hspace*{6pt}{n}="{I.2}"\hspace*{6pt}{type}="{chapter}"\hspace*{6pt}{xml:id}="{WN102}">}\mbox{}\newline 
\textit{<!-- ... -->}\mbox{}\newline 
\hspace*{6pt}\hspace*{6pt}{</\textbf{div}>}\mbox{}\newline 
\textit{<!-- ... -->}\mbox{}\newline 
\hspace*{6pt}\hspace*{6pt}{<\textbf{div}\hspace*{6pt}{n}="{I.10}"\hspace*{6pt}{type}="{chapter}"\mbox{}\newline 
\hspace*{6pt}\hspace*{6pt}\hspace*{6pt}{xml:id}="{WN110}">}\mbox{}\newline 
\hspace*{6pt}\hspace*{6pt}\hspace*{6pt}{<\textbf{div}\hspace*{6pt}{n}="{I.10.1}"\hspace*{6pt}{type}="{part}"\mbox{}\newline 
\hspace*{6pt}\hspace*{6pt}\hspace*{6pt}\hspace*{6pt}{xml:id}="{WN1101}">}\mbox{}\newline 
\textit{<!-- ... -->}\mbox{}\newline 
\hspace*{6pt}\hspace*{6pt}\hspace*{6pt}{</\textbf{div}>}\mbox{}\newline 
\hspace*{6pt}\hspace*{6pt}\hspace*{6pt}{<\textbf{div}\hspace*{6pt}{n}="{I.10.2}"\hspace*{6pt}{type}="{part}"\mbox{}\newline 
\hspace*{6pt}\hspace*{6pt}\hspace*{6pt}\hspace*{6pt}{xml:id}="{WN1102}">}\mbox{}\newline 
\textit{<!-- ... -->}\mbox{}\newline 
\hspace*{6pt}\hspace*{6pt}\hspace*{6pt}{</\textbf{div}>}\mbox{}\newline 
\hspace*{6pt}\hspace*{6pt}{</\textbf{div}>}\mbox{}\newline 
\textit{<!-- ... -->}\mbox{}\newline 
\hspace*{6pt}{</\textbf{div}>}\mbox{}\newline 
\hspace*{6pt}{<\textbf{div}\hspace*{6pt}{n}="{II}"\hspace*{6pt}{type}="{book}"\hspace*{6pt}{xml:id}="{WN2}">}\mbox{}\newline 
\textit{<!-- ... -->}\mbox{}\newline 
\hspace*{6pt}{</\textbf{div}>}\mbox{}\newline 
{</\textbf{body}>}\end{shaded}\egroup\par \par
A different numbering scheme may be used for \textit{@xml:id} and \textit{@n} attributes: this is often useful where a canonical reference scheme is used which does not tally with the structure of the work. For example, in a novel divided into books each containing chapters, where the chapters are numbered sequentially through the whole work, rather than within each book, one might use a scheme such as the following: \par\bgroup\exampleFont \begin{shaded}\noindent\mbox{}{<\textbf{body}>}\mbox{}\newline 
\hspace*{6pt}{<\textbf{div}\hspace*{6pt}{n}="{1}"\hspace*{6pt}{type}="{Volume}"\hspace*{6pt}{xml:id}="{TS01}">}\mbox{}\newline 
\hspace*{6pt}\hspace*{6pt}{<\textbf{div}\hspace*{6pt}{n}="{1}"\hspace*{6pt}{type}="{Chapter}"\hspace*{6pt}{xml:id}="{TS011}">}\mbox{}\newline 
\textit{<!-- ... -->}\mbox{}\newline 
\hspace*{6pt}\hspace*{6pt}{</\textbf{div}>}\mbox{}\newline 
\hspace*{6pt}\hspace*{6pt}{<\textbf{div}\hspace*{6pt}{n}="{2}"\hspace*{6pt}{xml:id}="{TS012}">}\mbox{}\newline 
\textit{<!-- ... -->}\mbox{}\newline 
\hspace*{6pt}\hspace*{6pt}{</\textbf{div}>}\mbox{}\newline 
\hspace*{6pt}{</\textbf{div}>}\mbox{}\newline 
\hspace*{6pt}{<\textbf{div}\hspace*{6pt}{n}="{2}"\hspace*{6pt}{type}="{Volume}"\hspace*{6pt}{xml:id}="{TS02}">}\mbox{}\newline 
\hspace*{6pt}\hspace*{6pt}{<\textbf{div}\hspace*{6pt}{n}="{3}"\hspace*{6pt}{type}="{Chapter}"\hspace*{6pt}{xml:id}="{TS021}">}\mbox{}\newline 
\textit{<!-- ... -->}\mbox{}\newline 
\hspace*{6pt}\hspace*{6pt}{</\textbf{div}>}\mbox{}\newline 
\hspace*{6pt}\hspace*{6pt}{<\textbf{div}\hspace*{6pt}{n}="{4}"\hspace*{6pt}{xml:id}="{TS022}">}\mbox{}\newline 
\textit{<!-- ... -->}\mbox{}\newline 
\hspace*{6pt}\hspace*{6pt}{</\textbf{div}>}\mbox{}\newline 
\hspace*{6pt}{</\textbf{div}>}\mbox{}\newline 
{</\textbf{body}>}\end{shaded}\egroup\par \noindent  Here the work has two volumes, each containing two chapters. The chapters are numbered conventionally 1 to 4, but the \textit{@xml:id} values specified allow them to be regarded additionally as if they were numbered 1.1, 1.2, 2.1, 2.2.
\subsection[{Headings and Closings}]{Headings and Closings}\label{h25}\par
Every \texttt{<div>} may have a title or heading at its start, and (less commonly) a closing such as ‘End of Chapter 1’. The following elements may be used to transcribe them:  Some other elements which may be necessary at the beginning or ending of text divisions are discussed below in section \textit{\hyperref[h52]{18.1.2. Prefatory Matter}}.\par
Whether or not headings and trailers are included in a transcription is a matter for the individual transcriber to decide. Where a heading is completely regular (for example ‘Chapter 1’) or may be automatically constructed from attribute values (e.g. \texttt{<div type="Chapter" n="1">}), it may be omitted; where it contains otherwise unrecoverable text it should always be included. For example, the start of Hardy's \textit{Under the Greenwood Tree} might be encoded as follows: \par\bgroup\exampleFont \begin{shaded}\noindent\mbox{}{<\textbf{div}\hspace*{6pt}{n}="{Winter}"\hspace*{6pt}{type}="{Part}"\hspace*{6pt}{xml:id}="{UGT1}">}\mbox{}\newline 
\hspace*{6pt}{<\textbf{div}\hspace*{6pt}{n}="{1}"\hspace*{6pt}{type}="{Chapter}"\hspace*{6pt}{xml:id}="{UGT11}">}\mbox{}\newline 
\hspace*{6pt}\hspace*{6pt}{<\textbf{head}>}Mellstock-Lane{</\textbf{head}>}\mbox{}\newline 
\hspace*{6pt}\hspace*{6pt}{<\textbf{p}>}To dwellers in a wood almost every species of tree ...\mbox{}\newline 
\hspace*{6pt}\hspace*{6pt}{</\textbf{p}>}\mbox{}\newline 
\hspace*{6pt}{</\textbf{div}>}\mbox{}\newline 
{</\textbf{div}>}\end{shaded}\egroup\par 
\subsection[{Prose, Verse and Drama}]{Prose, Verse and Drama}\label{vedr}\par
As noted above, the paragraphs making up a textual division should be tagged with the \texttt{<p>} tag. For example: \par\bgroup\exampleFont \begin{shaded}\noindent\mbox{}{<\textbf{p}>}I fully appreciate Gen. Pope's splendid achievements\mbox{}\newline 
 with their invaluable results; but you must know that\mbox{}\newline 
 Major Generalships in the Regular Army, are not as\mbox{}\newline 
 plenty as blackberries.\mbox{}\newline 
{</\textbf{p}>}\end{shaded}\egroup\par \noindent   \par
A number of different tags are provided for the encoding of the structural components of verse and performance texts (drama, film, etc.): \par
Here, for example, is the start of a poetic text in which verse lines and stanzas are tagged: \par\bgroup\exampleFont \begin{shaded}\noindent\mbox{}{<\textbf{lg}\hspace*{6pt}{n}="{I}">}\mbox{}\newline 
\hspace*{6pt}{<\textbf{l}>}I Sing the progresse of a\mbox{}\newline 
\hspace*{6pt}\hspace*{6pt} deathlesse soule,{</\textbf{l}>}\mbox{}\newline 
\hspace*{6pt}{<\textbf{l}>}Whom Fate, with God made,\mbox{}\newline 
\hspace*{6pt}\hspace*{6pt} but doth not controule,{</\textbf{l}>}\mbox{}\newline 
\hspace*{6pt}{<\textbf{l}>}Plac'd in most shapes; all times\mbox{}\newline 
\hspace*{6pt}\hspace*{6pt} before the law{</\textbf{l}>}\mbox{}\newline 
\hspace*{6pt}{<\textbf{l}>}Yoak'd us, and when, and since,\mbox{}\newline 
\hspace*{6pt}\hspace*{6pt} in this I sing.{</\textbf{l}>}\mbox{}\newline 
\hspace*{6pt}{<\textbf{l}>}And the great world to his aged evening;{</\textbf{l}>}\mbox{}\newline 
\hspace*{6pt}{<\textbf{l}>}From infant morne, through manly noone I draw.{</\textbf{l}>}\mbox{}\newline 
\hspace*{6pt}{<\textbf{l}>}What the gold Chaldee, of silver Persian saw,{</\textbf{l}>}\mbox{}\newline 
\hspace*{6pt}{<\textbf{l}>}Greeke brass, or Roman iron, is in this one;{</\textbf{l}>}\mbox{}\newline 
\hspace*{6pt}{<\textbf{l}>}A worke t'out weare Seths pillars, bricke and stone,{</\textbf{l}>}\mbox{}\newline 
\hspace*{6pt}{<\textbf{l}>}And (holy writs excepted) made to yeeld to none,{</\textbf{l}>}\mbox{}\newline 
{</\textbf{lg}>}\end{shaded}\egroup\par \par
Note that the \texttt{<l>} element marks verse lines, not typographic lines: the original lineation of the first few lines above has not therefore been made explicit by this encoding, and may be lost. The \texttt{<lb>} element described in section \textit{\hyperref[U5-pln]{5. Page and Line Numbers}} may be used to mark typographic lines if so desired.\par
Sometimes, particularly in dramatic texts, verse lines are split between speakers. The easiest way of encoding this is to use the \textit{@part} attribute to indicate that the lines so fragmented are incomplete, as in this example: \par\bgroup\exampleFont \begin{shaded}\noindent\mbox{}{<\textbf{div}\hspace*{6pt}{n}="{I}"\hspace*{6pt}{type}="{Act}">}\mbox{}\newline 
\hspace*{6pt}{<\textbf{head}>}ACT I{</\textbf{head}>}\mbox{}\newline 
\hspace*{6pt}{<\textbf{div}\hspace*{6pt}{n}="{1}"\hspace*{6pt}{type}="{Scene}">}\mbox{}\newline 
\hspace*{6pt}\hspace*{6pt}{<\textbf{head}>}SCENE I{</\textbf{head}>}\mbox{}\newline 
\hspace*{6pt}\hspace*{6pt}{<\textbf{stage}\hspace*{6pt}{rend}="{italic}">}Enter Barnardo and Francisco, two Sentinels, at several doors{</\textbf{stage}>}\mbox{}\newline 
\hspace*{6pt}\hspace*{6pt}{<\textbf{sp}>}\mbox{}\newline 
\hspace*{6pt}\hspace*{6pt}\hspace*{6pt}{<\textbf{speaker}>}Barn{</\textbf{speaker}>}\mbox{}\newline 
\hspace*{6pt}\hspace*{6pt}\hspace*{6pt}{<\textbf{l}\hspace*{6pt}{part}="{Y}">}Who's there?{</\textbf{l}>}\mbox{}\newline 
\hspace*{6pt}\hspace*{6pt}{</\textbf{sp}>}\mbox{}\newline 
\hspace*{6pt}\hspace*{6pt}{<\textbf{sp}>}\mbox{}\newline 
\hspace*{6pt}\hspace*{6pt}\hspace*{6pt}{<\textbf{speaker}>}Fran{</\textbf{speaker}>}\mbox{}\newline 
\hspace*{6pt}\hspace*{6pt}\hspace*{6pt}{<\textbf{l}>}Nay, answer me. Stand and unfold\mbox{}\newline 
\hspace*{6pt}\hspace*{6pt}\hspace*{6pt}\hspace*{6pt}\hspace*{6pt}\hspace*{6pt} yourself.{</\textbf{l}>}\mbox{}\newline 
\hspace*{6pt}\hspace*{6pt}{</\textbf{sp}>}\mbox{}\newline 
\hspace*{6pt}\hspace*{6pt}{<\textbf{sp}>}\mbox{}\newline 
\hspace*{6pt}\hspace*{6pt}\hspace*{6pt}{<\textbf{speaker}>}Barn{</\textbf{speaker}>}\mbox{}\newline 
\hspace*{6pt}\hspace*{6pt}\hspace*{6pt}{<\textbf{l}\hspace*{6pt}{part}="{I}">}Long live the King!{</\textbf{l}>}\mbox{}\newline 
\hspace*{6pt}\hspace*{6pt}{</\textbf{sp}>}\mbox{}\newline 
\hspace*{6pt}\hspace*{6pt}{<\textbf{sp}>}\mbox{}\newline 
\hspace*{6pt}\hspace*{6pt}\hspace*{6pt}{<\textbf{speaker}>}Fran{</\textbf{speaker}>}\mbox{}\newline 
\hspace*{6pt}\hspace*{6pt}\hspace*{6pt}{<\textbf{l}\hspace*{6pt}{part}="{M}">}Barnardo?{</\textbf{l}>}\mbox{}\newline 
\hspace*{6pt}\hspace*{6pt}{</\textbf{sp}>}\mbox{}\newline 
\hspace*{6pt}\hspace*{6pt}{<\textbf{sp}>}\mbox{}\newline 
\hspace*{6pt}\hspace*{6pt}\hspace*{6pt}{<\textbf{speaker}>}Barn{</\textbf{speaker}>}\mbox{}\newline 
\hspace*{6pt}\hspace*{6pt}\hspace*{6pt}{<\textbf{l}\hspace*{6pt}{part}="{F}">}He.{</\textbf{l}>}\mbox{}\newline 
\hspace*{6pt}\hspace*{6pt}{</\textbf{sp}>}\mbox{}\newline 
\hspace*{6pt}\hspace*{6pt}{<\textbf{sp}>}\mbox{}\newline 
\hspace*{6pt}\hspace*{6pt}\hspace*{6pt}{<\textbf{speaker}>}Fran{</\textbf{speaker}>}\mbox{}\newline 
\hspace*{6pt}\hspace*{6pt}\hspace*{6pt}{<\textbf{l}>}You come most carefully upon\mbox{}\newline 
\hspace*{6pt}\hspace*{6pt}\hspace*{6pt}\hspace*{6pt}\hspace*{6pt}\hspace*{6pt} your hour.{</\textbf{l}>}\mbox{}\newline 
\hspace*{6pt}\hspace*{6pt}{</\textbf{sp}>}\mbox{}\newline 
\textit{<!-- ... -->}\mbox{}\newline 
\hspace*{6pt}{</\textbf{div}>}\mbox{}\newline 
{</\textbf{div}>}\end{shaded}\egroup\par \par
The same mechanism may be applied to stanzas which are divided between two speakers: \par\bgroup\exampleFont \begin{shaded}\noindent\mbox{}{<\textbf{div}>}\mbox{}\newline 
\hspace*{6pt}{<\textbf{sp}>}\mbox{}\newline 
\hspace*{6pt}\hspace*{6pt}{<\textbf{speaker}>}First voice{</\textbf{speaker}>}\mbox{}\newline 
\hspace*{6pt}\hspace*{6pt}{<\textbf{lg}\hspace*{6pt}{part}="{I}"\hspace*{6pt}{type}="{stanza}">}\mbox{}\newline 
\hspace*{6pt}\hspace*{6pt}\hspace*{6pt}{<\textbf{l}>}But why drives on that ship so fast{</\textbf{l}>}\mbox{}\newline 
\hspace*{6pt}\hspace*{6pt}\hspace*{6pt}{<\textbf{l}>}Withouten wave or wind?{</\textbf{l}>}\mbox{}\newline 
\hspace*{6pt}\hspace*{6pt}{</\textbf{lg}>}\mbox{}\newline 
\hspace*{6pt}{</\textbf{sp}>}\mbox{}\newline 
\hspace*{6pt}{<\textbf{sp}>}\mbox{}\newline 
\hspace*{6pt}\hspace*{6pt}{<\textbf{speaker}>}Second Voice{</\textbf{speaker}>}\mbox{}\newline 
\hspace*{6pt}\hspace*{6pt}{<\textbf{lg}\hspace*{6pt}{part}="{F}">}\mbox{}\newline 
\hspace*{6pt}\hspace*{6pt}\hspace*{6pt}{<\textbf{l}>}The air is cut away before.{</\textbf{l}>}\mbox{}\newline 
\hspace*{6pt}\hspace*{6pt}\hspace*{6pt}{<\textbf{l}>}And closes from behind.{</\textbf{l}>}\mbox{}\newline 
\hspace*{6pt}\hspace*{6pt}{</\textbf{lg}>}\mbox{}\newline 
\hspace*{6pt}{</\textbf{sp}>}\mbox{}\newline 
\textit{<!-- ... -->}\mbox{}\newline 
{</\textbf{div}>}\end{shaded}\egroup\par \noindent \par
This example shows how dialogue presented in a prose work as if it were drama should be encoded. It also demonstrates the use of the \textit{@who} attribute to bear a code identifying the speaker of the piece of dialogue concerned: \par\bgroup\exampleFont \begin{shaded}\noindent\mbox{}{<\textbf{div}>}\mbox{}\newline 
\hspace*{6pt}{<\textbf{sp}\hspace*{6pt}{who}="{OPI}">}\mbox{}\newline 
\hspace*{6pt}\hspace*{6pt}{<\textbf{speaker}>}The reverend Doctor Opimiam{</\textbf{speaker}>}\mbox{}\newline 
\hspace*{6pt}\hspace*{6pt}{<\textbf{p}>}I do not think I have named a single unpresentable fish.{</\textbf{p}>}\mbox{}\newline 
\hspace*{6pt}{</\textbf{sp}>}\mbox{}\newline 
\hspace*{6pt}{<\textbf{sp}\hspace*{6pt}{who}="{GRM}">}\mbox{}\newline 
\hspace*{6pt}\hspace*{6pt}{<\textbf{speaker}>}Mr Gryll{</\textbf{speaker}>}\mbox{}\newline 
\hspace*{6pt}\hspace*{6pt}{<\textbf{p}>}Bream, Doctor: there is not much to be said for bream.{</\textbf{p}>}\mbox{}\newline 
\hspace*{6pt}{</\textbf{sp}>}\mbox{}\newline 
\hspace*{6pt}{<\textbf{sp}\hspace*{6pt}{who}="{OPI}">}\mbox{}\newline 
\hspace*{6pt}\hspace*{6pt}{<\textbf{speaker}>}The Reverend Doctor Opimiam{</\textbf{speaker}>}\mbox{}\newline 
\hspace*{6pt}\hspace*{6pt}{<\textbf{p}>}On the contrary, sir, I think there is much to be said for him.\mbox{}\newline 
\hspace*{6pt}\hspace*{6pt}\hspace*{6pt}\hspace*{6pt} In the first place....{</\textbf{p}>}\mbox{}\newline 
\hspace*{6pt}\hspace*{6pt}{<\textbf{p}>}Fish, Miss Gryll -- I could discourse to you on fish by\mbox{}\newline 
\hspace*{6pt}\hspace*{6pt}\hspace*{6pt}\hspace*{6pt} the hour: but for the present I will forbear.{</\textbf{p}>}\mbox{}\newline 
\hspace*{6pt}{</\textbf{sp}>}\mbox{}\newline 
{</\textbf{div}>}\end{shaded}\egroup\par 
\section[{Page and Line Numbers}]{Page and Line Numbers}\label{U5-pln}\par
Page and line breaks may be marked with the following empty elements.  These elements mark a single point in the text, not a span of text. The global \textit{@n} attribute should be used to supply the number of the page or line beginning at the tag.\par
When working from a paginated original, it is often useful to record its pagination, if only to simplify later proof-reading. Recording the line breaks may be useful for the same reason; treatment of end-of-line hyphenation in printed source texts will require some consideration.\par
If pagination, etc., are marked for more than one edition, specify the edition in question using the \textit{@ed} attribute, and supply as many tags are necessary. For example, in the following passage we indicate where the page breaks occur in two different editions (‘ED1’ and ‘ED2’) \par\bgroup\exampleFont \begin{shaded}\noindent\mbox{}{<\textbf{p}>}I wrote to Moor House and to Cambridge immediately, to\mbox{}\newline 
 say what I had done: fully explaining also why I had thus\mbox{}\newline 
 acted. Diana and {<\textbf{pb}\hspace*{6pt}{ed}="{ED1}"\hspace*{6pt}{n}="{475}"/>} Mary approved the\mbox{}\newline 
 step unreservedly. Diana announced that she would\mbox{}\newline 
{<\textbf{pb}\hspace*{6pt}{ed}="{ED2}"\hspace*{6pt}{n}="{485}"/>}just give me time to get over the\mbox{}\newline 
 honeymoon, and then she would come and see me.{</\textbf{p}>}\end{shaded}\egroup\par \par
The \texttt{<pb>} and \texttt{<lb>} elements are special cases of the general class of \textit{milestone} elements which mark reference points within a text. TEI Lite also includes a generic \texttt{<milestone>} element, which is not restricted to special cases but can mark any kind of reference point: for example, a column break, the start of a new kind of section not otherwise tagged, or in general any significant change in the text not marked by an XML element. The names used for types of unit and for editions referred to by the \textit{@ed} and \textit{@unit} attributes may be chosen freely, but should be documented in the header. The \texttt{<milestone>} element may be used to replace the others, or the others may be used as a set; they should not be mixed arbitrarily.
\section[{Marking Highlighted Phrases}]{Marking Highlighted Phrases}\label{U5-hilites}
\subsection[{Changes of Typeface, etc.}]{Changes of Typeface, etc.}\label{faces}\par
Highlighted words or phrases are those made visibly different from the rest of the text, typically by a change of type font, handwriting style, ink colour etc., which is intended to draw the reader's attention to some associated change.\par
The global \textit{@rend} attribute can be attached to any element, and used wherever necessary to specify details of the highlighting used for it. For example, a heading rendered in bold might be tagged \texttt{<head rend="bold">}, and one in italic \texttt{<head rend="italic">}.\par
It is not always possible or desirable to interpret the reasons for such changes of rendering in a text. In such cases, the element \texttt{<hi>} may be used to mark a sequence of highlighted text without making any claim as to its status. \par
In the following example, the use of a distinct typeface for the subheading and for the included name are recorded but not interpreted: \par\bgroup\exampleFont \begin{shaded}\noindent\mbox{}{<\textbf{p}>}\mbox{}\newline 
\hspace*{6pt}{<\textbf{hi}\hspace*{6pt}{rend}="{gothic}">}And this Indenture further witnesseth{</\textbf{hi}>}\mbox{}\newline 
 that the said {<\textbf{hi}\hspace*{6pt}{rend}="{italic}">}Walter Shandy{</\textbf{hi}>}, merchant,\mbox{}\newline 
 in consideration of the said intended marriage ...\mbox{}\newline 
{</\textbf{p}>}\end{shaded}\egroup\par \par
Alternatively, where the cause for the highlighting can be identified with confidence, a number of other, more specific, elements are available. \par
Some features (notably quotations and glosses) may be found in a text either marked by highlighting, or with quotation marks. In either case, the elements \texttt{<q>} and \texttt{<gloss>} (as discussed in the following section) should be used. If the rendition is to be recorded, use the global \textit{@rend} attribute.\par
As an example of the elements defined here, consider the following sentence: 
\begin{quote}On the one hand the \textit{Nibelungenlied} is associated with the new rise of romance of twelfth-century France, the {\itshape romans d'antiquité}, the romances of Chrétien de Troyes, and the German adaptations of these works by Heinrich van Veldeke, Hartmann von Aue, and Wolfram von Eschenbach.\end{quote}
 Interpreting the role of the highlighting, the sentence might look like this: \par\bgroup\exampleFont \begin{shaded}\noindent\mbox{}{<\textbf{p}>}On the one hand the {<\textbf{title}>}Nibelungenlied{</\textbf{title}>} is associated\mbox{}\newline 
 with the new rise of romance of twelfth-century France, the\mbox{}\newline 
{<\textbf{foreign}>}romans d'antiquité{</\textbf{foreign}>}, the romances of\mbox{}\newline 
 Chrétien de Troyes, ...{</\textbf{p}>}\end{shaded}\egroup\par \noindent  Describing only the appearance of the original, it might look like this: \par\bgroup\exampleFont \begin{shaded}\noindent\mbox{}{<\textbf{p}>}On the one hand the {<\textbf{hi}\hspace*{6pt}{rend}="{italic}">}Nibelungenlied{</\textbf{hi}>}\mbox{}\newline 
 is associated with the new rise of romance of twelfth-century\mbox{}\newline 
 France, the {<\textbf{hi}\hspace*{6pt}{rend}="{italic}">}romans\mbox{}\newline 
\hspace*{6pt}\hspace*{6pt} d'antiquité{</\textbf{hi}>}, the romances of\mbox{}\newline 
 Chrétien de Troyes, ...{</\textbf{p}>}\end{shaded}\egroup\par 
\subsection[{Quotations and Related Features}]{Quotations and Related Features}\label{z635}\par
Like changes of typeface, quotation marks are conventionally used to denote several different features within a text, of which the most frequent is quotation. When possible, we recommend that the underlying feature be tagged, rather than the simple fact that quotation marks appear in the text, using the following elements: \par
Here is a simple example of a quotation: \par\bgroup\exampleFont \begin{shaded}\noindent\mbox{}{<\textbf{p}>}Few dictionary makers are likely to forget\mbox{}\newline 
 Dr. Johnson's description of the\mbox{}\newline 
 lexicographer as {<\textbf{q}>}a harmless drudge.{</\textbf{q}>}\mbox{}\newline 
{</\textbf{p}>}\end{shaded}\egroup\par \noindent \par
To record how a quotation was printed (for example, \textit{in-line} or set off as a \textit{display} or \textit{block quotation}), the \textit{@rend} attribute should be used. This may also be used to indicate the kind of quotation marks used.\par
Direct speech interrupted by a narrator can be represented simply by ending the quotation and beginning it again after the interruption, as in the following example: \par\bgroup\exampleFont \begin{shaded}\noindent\mbox{}{<\textbf{p}>}\mbox{}\newline 
\hspace*{6pt}{<\textbf{q}>}Who-e debel you?{</\textbf{q}>} — he at last said — {<\textbf{q}>}you\mbox{}\newline 
\hspace*{6pt}\hspace*{6pt} no speak-e, damme, I kill-e.{</\textbf{q}>} And so saying, the lighted\mbox{}\newline 
 tomahawk began flourishing about me in the dark.\mbox{}\newline 
{</\textbf{p}>}\end{shaded}\egroup\par \noindent  If it is important to convey the idea that the two \texttt{<q>} elements together make up a single speech, the linking attributes \textit{@next} and \textit{@prev} may be used, as described in section \textit{\hyperref[xatts]{8.3. Special kinds of Linking}}.\par
Quotations may be accompanied by a reference to the source or speaker, using the \textit{@who} attribute, whether or not the source is given in the text, as in the following example: \par\bgroup\exampleFont \begin{shaded}\noindent\mbox{}{<\textbf{q}\hspace*{6pt}{who}="{Wilson}">}Spaulding, he came down into the office just this\mbox{}\newline 
 day eight weeks with this very paper in his hand, and he\mbox{}\newline 
 says:—{<\textbf{q}\hspace*{6pt}{who}="{Spaulding}">}I wish to the Lord, Mr. Wilson, that\mbox{}\newline 
\hspace*{6pt}\hspace*{6pt} I was a red-headed man.{</\textbf{q}>}\mbox{}\newline 
{</\textbf{q}>}\end{shaded}\egroup\par \noindent  This example also demonstrates how quotations may be embedded within other quotations: one speaker (Wilson) quotes another speaker (Spaulding).\par
The creator of the electronic text must decide whether quotation marks are replaced by the tags or whether the tags are added and the quotation marks kept. If the quotation marks are removed from the text, the \textit{@rend} attribute may be used to record the way in which they were rendered in the copy text.\par
As with highlighting, it is not always possible and may not be considered desirable to interpret the function of quotation marks in a text in this way. In such cases, the tag \texttt{<hi rend="quoted">} might be used to mark quoted text without making any claim as to its status.
\subsection[{Foreign Words or Expressions}]{Foreign Words or Expressions}\label{z636}\par
Words or phrases which are not in the main language of the texts may be tagged as such in one of two ways. If the word or phrase is already tagged for some reason, the element indicated should bear a value for the global \textit{@xml:lang} attribute indicating the language used. Where there is no applicable element, the element \texttt{<foreign>} may be used, again using the \textit{@xml:lang} attribute. For example: \par\bgroup\exampleFont \begin{shaded}\noindent\mbox{}{<\textbf{p}>}John has real\mbox{}\newline 
{<\textbf{foreign}\hspace*{6pt}{xml:lang}="{fra}">}savoir-faire{</\textbf{foreign}>}.{</\textbf{p}>}\mbox{}\newline 
{<\textbf{p}>}Have you read {<\textbf{title}\hspace*{6pt}{xml:lang}="{deu}">}Die Dreigroschenoper{</\textbf{title}>}?{</\textbf{p}>}\mbox{}\newline 
{<\textbf{p}>}\mbox{}\newline 
\hspace*{6pt}{<\textbf{mentioned}\hspace*{6pt}{xml:lang}="{fra}">}Savoir-faire{</\textbf{mentioned}>} is French for\mbox{}\newline 
 know-how.\mbox{}\newline 
{</\textbf{p}>}\mbox{}\newline 
{<\textbf{p}>}The court issued a writ of {<\textbf{term}\hspace*{6pt}{xml:lang}="{lat}">}mandamus{</\textbf{term}>}.{</\textbf{p}>}\end{shaded}\egroup\par \par
As these examples show, the \texttt{<foreign>} element should not be used to tag foreign words if some other more specific element such as \texttt{<title>}, \texttt{<mentioned>}, or \texttt{<term>} applies. The global \textit{@xml:lang} attribute may be attached to any element to show that it uses some other language than that of the surrounding text.\par
The codes used to identify languages, supplied on the \textit{@xml:lang} attribute, must be constructed in a particular way, and must conform to common Internet standards\footnote{The relevant standards are RFC 3066, and the lists of two and three language identifiers maintained as part of ISO 639 (see http://www.w3.org/WAI/ER/IG/ert/iso639.htm)}, as further explained in the relevant section of the TEI Guidelines. Some simple example codes for a few languages are given here:  \par 
\begin{longtable}{P{0.18133333333333335\textwidth}P{0.22666666666666666\textwidth}P{0.221\textwidth}P{0.221\textwidth}}
zh or zho\tabcellsep Chinese\tabcellsep grc\tabcellsep Ancient Greek\\
en\tabcellsep English\tabcellsep ell or el\tabcellsep Greek\\
enm\tabcellsep Middle English\tabcellsep ja or jpn\tabcellsep Japanese\\
fr or fra\tabcellsep French\tabcellsep la or lat\tabcellsep Latin\\
de or deu\tabcellsep German\tabcellsep sa or san\tabcellsep Sanskrit\end{longtable} \par
 
\section[{Notes}]{Notes}\label{U5-notes}\par
All notes, whether printed as footnotes, endnotes, marginalia, or elsewhere, should be marked using the same element:  Where possible, the body of a note should be inserted in the text at the point at which its identifier or mark first appears. This may not be possible for example with marginalia, which may not be anchored to an exact location. For simplicity, it may be adequate to position marginal notes before the relevant paragraph or other element. Notes may also be placed in a separate division of the text (as end-notes are, in printed books) and linked to the relevant portion of the text using their \textit{@target} attribute.\par
The \textit{@n} attribute may be used to supply the number or identifier of a note if this is required. The \textit{@resp} attribute should be used consistently to distinguish between authorial and editorial notes, if the work has both kinds; otherwise, the TEI header should state which kind they are.\par
Examples: \par\bgroup\exampleFont \begin{shaded}\noindent\mbox{}{<\textbf{p}>}Collections are ensembles of distinct\mbox{}\newline 
 entities or objects of any sort.\mbox{}\newline 
{<\textbf{note}\hspace*{6pt}{n}="{1}"\hspace*{6pt}{place}="{foot}">}We explain below why we use the uncommon term\mbox{}\newline 
\hspace*{6pt}{<\textbf{mentioned}>}collection{</\textbf{mentioned}>}\mbox{}\newline 
\hspace*{6pt}\hspace*{6pt} instead of the expected {<\textbf{mentioned}>}set{</\textbf{mentioned}>}.\mbox{}\newline 
\hspace*{6pt}\hspace*{6pt} Our usage corresponds to the {<\textbf{mentioned}>}aggregate{</\textbf{mentioned}>}\mbox{}\newline 
\hspace*{6pt}\hspace*{6pt} of many mathematical writings and to the sense of\mbox{}\newline 
\hspace*{6pt}{<\textbf{mentioned}>}class{</\textbf{mentioned}>} found\mbox{}\newline 
\hspace*{6pt}\hspace*{6pt} in older logical writings.\mbox{}\newline 
\hspace*{6pt}{</\textbf{note}>}\mbox{}\newline 
 The elements ...{</\textbf{p}>}\end{shaded}\egroup\par \noindent  \par\bgroup\exampleFont \begin{shaded}\noindent\mbox{}{<\textbf{lg}\hspace*{6pt}{xml:id}="{RAM609}">}\mbox{}\newline 
\hspace*{6pt}{<\textbf{note}\hspace*{6pt}{place}="{margin}">}The curse is finally expiated{</\textbf{note}>}\mbox{}\newline 
\hspace*{6pt}{<\textbf{l}>}And now this spell was snapt: once more{</\textbf{l}>}\mbox{}\newline 
\hspace*{6pt}{<\textbf{l}>}I viewed the ocean green,{</\textbf{l}>}\mbox{}\newline 
\hspace*{6pt}{<\textbf{l}>}And looked far forth, yet little saw{</\textbf{l}>}\mbox{}\newline 
\hspace*{6pt}{<\textbf{l}>}Of what had else been seen —{</\textbf{l}>}\mbox{}\newline 
{</\textbf{lg}>}\end{shaded}\egroup\par 
\section[{Cross References and Links}]{Cross References and Links}\label{U5-ptrs}\par
Explicit cross references or links from one point in a text to another in the same or another document may be encoded using the elements described in this section. Implicit links (such as the association between two parallel texts, or that between a text and its interpretation) may be encoded using the linking attributes discussed in section \textit{\hyperref[xatts]{8.3. Special kinds of Linking}}.
\subsection[{Simple Cross References}]{Simple Cross References}\label{ptrs}\par
A cross reference from one point within a single document to another can be encoded using either of the following elements: \par
The difference between these two elements is that \texttt{<ptr>} is an empty element, simply marking a point from which a link is to be made, whereas \texttt{<ref>} may contain some text as well — typically the text of the cross-reference itself. The \texttt{<ptr>} element would be used for a cross reference which is to be indicated by some non-verbal means such as a symbol or icon, or in an electronic text by a button. It is also useful in document production systems, where the formatter can generate the correct verbal form of the cross reference.\par
The following two forms, for example, are logically equivalent (assuming we have documented somewhere the exact verbal form of cross references represented by \texttt{<ptr>} elements): \par\bgroup\exampleFont \begin{shaded}\noindent\mbox{}See especially {<\textbf{ref}\hspace*{6pt}{target}="{\#SEC12}">}section 12 on page\mbox{}\newline 
 34{</\textbf{ref}>}.\end{shaded}\egroup\par \noindent  \par\bgroup\exampleFont \begin{shaded}\noindent\mbox{}See especially {<\textbf{ptr}\hspace*{6pt}{target}="{\#SEC12}"/>}.\end{shaded}\egroup\par \noindent  The value of the \textit{@target} attribute must have been used as the identifier of some other element within the current document. This implies that the passage or phrase being pointed at must bear an identifier, and must therefore be tagged as an element of some kind. In the following example, the cross reference is to a \texttt{<div>} element: \par\bgroup\exampleFont \begin{shaded}\noindent\mbox{} ...\mbox{}\newline 
 see especially {<\textbf{ptr}\hspace*{6pt}{target}="{\#SEC12}"/>}.\mbox{}\newline 
 ...\mbox{}\newline 
\mbox{}\newline 
{<\textbf{div}\hspace*{6pt}{xml:id}="{SEC12}">}\mbox{}\newline 
\hspace*{6pt}{<\textbf{head}>}Concerning Identifiers{</\textbf{head}>}\mbox{}\newline 
\textit{<!-- ... -->}\mbox{}\newline 
{</\textbf{div}>}\end{shaded}\egroup\par \noindent \par
Because the \textit{@xml:id} attribute is global, any element in a document may be pointed to in this way. In the following example, a paragraph has been given an identifier so that it may be pointed at: \par\bgroup\exampleFont \begin{shaded}\noindent\mbox{} ...\mbox{}\newline 
 this is discussed in {<\textbf{ref}\hspace*{6pt}{target}="{\#pspec}">}the paragraph on links{</\textbf{ref}>}\mbox{}\newline 
 ...\mbox{}\newline 
\mbox{}\newline 
{<\textbf{p}\hspace*{6pt}{xml:id}="{pspec}">}Links may be made to any kind of element\mbox{}\newline 
 ...{</\textbf{p}>}\end{shaded}\egroup\par \par
Sometimes the target of a cross reference does not correspond with any particular feature of a text, and so may not be tagged as an element of some kind. If the desired target is simply a point in the current document, the easiest way to mark it is by introducing an \texttt{<anchor>} element at the appropriate spot. If the target is some sequence of words not otherwise tagged, the \texttt{<seg>} element may be introduced to mark them. These two elements are described as follows: \par
In the following (imaginary) example, \texttt{<ref>} elements have been used to represent points in this text which are to be linked in some way to other parts of it; in the first case to a point, and in the second, to a sequence of words: \par\bgroup\exampleFont \begin{shaded}\noindent\mbox{} Returning to {<\textbf{ref}\hspace*{6pt}{target}="{\#ABCD}">}the point where I dozed\mbox{}\newline 
 off{</\textbf{ref}>}, I noticed that {<\textbf{ref}\hspace*{6pt}{target}="{\#EFGH}">}three\mbox{}\newline 
 words{</\textbf{ref}>} had been circled in red by a previous reader\end{shaded}\egroup\par \par
This encoding requires that elements with the specified identifiers (\textit{@ABCD} and \textit{@EFGH} in this example) are to be found somewhere else in the current document. Assuming that no element already exists to carry these identifiers, the \texttt{<anchor>} and \texttt{<seg>} elements may be used: \par\bgroup\exampleFont \begin{shaded}\noindent\mbox{} .... {<\textbf{anchor}\hspace*{6pt}{type}="{bookmark}"\hspace*{6pt}{xml:id}="{ABCD}"/>} ....\mbox{}\newline 
 ....{<\textbf{seg}\hspace*{6pt}{type}="{target}"\hspace*{6pt}{xml:id}="{EFGH}">} ... {</\textbf{seg}>} ...\end{shaded}\egroup\par \par
The \textit{@type} attribute should be used (as above) to distinguish amongst different purposes for which these general purpose elements might be used in a text. Some other uses are discussed in section \textit{\hyperref[xatts]{8.3. Special kinds of Linking}} below.
\subsection[{Pointing to other documents}]{Pointing to other documents}\label{xptrs}\par
So far, we have shown how the elements \texttt{<ptr>} and \texttt{<ref>} may be used for cross-references or links whose targets occur within the same document as their source. However, the same elements may also be used to refer to elements in any other XML document or resource, such as a document on the web, or a database component. This is possible because the value of the \textit{@target} attribute may be any valid \textit{universal resource indicator} (URI). A full definition of this term, defined by the W3C (the consortium which manages the development and maintenance of the World Wide Web), is beyond the scope of this tutorial: however, the most frequently encountered version of a URI is the familiar ‘URL’ used to indicate a web page, such as \texttt{http://www.tei-c.org/index.xml}.\par
A URL may reference a web page or just a part of one, for example \texttt{http://www.tei-c.org/index.xml\#SEC2}. The sharp sign indicates that what follows it is the identifier of an element to be located within the XML document identified by what precedes it: this example will therefore locate an element which has an \textit{@xml:id} attribute value of ‘SEC2’ within the document retrieved from \texttt{http://www.tei-c.org/index.xml}. In the examples we have discussed so far, the part to the left of the sharp sign has been omitted: this is understood to mean that the referenced element is to be located within the current document.\par
Within a URL, parts of an XML document can be specified by means of other more sophisticated mechanisms, using a special language called Xpath, also defined by the W3C. This is particularly useful where the elements to be linked to do not bear identifiers and must therefore be located by some other means. A full specification of the language is well beyond the scope of this document; here we provide only a flavour of its power. \par
In the XPath language, locations are defined as a series of \textit{steps}, each one identifying some part of the document, often in terms of the locations identified by the previous step. For example, you would point to the third sentence of the second paragraph of chapter two by selecting chapter two in the first step, the second paragraph in the second step, and the third sentence in the last step. A step can be defined in terms of the document tree itself, using such concepts as ‘parent’, ‘descendent’, ‘preceding’, etc. or, more loosely, in terms of text patterns, word or character positions.
\subsection[{Special kinds of Linking}]{Special kinds of Linking}\label{xatts}\par
The following special purpose \textit{linking} attributes are defined for every element in the TEI Lite scheme: \begin{description}

\item[{\textit{@ana}}]links an element with its interpretation.
\item[{\textit{@corresp}}]links an element with one or more other corresponding elements.
\item[{\textit{@next}}]links an element to the next element in an aggregate.
\item[{\textit{@prev}}]links an element to the previous element in an aggregate.
\end{description} \par
The \textit{@ana} (analysis) attribute is intended for use where a set of abstract analyses or interpretations have been defined somewhere within a document, as further discussed in section \textit{\hyperref[U5-anal]{15. Interpretation and Analysis}}. For example, a linguistic analysis of the sentence ‘John loves Nancy’ might be encoded as follows: \par\bgroup\exampleFont \begin{shaded}\noindent\mbox{}{<\textbf{seg}\hspace*{6pt}{ana}="{SVO}"\hspace*{6pt}{type}="{sentence}">}\mbox{}\newline 
\hspace*{6pt}{<\textbf{seg}\hspace*{6pt}{ana}="{\#NP1}"\hspace*{6pt}{type}="{lex}">}John{</\textbf{seg}>}\mbox{}\newline 
\hspace*{6pt}{<\textbf{seg}\hspace*{6pt}{ana}="{\#VVI}"\hspace*{6pt}{type}="{lex}">}loves{</\textbf{seg}>}\mbox{}\newline 
\hspace*{6pt}{<\textbf{seg}\hspace*{6pt}{ana}="{\#NP1}"\hspace*{6pt}{type}="{lex}">}Nancy{</\textbf{seg}>}\mbox{}\newline 
{</\textbf{seg}>}\end{shaded}\egroup\par \noindent  This encoding implies the existence elsewhere in the document of elements with identifiers ‘SVO’, ‘NP1’, and ‘VV1’ where the significance of these particular codes is explained. Note the use of the \texttt{<seg>} element to mark particular components of the analysis, distinguished by the \textit{@type} attribute.\par
The \textit{@corresp} (corresponding) attribute provides a simple way of representing some form of correspondence between two elements in a text. For example, in a multilingual text, it may be used to link translation equivalents, as in the following example \par\bgroup\exampleFont \begin{shaded}\noindent\mbox{}{<\textbf{seg}\hspace*{6pt}{corresp}="{\#EN1}"\hspace*{6pt}{xml:id}="{FR1}"\mbox{}\newline 
\hspace*{6pt}{xml:lang}="{fra}">}Jean aime Nancy{</\textbf{seg}>}\mbox{}\newline 
{<\textbf{seg}\hspace*{6pt}{corresp}="{\#FR1}"\hspace*{6pt}{xml:id}="{EN1}"\mbox{}\newline 
\hspace*{6pt}{xml:lang}="{en}">}John loves Nancy{</\textbf{seg}>}\end{shaded}\egroup\par \par
The same mechanism may be used for a variety of purposes. In the following example, it has been used to represent anaphoric correspondences between ‘the show’ and ‘Shirley’, and between ‘NBC’ and ‘the network’: \par\bgroup\exampleFont \begin{shaded}\noindent\mbox{}{<\textbf{p}>}\mbox{}\newline 
\hspace*{6pt}{<\textbf{title}\hspace*{6pt}{xml:id}="{shirley}">}Shirley{</\textbf{title}>}, which made\mbox{}\newline 
 its Friday night debut only a month ago, was\mbox{}\newline 
 not listed on {<\textbf{name}\hspace*{6pt}{xml:id}="{nbc}">}NBC{</\textbf{name}>}'s new schedule,\mbox{}\newline 
 although {<\textbf{seg}\hspace*{6pt}{corresp}="{\#nbc}"\hspace*{6pt}{xml:id}="{network}">}the network{</\textbf{seg}>}\mbox{}\newline 
 says {<\textbf{seg}\hspace*{6pt}{corresp}="{\#shirley}"\hspace*{6pt}{xml:id}="{show}">}the show{</\textbf{seg}>}\mbox{}\newline 
 still is being considered.\mbox{}\newline 
{</\textbf{p}>}\end{shaded}\egroup\par \par
The \textit{@next} and \textit{@prev} attributes provide a simple way of linking together the components of a discontinuous element, as in the following example: \par\bgroup\exampleFont \begin{shaded}\noindent\mbox{}{<\textbf{q}\hspace*{6pt}{next}="{\#Q1b}"\hspace*{6pt}{xml:id}="{Q1a}">}Who-e debel you?{</\textbf{q}>}\mbox{}\newline 
 — he at last said — {<\textbf{q}\hspace*{6pt}{prev}="{\#Q1a}"\hspace*{6pt}{xml:id}="{Q1b}">}you no speak-e,\mbox{}\newline 
 damme, I kill-e.{</\textbf{q}>} And so saying,\mbox{}\newline 
 the lighted tomahawk began flourishing\mbox{}\newline 
 about me in the dark.\end{shaded}\egroup\par 
\section[{Editorial  Interventions}]{Editorial Interventions}\label{U5-edit1}\par
The process of encoding an electronic text has much in common with the process of editing a manuscript or other text for printed publication. In either case a conscientious editor may wish to record both the original state of the source and any editorial correction or other change made in it. The elements discussed in this and the next section provide some facilities for meeting these needs.
\subsection[{Correction and Normalization}]{Correction and Normalization}\par
The following elements may be used to mark \textit{correction}, that is editorial changes introduced where the editor believes the original to be erroneous: \par
The following elements may be used to mark \textit{normalization}, that is editorial changes introduced for the sake of consistency or modernization of a text: \par
As an example, consider this extract from the quarto printing of Shakespeare's \textit{Henry V}. \par\hfill\bgroup\exampleFont\vskip 10pt\begin{shaded}
\obeyspaces  ... for his nose was as sharp as a pen and a table of green\newline
feelds\end{shaded}
\par\egroup 
\par
A modern editor might wish to make a number of interventions here, specifically to modernize (or normalise) the Elizabethan spellings of \textit{a'} and \textit{feelds} for \textit{he} and \textit{fields} respectively. He or she might also want to emend \textit{table} to \textit{babbl'd}, following an editorial tradition that goes back to the 18th century Shakesperean scholar Theobald. The following encoding would then be appropriate: \par\bgroup\exampleFont \begin{shaded}\noindent\mbox{}... for his nose was as sharp as a pen and {<\textbf{reg}>}he{</\textbf{reg}>}\mbox{}\newline 
{<\textbf{corr}\hspace*{6pt}{resp}="{\#Theobald}">}babbl'd{</\textbf{corr}>} of green\mbox{}\newline 
\mbox{}\newline 
{<\textbf{reg}>}fields{</\textbf{reg}>}\end{shaded}\egroup\par \par
A more conservative or source-oriented editor, however, might want to retain the original, but at the same time signal that some of the readings it contains are in some sense anomalous: \par\bgroup\exampleFont \begin{shaded}\noindent\mbox{}... for his nose was as sharp as a pen and {<\textbf{orig}>}a{</\textbf{orig}>}\mbox{}\newline 
{<\textbf{sic}>}table{</\textbf{sic}>} of green\mbox{}\newline 
\mbox{}\newline 
{<\textbf{orig}>}feelds{</\textbf{orig}>}\end{shaded}\egroup\par \par
Finally, a modern digital editor may decide to combine both possibilities in a single composite text, using the \texttt{<choice>} element.  This allows an editor to mark where alternative readings are possible: \par\bgroup\exampleFont \begin{shaded}\noindent\mbox{}... for his nose was\mbox{}\newline 
 as sharp as a pen and\mbox{}\newline 
{<\textbf{choice}>}\mbox{}\newline 
\hspace*{6pt}{<\textbf{orig}>}a{</\textbf{orig}>}\mbox{}\newline 
\hspace*{6pt}{<\textbf{reg}>}he{</\textbf{reg}>}\mbox{}\newline 
{</\textbf{choice}>}\mbox{}\newline 
{<\textbf{choice}>}\mbox{}\newline 
\hspace*{6pt}{<\textbf{corr}\hspace*{6pt}{resp}="{\#Theobald}">}babbl'd{</\textbf{corr}>}\mbox{}\newline 
\hspace*{6pt}{<\textbf{sic}>}table{</\textbf{sic}>}\mbox{}\newline 
{</\textbf{choice}>}\mbox{}\newline 
 of green\mbox{}\newline 
\mbox{}\newline 
{<\textbf{choice}>}\mbox{}\newline 
\hspace*{6pt}{<\textbf{orig}>}feelds{</\textbf{orig}>}\mbox{}\newline 
\hspace*{6pt}{<\textbf{reg}>}fields{</\textbf{reg}>}\mbox{}\newline 
{</\textbf{choice}>}\end{shaded}\egroup\par 
\subsection[{Omissions, Deletions, and  Additions}]{Omissions, Deletions, and Additions}\label{U5-edit2}\par
In addition to correcting or normalizing words and phrases, editors and transcribers may also supply missing material, omit material, or transcribe material deleted or crossed out in the source. In addition, some material may be particularly hard to transcribe because it is hard to make out on the page. The following elements may be used to record such phenomena: \par
These elements may be used to record changes made by an editor, by the transcriber, or (in manuscript material) by the author or scribe. For example, if the source for an electronic text read \par\hfill\bgroup\exampleFont\vskip 10pt\begin{shaded}
\obeyspaces The following elements are provided for for simple editorial\newline
interventions.\end{shaded}
\par\egroup 
 then it might be felt desirable to correct the obvious error, but at the same time to record the deletion of the superfluous second \textit{for}, thus: \par\bgroup\exampleFont \begin{shaded}\noindent\mbox{}The following elements are provided for\mbox{}\newline 
{<\textbf{del}\hspace*{6pt}{resp}="{\#LB}">}for{</\textbf{del}>} simple editorial interventions.\end{shaded}\egroup\par \noindent  The attribute value \texttt{LB} on the \textit{@resp} attribute indicates that ‘LB’ corrected the duplication of \textit{for}.\par
If the source read\par\hfill\bgroup\exampleFont\vskip 10pt\begin{shaded}
\obeyspaces The following elements provided for\newline
simple editorial interventions.\end{shaded}
\par\egroup 
 (i.e. if the verb had been inadvertently dropped) then the corrected text might read: \par\bgroup\exampleFont \begin{shaded}\noindent\mbox{}The following elements {<\textbf{add}\hspace*{6pt}{resp}="{\#LB}">}are{</\textbf{add}>} provided for\mbox{}\newline 
 simple editorial interventions.\end{shaded}\egroup\par \par
These elements are not limited to changes made by an editor; they can also be used to record authorial changes in manuscripts. A manuscript in which the author has first written ‘How it galls me, what a galling shadow’, then crossed out the word \textit{galls} and inserted \textit{dogs} might be encoded thus: \par\bgroup\exampleFont \begin{shaded}\noindent\mbox{}How it {<\textbf{del}\hspace*{6pt}{hand}="{DHL}"\hspace*{6pt}{type}="{overstrike}">}galls{</\textbf{del}>}\mbox{}\newline 
{<\textbf{add}\hspace*{6pt}{hand}="{DHL}"\hspace*{6pt}{place}="{supralinear}">}dogs{</\textbf{add}>} me,\mbox{}\newline 
 what a galling shadow\end{shaded}\egroup\par \par
Similarly, the \texttt{<unclear>} and \texttt{<gap>} elements may be used together to indicate the omission of illegible material; the following example also shows the use of \texttt{<add>} for a conjectural emendation: \par\bgroup\exampleFont \begin{shaded}\noindent\mbox{}One hundred \& twenty good regulars joined to me\mbox{}\newline 
{<\textbf{unclear}>}\mbox{}\newline 
\hspace*{6pt}{<\textbf{gap}\hspace*{6pt}{reason}="{indecipherable}"/>}\mbox{}\newline 
{</\textbf{unclear}>}\mbox{}\newline 
 \& instantly, would aid me signally {<\textbf{add}\hspace*{6pt}{hand}="{ed}">}in?{</\textbf{add}>}\mbox{}\newline 
 an enterprise against Wilmington.\end{shaded}\egroup\par \par
The \texttt{<del>} element marks material which is transcribed as part of the electronic text despite being marked as deleted, while \texttt{<gap>} marks the location of material which is omitted from the electronic text, whether it is legible or not. A language corpus, for example, might omit long quotations in foreign languages: \par\bgroup\exampleFont \begin{shaded}\noindent\mbox{}{<\textbf{p}>} ... An example of a list appearing in a fief ledger of\mbox{}\newline 
{<\textbf{name}\hspace*{6pt}{type}="{place}">}Koldinghus{</\textbf{name}>}\mbox{}\newline 
\hspace*{6pt}{<\textbf{date}>}1611/12{</\textbf{date}>}\mbox{}\newline 
 is given below. It shows cash income from a sale of\mbox{}\newline 
 honey.{</\textbf{p}>}\mbox{}\newline 
{<\textbf{gap}>}\mbox{}\newline 
\hspace*{6pt}{<\textbf{desc}>}quotation from ledger (in Danish){</\textbf{desc}>}\mbox{}\newline 
{</\textbf{gap}>}\mbox{}\newline 
{<\textbf{p}>}A description of the overall structure of the account is\mbox{}\newline 
 once again ... {</\textbf{p}>}\end{shaded}\egroup\par \par
Other corpora (particular those constructed before the widespread use of scanners) systematically omit figures and mathematics: \par\bgroup\exampleFont \begin{shaded}\noindent\mbox{}{<\textbf{p}>}At the bottom of your screen below the mode line is the\mbox{}\newline 
{<\textbf{term}>}minibuffer{</\textbf{term}>}. This is the area where Emacs\mbox{}\newline 
 echoes the commands you enter and where you specify\mbox{}\newline 
 filenames for Emacs to find, values for search and replace,\mbox{}\newline 
 and so on.\mbox{}\newline 
{<\textbf{gap}\hspace*{6pt}{reason}="{graphic}">}\mbox{}\newline 
\hspace*{6pt}\hspace*{6pt}{<\textbf{desc}>}diagram of Emacs screen{</\textbf{desc}>}\mbox{}\newline 
\hspace*{6pt}{</\textbf{gap}>}\mbox{}\newline 
{</\textbf{p}>}\end{shaded}\egroup\par 
\subsection[{Abbreviations and their Expansion}]{Abbreviations and their Expansion}\par
Like names, dates, and numbers, abbreviations may be transcribed as they stand or expanded; they may be left unmarked, or encoded using the following elements: \par
The \texttt{<abbr>} element is useful as a means of distinguishing semi-lexical items such as acronyms or jargon: \par\bgroup\exampleFont \begin{shaded}\noindent\mbox{}We can sum up the above discussion as follows: the identity of a\mbox{}\newline 
{<\textbf{abbr}>}CC{</\textbf{abbr}>} is defined by that calibration of values which\mbox{}\newline 
 motivates the elements of its {<\textbf{abbr}>}GSP{</\textbf{abbr}>};\end{shaded}\egroup\par \noindent  \par\bgroup\exampleFont \begin{shaded}\noindent\mbox{}Every manufacturer of {<\textbf{abbr}>}3GL{</\textbf{abbr}>} or {<\textbf{abbr}>}4GL{</\textbf{abbr}>}\mbox{}\newline 
 languages is currently nailing on {<\textbf{abbr}>}OOP{</\textbf{abbr}>} extensions\end{shaded}\egroup\par \noindent \par
The \textit{@type} attribute may be used to distinguish types of abbreviation by their function.\par
The \texttt{<expan>} element is used to mark an expansion supplied by an encoder. This element is particularly useful in the transcription of manuscript materials. For example, the character p with a bar through its descender as a conventional representation for the word ‘per’ is commonly encountered in Medieval European manuscripts. An encoder may choose to expand this as follows: \par\bgroup\exampleFont \begin{shaded}\noindent\mbox{}{<\textbf{expan}>}per{</\textbf{expan}>}\end{shaded}\egroup\par \noindent \par
The expansion corresponding with an abbreviated form may not always contain the same letters as the abbreviation. Where it does, however, common editorial practice is to italicize or otherwise signal which letters have been supplied. The \texttt{<expan>} element should not be used for this purpose since its function is to indicate an expanded form, not a part of one. For example, consider the common abbreviation ‘wt’ (for ‘with’) found in medieval texts. In a modern edition, an editor might wish to represent this as ‘w{\itshape i}t{\itshape h}’, italicising the letters not found in the source. An appropriate encoding for this purpose would be \par\bgroup\exampleFont \begin{shaded}\noindent\mbox{}{<\textbf{expan}>}w{<\textbf{hi}>}i{</\textbf{hi}>}t{<\textbf{hi}>}h{</\textbf{hi}>}\mbox{}\newline 
{</\textbf{expan}>}\end{shaded}\egroup\par \noindent \par
To record both an abbreviation and its expansion, the \texttt{<choice>} element mentioned above may be used to group the abbreviated form with its proposed expansion: \par\bgroup\exampleFont \begin{shaded}\noindent\mbox{}{<\textbf{choice}>}\mbox{}\newline 
\hspace*{6pt}{<\textbf{abbr}>}wt{</\textbf{abbr}>}\mbox{}\newline 
\hspace*{6pt}{<\textbf{expan}>}with{</\textbf{expan}>}\mbox{}\newline 
{</\textbf{choice}>}\end{shaded}\egroup\par \noindent 
\section[{Names, Dates, and  Numbers}]{Names, Dates, and Numbers}\label{U5-names}\par
The TEI scheme defines elements for a large number of ‘data-like’ features which may appear almost anywhere within almost any kind of text. These features may be of particular interest in a range of disciplines; they all relate to objects external to the text itself, such as the names of persons and places, numbers and dates. They also pose particular problems for many natural language processing (NLP) applications because of the variety of ways in which they may be presented within a text. The elements described here, by making such features explicit, reduce the complexity of processing texts containing them.
\subsection[{Names and Referring Strings}]{Names and Referring Strings}\label{nomen}\par
A \textit{referring string} is a phrase which refers to some person, place, object, etc. Two elements are provided to mark such strings: \par
The \textit{@type} attribute is used to distinguish amongst (for example) names of persons, places and organizations, where this is possible: \par\bgroup\exampleFont \begin{shaded}\noindent\mbox{}{<\textbf{q}>}My dear {<\textbf{rs}\hspace*{6pt}{type}="{person}">}Mr. Bennet{</\textbf{rs}>}, {</\textbf{q}>}\mbox{}\newline 
 said his lady to him one day, \mbox{}\newline 
{<\textbf{q}>}have you heard\mbox{}\newline 
 that {<\textbf{rs}\hspace*{6pt}{type}="{place}">}Netherfield Park{</\textbf{rs}>} is let\mbox{}\newline 
 at last?{</\textbf{q}>}\end{shaded}\egroup\par \noindent  \par\bgroup\exampleFont \begin{shaded}\noindent\mbox{}It being one of the principles of the\mbox{}\newline 
{<\textbf{rs}\hspace*{6pt}{type}="{organization}">}Circumlocution Office{</\textbf{rs}>} never,\mbox{}\newline 
 on any account whatsoever, to give a straightforward answer,\mbox{}\newline 
{<\textbf{rs}\hspace*{6pt}{type}="{person}">}Mr Barnacle{</\textbf{rs}>} said, \mbox{}\newline 
{<\textbf{q}>}Possibly.{</\textbf{q}>}\end{shaded}\egroup\par \par
As the following example shows, the \texttt{<rs>} element may be used for any reference to a person, place, etc, not necessarily one in the form of a proper noun or noun phrase. \par\bgroup\exampleFont \begin{shaded}\noindent\mbox{}{<\textbf{q}>}My dear {<\textbf{rs}\hspace*{6pt}{type}="{person}">}Mr. Bennet{</\textbf{rs}>},{</\textbf{q}>}\mbox{}\newline 
 said {<\textbf{rs}\hspace*{6pt}{type}="{person}">}his lady{</\textbf{rs}>} to him\mbox{}\newline 
 one day...\end{shaded}\egroup\par \par
The \texttt{<name>} element by contrast is provided for the special case of referencing strings which consist only of proper nouns; it may be used synonymously with the \texttt{<rs>} element, or nested within it if a referring string contains a mixture of common and proper nouns.\par
Simply tagging something as a name is rarely enough to enable automatic processing of personal names into the canonical forms usually required for reference purposes. The name as it appears in the text may be inconsistently spelled, partial, or vague. Moreover, name prefixes such as \textit{van} or \textit{de la}, may or may not be included as part of the reference form of a name, depending on the language and country of origin of the bearer.\par
The \textit{@key} attribute provides an alternative normalized identifier for the object being named, like a database record key. It may thus be useful as a means of gathering together all references to the same individual or location scattered throughout a document: \par\bgroup\exampleFont \begin{shaded}\noindent\mbox{}{<\textbf{q}>}My dear {<\textbf{rs}\hspace*{6pt}{key}="{BENM1}"\hspace*{6pt}{type}="{person}">}Mr. Bennet{</\textbf{rs}>},\mbox{}\newline 
{</\textbf{q}>} said {<\textbf{rs}\hspace*{6pt}{key}="{BENM2}"\hspace*{6pt}{type}="{person}">}his lady{</\textbf{rs}>}\mbox{}\newline 
 to him one day, \mbox{}\newline 
{<\textbf{q}>}have you heard that\mbox{}\newline 
{<\textbf{rs}\hspace*{6pt}{key}="{NETP1}"\hspace*{6pt}{type}="{place}">}Netherfield Park{</\textbf{rs}>}\mbox{}\newline 
 is let at last?{</\textbf{q}>}\end{shaded}\egroup\par \par
This use should be distinguished from the case of the \texttt{<reg>} (regularization) element, which provides a means of marking the standard form of a referencing string as demonstrated below: \par\bgroup\exampleFont \begin{shaded}\noindent\mbox{}{<\textbf{name}\hspace*{6pt}{key}="{WADLM1}"\hspace*{6pt}{type}="{person}">}\mbox{}\newline 
\hspace*{6pt}{<\textbf{choice}>}\mbox{}\newline 
\hspace*{6pt}\hspace*{6pt}{<\textbf{sic}>}Walter de la Mare{</\textbf{sic}>}\mbox{}\newline 
\hspace*{6pt}\hspace*{6pt}{<\textbf{reg}>}de la Mare, Walter{</\textbf{reg}>}\mbox{}\newline 
\hspace*{6pt}{</\textbf{choice}>}\mbox{}\newline 
{</\textbf{name}>} was born at\mbox{}\newline 
{<\textbf{name}\hspace*{6pt}{key}="{Ch1}"\hspace*{6pt}{type}="{place}">}Charlton{</\textbf{name}>}, in\mbox{}\newline 
{<\textbf{name}\hspace*{6pt}{key}="{KT1}"\hspace*{6pt}{type}="{county}">}Kent{</\textbf{name}>}, in 1873.\end{shaded}\egroup\par \par
The \texttt{<index>} element discussed in \url{indexing} may be more appropriate if the function of the regularization is to provide a consistent index: \par\bgroup\exampleFont \begin{shaded}\noindent\mbox{}{<\textbf{p}>}\mbox{}\newline 
\hspace*{6pt}{<\textbf{name}\hspace*{6pt}{type}="{place}">}Montaillou{</\textbf{name}>} is not a large parish.\mbox{}\newline 
 At the time of the events which led to\mbox{}\newline 
{<\textbf{name}\hspace*{6pt}{type}="{person}">}Fournier{</\textbf{name}>}'s {<\textbf{index}>}\mbox{}\newline 
\hspace*{6pt}\hspace*{6pt}{<\textbf{term}>}Benedict XII, Pope of Avignon (Jacques Fournier){</\textbf{term}>}\mbox{}\newline 
\hspace*{6pt}{</\textbf{index}>}\mbox{}\newline 
 investigations, the local population consisted of between 200 and 250 inhabitants.\mbox{}\newline 
{</\textbf{p}>}\end{shaded}\egroup\par \noindent  Although adequate for many simple applications, these methods have two inconveniences: if the name occurs many times, then its regularised form must be repeated many times; and the burden of additional XML markup in the body of the text may be inconvenient to maintain and complex to process. For applications such as onomastics, relating to persons or places named rather than the name itself, or wherever a detailed analysis of the component parts of a name is needed, the full TEI Guidelines provide a range of other solutions.
\subsection[{Dates and Times}]{Dates and Times}\par
Tags for the more detailed encoding of times and dates include the following: \par
The \textit{@value} attribute specifies a normalized form for the date or time, using one of the standard formats defined by ISO 8601. Partial dates or times (e.g. ‘1990’, ‘September 1990’, ‘twelvish’) can be expressed by omitting a part of the value supplied, as in the following examples: \par\bgroup\exampleFont \begin{shaded}\noindent\mbox{}{<\textbf{date}\hspace*{6pt}{when}="{1980-02-21}">}21 Feb 1980{</\textbf{date}>}\mbox{}\newline 
{<\textbf{date}\hspace*{6pt}{when}="{1990}">}1990{</\textbf{date}>}\mbox{}\newline 
{<\textbf{date}\hspace*{6pt}{when}="{1990-09}">}September 1990{</\textbf{date}>}\mbox{}\newline 
{<\textbf{date}\hspace*{6pt}{when}="{--09}">}September{</\textbf{date}>}\mbox{}\newline 
{<\textbf{date}\hspace*{6pt}{when}="{2001-09-11T12:48:00}">}Sept 11th, 12 minutes before 9 am{</\textbf{date}>}\end{shaded}\egroup\par \noindent Note in the last example the use of a normalized representation for the date string which includes a time: this example could thus equally well be tagged using the \texttt{<time>} element.\par
\par\bgroup\exampleFont \begin{shaded}\noindent\mbox{}Given on the {<\textbf{date}\hspace*{6pt}{when}="{1977-06-12}">}Twelfth Day of June\mbox{}\newline 
 in the Year of Our Lord One Thousand Nine Hundred and\mbox{}\newline 
 Seventy-seven of the Republic the Two Hundredth and first\mbox{}\newline 
 and of the University the Eighty-Sixth.{</\textbf{date}>}\end{shaded}\egroup\par \noindent  \par\bgroup\exampleFont \begin{shaded}\noindent\mbox{}{<\textbf{l}>}specially when it's nine below zero{</\textbf{l}>}\mbox{}\newline 
{<\textbf{l}>}and {<\textbf{time}\hspace*{6pt}{when}="{15:00:00}">}three o'clock in the\mbox{}\newline 
\hspace*{6pt}\hspace*{6pt} afternoon{</\textbf{time}>}\mbox{}\newline 
{</\textbf{l}>}\end{shaded}\egroup\par \noindent 
\subsection[{Numbers }]{Numbers }\par
Numbers can be written with either letters or digits (\texttt{twenty-one}, \texttt{xxi}, and \texttt{21}) and their presentation is language-dependent (e.g. English \textit{5th} becomes Greek \textit{5.}; English \textit{123,456.78} equals French \textit{123.456,78}). In natural-language processing or machine-translation applications, it is often helpful to distinguish them from other, more ‘lexical’ parts of the text. In other applications, the ability to record a number's value in standard notation is important. The \texttt{<num>} element provides this possibility: \par
For example: \par\bgroup\exampleFont \begin{shaded}\noindent\mbox{}{<\textbf{num}\hspace*{6pt}{value}="{33}">}xxxiii{</\textbf{num}>}\mbox{}\newline 
{<\textbf{num}\hspace*{6pt}{type}="{cardinal}"\hspace*{6pt}{value}="{21}">}twenty-one{</\textbf{num}>}\mbox{}\newline 
{<\textbf{num}\hspace*{6pt}{type}="{percentage}"\hspace*{6pt}{value}="{10}">}ten percent{</\textbf{num}>}\mbox{}\newline 
{<\textbf{num}\hspace*{6pt}{type}="{percentage}"\hspace*{6pt}{value}="{10}">}10\%{</\textbf{num}>}\mbox{}\newline 
{<\textbf{num}\hspace*{6pt}{type}="{ordinal}"\hspace*{6pt}{value}="{5}">}5th{</\textbf{num}>}\end{shaded}\egroup\par 
\section[{Lists}]{Lists}\label{U5-lists}\par
The element \texttt{<list>} is used to mark any kind of \textit{list}. A list is a sequence of text items, which may be ordered, unordered, or a glossary list. Each item may be preceded by an item label (in a glossary list, this label is the term being defined): \par
Individual list items are tagged with \texttt{<item>}. The first \texttt{<item>} may optionally be preceded by a \texttt{<head>}, which gives a heading for the list. The numbering of a list may be omitted, indicated using the \textit{@n} attribute on each item, or (rarely) tagged as content using the \texttt{<label>} element. The following are all thus equivalent: \par\bgroup\exampleFont \begin{shaded}\noindent\mbox{}{<\textbf{list}>}\mbox{}\newline 
\hspace*{6pt}{<\textbf{head}>}A short list{</\textbf{head}>}\mbox{}\newline 
\hspace*{6pt}{<\textbf{item}>}First item in list.{</\textbf{item}>}\mbox{}\newline 
\hspace*{6pt}{<\textbf{item}>}Second item in list.{</\textbf{item}>}\mbox{}\newline 
\hspace*{6pt}{<\textbf{item}>}Third item in list.{</\textbf{item}>}\mbox{}\newline 
{</\textbf{list}>}\mbox{}\newline 
{<\textbf{list}>}\mbox{}\newline 
\hspace*{6pt}{<\textbf{head}>}A short list{</\textbf{head}>}\mbox{}\newline 
\hspace*{6pt}{<\textbf{item}\hspace*{6pt}{n}="{1}">}First item in list.{</\textbf{item}>}\mbox{}\newline 
\hspace*{6pt}{<\textbf{item}\hspace*{6pt}{n}="{2}">}Second item in list.{</\textbf{item}>}\mbox{}\newline 
\hspace*{6pt}{<\textbf{item}\hspace*{6pt}{n}="{3}">}Third item in list.{</\textbf{item}>}\mbox{}\newline 
{</\textbf{list}>}\mbox{}\newline 
{<\textbf{list}>}\mbox{}\newline 
\hspace*{6pt}{<\textbf{head}>}A short list{</\textbf{head}>}\mbox{}\newline 
\hspace*{6pt}{<\textbf{label}>}1{</\textbf{label}>}\mbox{}\newline 
\hspace*{6pt}{<\textbf{item}>}First item in list.{</\textbf{item}>}\mbox{}\newline 
\hspace*{6pt}{<\textbf{label}>}2{</\textbf{label}>}\mbox{}\newline 
\hspace*{6pt}{<\textbf{item}>}Second item in list.{</\textbf{item}>}\mbox{}\newline 
\hspace*{6pt}{<\textbf{label}>}3{</\textbf{label}>}\mbox{}\newline 
\hspace*{6pt}{<\textbf{item}>}Third item in list.{</\textbf{item}>}\mbox{}\newline 
{</\textbf{list}>}\end{shaded}\egroup\par \noindent  The styles should not be mixed in the same list.\par
A simple two-column table may be treated as a \textit{glossary list}, tagged \texttt{<list type="gloss">}. Here, each item comprises a \textit{term} and a \textit{gloss}, marked with \texttt{<label>} and \texttt{<item>} respectively. These correspond to the elements \texttt{<term>} and \texttt{<gloss>}, which can occur anywhere in prose text. \par\bgroup\exampleFont \begin{shaded}\noindent\mbox{}{<\textbf{list}\hspace*{6pt}{type}="{gloss}">}\mbox{}\newline 
\hspace*{6pt}{<\textbf{head}>}Vocabulary{</\textbf{head}>}\mbox{}\newline 
\hspace*{6pt}{<\textbf{label}\hspace*{6pt}{xml:lang}="{enm}">}nu{</\textbf{label}>}\mbox{}\newline 
\hspace*{6pt}{<\textbf{item}>}now{</\textbf{item}>}\mbox{}\newline 
\hspace*{6pt}{<\textbf{label}\hspace*{6pt}{xml:lang}="{enm}">}lhude{</\textbf{label}>}\mbox{}\newline 
\hspace*{6pt}{<\textbf{item}>}loudly{</\textbf{item}>}\mbox{}\newline 
\hspace*{6pt}{<\textbf{label}\hspace*{6pt}{xml:lang}="{enm}">}bloweth{</\textbf{label}>}\mbox{}\newline 
\hspace*{6pt}{<\textbf{item}>}blooms{</\textbf{item}>}\mbox{}\newline 
\hspace*{6pt}{<\textbf{label}\hspace*{6pt}{xml:lang}="{enm}">}med{</\textbf{label}>}\mbox{}\newline 
\hspace*{6pt}{<\textbf{item}>}meadow{</\textbf{item}>}\mbox{}\newline 
\hspace*{6pt}{<\textbf{label}\hspace*{6pt}{xml:lang}="{enm}">}wude{</\textbf{label}>}\mbox{}\newline 
\hspace*{6pt}{<\textbf{item}>}wood{</\textbf{item}>}\mbox{}\newline 
\hspace*{6pt}{<\textbf{label}\hspace*{6pt}{xml:lang}="{enm}">}awe{</\textbf{label}>}\mbox{}\newline 
\hspace*{6pt}{<\textbf{item}>}ewe{</\textbf{item}>}\mbox{}\newline 
\hspace*{6pt}{<\textbf{label}\hspace*{6pt}{xml:lang}="{enm}">}lhouth{</\textbf{label}>}\mbox{}\newline 
\hspace*{6pt}{<\textbf{item}>}lows{</\textbf{item}>}\mbox{}\newline 
\hspace*{6pt}{<\textbf{label}\hspace*{6pt}{xml:lang}="{enm}">}sterteth{</\textbf{label}>}\mbox{}\newline 
\hspace*{6pt}{<\textbf{item}>}bounds, frisks{</\textbf{item}>}\mbox{}\newline 
\hspace*{6pt}{<\textbf{label}\hspace*{6pt}{xml:lang}="{enm}">}verteth{</\textbf{label}>}\mbox{}\newline 
\hspace*{6pt}{<\textbf{item}\hspace*{6pt}{xml:lang}="{lat}">}pedit{</\textbf{item}>}\mbox{}\newline 
\hspace*{6pt}{<\textbf{label}\hspace*{6pt}{xml:lang}="{enm}">}murie{</\textbf{label}>}\mbox{}\newline 
\hspace*{6pt}{<\textbf{item}>}merrily{</\textbf{item}>}\mbox{}\newline 
\hspace*{6pt}{<\textbf{label}\hspace*{6pt}{xml:lang}="{enm}">}swik{</\textbf{label}>}\mbox{}\newline 
\hspace*{6pt}{<\textbf{item}>}cease{</\textbf{item}>}\mbox{}\newline 
\hspace*{6pt}{<\textbf{label}\hspace*{6pt}{xml:lang}="{enm}">}naver{</\textbf{label}>}\mbox{}\newline 
\hspace*{6pt}{<\textbf{item}>}never{</\textbf{item}>}\mbox{}\newline 
{</\textbf{list}>}\end{shaded}\egroup\par \par
Where the internal structure of a list item is more complex, it may be preferable to regard the list as a \textit{table}, for which special-purpose tagging is defined below (\textit{\hyperref[U5-tables]{13. Tables}}).\par
Lists of whatever kind can, of course, nest within list items to any depth required. Here, for example, a glossary list contains two items, each of which is itself a simple list: \par\bgroup\exampleFont \begin{shaded}\noindent\mbox{}{<\textbf{list}\hspace*{6pt}{type}="{gloss}">}\mbox{}\newline 
\hspace*{6pt}{<\textbf{label}>}EVIL{</\textbf{label}>}\mbox{}\newline 
\hspace*{6pt}{<\textbf{item}>}\mbox{}\newline 
\hspace*{6pt}\hspace*{6pt}{<\textbf{list}\hspace*{6pt}{type}="{simple}">}\mbox{}\newline 
\hspace*{6pt}\hspace*{6pt}\hspace*{6pt}{<\textbf{item}>}I am cast upon a horrible desolate island, void\mbox{}\newline 
\hspace*{6pt}\hspace*{6pt}\hspace*{6pt}\hspace*{6pt}\hspace*{6pt}\hspace*{6pt} of all hope of recovery.{</\textbf{item}>}\mbox{}\newline 
\hspace*{6pt}\hspace*{6pt}\hspace*{6pt}{<\textbf{item}>}I am singled out and separated as it were from\mbox{}\newline 
\hspace*{6pt}\hspace*{6pt}\hspace*{6pt}\hspace*{6pt}\hspace*{6pt}\hspace*{6pt} all the world to be miserable.{</\textbf{item}>}\mbox{}\newline 
\hspace*{6pt}\hspace*{6pt}\hspace*{6pt}{<\textbf{item}>}I am divided from mankind — a solitaire; one\mbox{}\newline 
\hspace*{6pt}\hspace*{6pt}\hspace*{6pt}\hspace*{6pt}\hspace*{6pt}\hspace*{6pt} banished from human society.{</\textbf{item}>}\mbox{}\newline 
\hspace*{6pt}\hspace*{6pt}{</\textbf{list}>}\mbox{}\newline 
\hspace*{6pt}{</\textbf{item}>}\mbox{}\newline 
\hspace*{6pt}{<\textbf{label}>}GOOD{</\textbf{label}>}\mbox{}\newline 
\hspace*{6pt}{<\textbf{item}>}\mbox{}\newline 
\hspace*{6pt}\hspace*{6pt}{<\textbf{list}\hspace*{6pt}{type}="{simple}">}\mbox{}\newline 
\hspace*{6pt}\hspace*{6pt}\hspace*{6pt}{<\textbf{item}>}But I am alive; and not drowned, as all my\mbox{}\newline 
\hspace*{6pt}\hspace*{6pt}\hspace*{6pt}\hspace*{6pt}\hspace*{6pt}\hspace*{6pt} ship's company were.{</\textbf{item}>}\mbox{}\newline 
\hspace*{6pt}\hspace*{6pt}\hspace*{6pt}{<\textbf{item}>}But I am singled out, too, from all the ship's\mbox{}\newline 
\hspace*{6pt}\hspace*{6pt}\hspace*{6pt}\hspace*{6pt}\hspace*{6pt}\hspace*{6pt} crew, to be spared from death...{</\textbf{item}>}\mbox{}\newline 
\hspace*{6pt}\hspace*{6pt}\hspace*{6pt}{<\textbf{item}>}But I am not starved, and perishing on a barren place,\mbox{}\newline 
\hspace*{6pt}\hspace*{6pt}\hspace*{6pt}\hspace*{6pt}\hspace*{6pt}\hspace*{6pt} affording no sustenances....{</\textbf{item}>}\mbox{}\newline 
\hspace*{6pt}\hspace*{6pt}{</\textbf{list}>}\mbox{}\newline 
\hspace*{6pt}{</\textbf{item}>}\mbox{}\newline 
{</\textbf{list}>}\end{shaded}\egroup\par \par
A list need not necessarily be displayed in list format. For example, \par\bgroup\exampleFont \begin{shaded}\noindent\mbox{}{<\textbf{p}>}On those remote pages it is written that animals are\mbox{}\newline 
 divided into {<\textbf{list}\hspace*{6pt}{rend}="{run-on}">}\mbox{}\newline 
\hspace*{6pt}\hspace*{6pt}{<\textbf{item}\hspace*{6pt}{n}="{a}">}those that belong to the\mbox{}\newline 
\hspace*{6pt}\hspace*{6pt}\hspace*{6pt}\hspace*{6pt} Emperor,{</\textbf{item}>}\mbox{}\newline 
\hspace*{6pt}\hspace*{6pt}{<\textbf{item}\hspace*{6pt}{n}="{b}">} embalmed ones, {</\textbf{item}>}\mbox{}\newline 
\hspace*{6pt}\hspace*{6pt}{<\textbf{item}\hspace*{6pt}{n}="{c}">} those\mbox{}\newline 
\hspace*{6pt}\hspace*{6pt}\hspace*{6pt}\hspace*{6pt} that are trained, {</\textbf{item}>}\mbox{}\newline 
\hspace*{6pt}\hspace*{6pt}{<\textbf{item}\hspace*{6pt}{n}="{d}">} suckling pigs, {</\textbf{item}>}\mbox{}\newline 
\hspace*{6pt}\hspace*{6pt}{<\textbf{item}\hspace*{6pt}{n}="{e}">}mermaids, {</\textbf{item}>}\mbox{}\newline 
\hspace*{6pt}\hspace*{6pt}{<\textbf{item}\hspace*{6pt}{n}="{f}">} fabulous ones, {</\textbf{item}>}\mbox{}\newline 
\hspace*{6pt}\hspace*{6pt}{<\textbf{item}\hspace*{6pt}{n}="{g}">} stray\mbox{}\newline 
\hspace*{6pt}\hspace*{6pt}\hspace*{6pt}\hspace*{6pt} dogs, {</\textbf{item}>}\mbox{}\newline 
\hspace*{6pt}\hspace*{6pt}{<\textbf{item}\hspace*{6pt}{n}="{h}">} those that are included in this\mbox{}\newline 
\hspace*{6pt}\hspace*{6pt}\hspace*{6pt}\hspace*{6pt} classification, {</\textbf{item}>}\mbox{}\newline 
\hspace*{6pt}\hspace*{6pt}{<\textbf{item}\hspace*{6pt}{n}="{i}">} those that tremble as if they\mbox{}\newline 
\hspace*{6pt}\hspace*{6pt}\hspace*{6pt}\hspace*{6pt} were mad, {</\textbf{item}>}\mbox{}\newline 
\hspace*{6pt}\hspace*{6pt}{<\textbf{item}\hspace*{6pt}{n}="{j}">} innumerable ones, {</\textbf{item}>}\mbox{}\newline 
\hspace*{6pt}\hspace*{6pt}{<\textbf{item}\hspace*{6pt}{n}="{k}">} those\mbox{}\newline 
\hspace*{6pt}\hspace*{6pt}\hspace*{6pt}\hspace*{6pt} drawn with a very fine camel's-hair brush, {</\textbf{item}>}\mbox{}\newline 
\hspace*{6pt}\hspace*{6pt}{<\textbf{item}\hspace*{6pt}{n}="{l}">}others, {</\textbf{item}>}\mbox{}\newline 
\hspace*{6pt}\hspace*{6pt}{<\textbf{item}\hspace*{6pt}{n}="{m}">} those that have just broken a flower\mbox{}\newline 
\hspace*{6pt}\hspace*{6pt}\hspace*{6pt}\hspace*{6pt} vase, {</\textbf{item}>}\mbox{}\newline 
\hspace*{6pt}\hspace*{6pt}{<\textbf{item}\hspace*{6pt}{n}="{n}">} those that resemble flies from a\mbox{}\newline 
\hspace*{6pt}\hspace*{6pt}\hspace*{6pt}\hspace*{6pt} distance.{</\textbf{item}>}\mbox{}\newline 
\hspace*{6pt}{</\textbf{list}>}\mbox{}\newline 
{</\textbf{p}>}\end{shaded}\egroup\par \par
Lists of bibliographic items should be tagged using the \texttt{<listBibl>} element, described in the next section.
\section[{Bibliographic Citations}]{Bibliographic Citations}\label{U5-bibls}\par
It is often useful to distinguish bibliographic citations where they occur within texts being transcribed for research, if only so that they will be properly formatted when the text is printed out. The element \texttt{<bibl>} is provided for this purpose. Where the components of a bibliographic reference are to be distinguished, the following elements may be used as appropriate. It is generally useful to mark at least those parts (such as the titles of articles, books, and journals) which will need special formatting. The other elements are provided for cases where particular interest attaches to such details. \par
For example, the following editorial note might be transcribed as shown: 
\begin{quote}He was a member of Parliament for Warwickshire in 1445, and died March 14, 1470 (according to Kittredge, \textit{Harvard Studies} 5. 88ff).\end{quote}
 \par\bgroup\exampleFont \begin{shaded}\noindent\mbox{}He was a member of Parliament for Warwickshire in 1445, and died\mbox{}\newline 
 March 14, 1470 (according to {<\textbf{bibl}>}\mbox{}\newline 
\hspace*{6pt}{<\textbf{author}>}Kittredge{</\textbf{author}>},\mbox{}\newline 
{<\textbf{title}>}Harvard Studies{</\textbf{title}>}\mbox{}\newline 
\hspace*{6pt}{<\textbf{biblScope}>}5. 88ff{</\textbf{biblScope}>}\mbox{}\newline 
{</\textbf{bibl}>}).\end{shaded}\egroup\par \par
For lists of bibliographic citations, the \texttt{<listBibl>} element should be used; it may contain a series of \texttt{<bibl>} elements.
\section[{Tables}]{Tables}\label{U5-tables}\par
Tables represent a challenge for any text processing system, but simple tables, at least, appear in so many texts that even in the simplified TEI tag set presented here, markup for tables is necessary. The following elements are provided for this purpose: \par
For example, Defoe uses mortality tables like the following in the \textit{Journal of the Plague Year} to show the rise and ebb of the epidemic:\par\bgroup\exampleFont \begin{shaded}\noindent\mbox{}{<\textbf{p}>}It was indeed coming on amain, for the burials that\mbox{}\newline 
 same week were in the next adjoining parishes thus:—\mbox{}\newline 
{<\textbf{table}\hspace*{6pt}{cols}="{4}"\hspace*{6pt}{rows}="{5}">}\mbox{}\newline 
\hspace*{6pt}\hspace*{6pt}{<\textbf{row}\hspace*{6pt}{role}="{data}">}\mbox{}\newline 
\hspace*{6pt}\hspace*{6pt}\hspace*{6pt}{<\textbf{cell}\hspace*{6pt}{role}="{label}">}St. Leonard's, Shoreditch{</\textbf{cell}>}\mbox{}\newline 
\hspace*{6pt}\hspace*{6pt}\hspace*{6pt}{<\textbf{cell}>}64{</\textbf{cell}>}\mbox{}\newline 
\hspace*{6pt}\hspace*{6pt}\hspace*{6pt}{<\textbf{cell}>}84{</\textbf{cell}>}\mbox{}\newline 
\hspace*{6pt}\hspace*{6pt}\hspace*{6pt}{<\textbf{cell}>}119{</\textbf{cell}>}\mbox{}\newline 
\hspace*{6pt}\hspace*{6pt}{</\textbf{row}>}\mbox{}\newline 
\hspace*{6pt}\hspace*{6pt}{<\textbf{row}\hspace*{6pt}{role}="{data}">}\mbox{}\newline 
\hspace*{6pt}\hspace*{6pt}\hspace*{6pt}{<\textbf{cell}\hspace*{6pt}{role}="{label}">}St. Botolph's, Bishopsgate{</\textbf{cell}>}\mbox{}\newline 
\hspace*{6pt}\hspace*{6pt}\hspace*{6pt}{<\textbf{cell}>}65{</\textbf{cell}>}\mbox{}\newline 
\hspace*{6pt}\hspace*{6pt}\hspace*{6pt}{<\textbf{cell}>}105{</\textbf{cell}>}\mbox{}\newline 
\hspace*{6pt}\hspace*{6pt}\hspace*{6pt}{<\textbf{cell}>}116{</\textbf{cell}>}\mbox{}\newline 
\hspace*{6pt}\hspace*{6pt}{</\textbf{row}>}\mbox{}\newline 
\hspace*{6pt}\hspace*{6pt}{<\textbf{row}\hspace*{6pt}{role}="{data}">}\mbox{}\newline 
\hspace*{6pt}\hspace*{6pt}\hspace*{6pt}{<\textbf{cell}\hspace*{6pt}{role}="{label}">}St. Giles's, Cripplegate{</\textbf{cell}>}\mbox{}\newline 
\hspace*{6pt}\hspace*{6pt}\hspace*{6pt}{<\textbf{cell}>}213{</\textbf{cell}>}\mbox{}\newline 
\hspace*{6pt}\hspace*{6pt}\hspace*{6pt}{<\textbf{cell}>}421{</\textbf{cell}>}\mbox{}\newline 
\hspace*{6pt}\hspace*{6pt}\hspace*{6pt}{<\textbf{cell}>}554{</\textbf{cell}>}\mbox{}\newline 
\hspace*{6pt}\hspace*{6pt}{</\textbf{row}>}\mbox{}\newline 
\hspace*{6pt}{</\textbf{table}>}\mbox{}\newline 
{</\textbf{p}>}\mbox{}\newline 
{<\textbf{p}>}This shutting up of houses was at first counted a very cruel\mbox{}\newline 
 and unchristian method, and the poor people so confined made\mbox{}\newline 
 bitter lamentations. ... {</\textbf{p}>}\end{shaded}\egroup\par 
\section[{Figures and Graphics}]{Figures and Graphics}\label{U5-figs}\par
Not all the components of a document are necessarily textual. The most straightforward text will often contain diagrams or illustrations, to say nothing of documents in which image and text are inextricably intertwined, or electronic resources in which the two are complementary. \par
The encoder may simply record the presence of a graphic within the text, possibly with a brief description of its content, by using the elements described in this section. The same elements may also be used to embed digitized versions of the graphic within an electronic document. \par
Any textual information accompanying the graphic, such as a heading and/or caption, may be included within the \texttt{<figure>} element itself, in a \texttt{<head>} and one or more \texttt{<p>} elements, as may also any text appearing within the graphic itself. It is strongly recommended that a prose description of the image be supplied, as the content of a \texttt{<figDesc>} element, for the use of applications which are not able to render the graphic, and to render the document accessible to vision-impaired readers. (Such text is not normally considered part of the document proper.)\par
The simplest use for these elements is to mark the position of a graphic and provide a link to it, as in this example; \par\bgroup\exampleFont \begin{shaded}\noindent\mbox{}{<\textbf{pb}\hspace*{6pt}{n}="{412}"/>}\mbox{}\newline 
{<\textbf{graphic}\hspace*{6pt}{url}="{p412fig.png}"/>}\mbox{}\newline 
{<\textbf{pb}\hspace*{6pt}{n}="{413}"/>}\end{shaded}\egroup\par \noindent  This indicates that the graphic contained by the file \textsf{p412fig.png} appears between pages 412 and 413.\par
The \texttt{<graphic>} element can appear anywhere that textual content is permitted, within but not between paragraphs or headings. In the following example, the encoder has decided to treat a specific printer's ornament as a heading: \par\bgroup\exampleFont \begin{shaded}\noindent\mbox{}{<\textbf{head}>}\mbox{}\newline 
\hspace*{6pt}{<\textbf{graphic}\hspace*{6pt}{url}="{http://www.iath.virginia.edu/gants/Ornaments/Heads/hp-ral02.gif}"/>}\mbox{}\newline 
{</\textbf{head}>}\end{shaded}\egroup\par \noindent \par
More usually, a graphic will have at the least an identifying title, which may be encoded using the \texttt{<head>} element, or a number of figures may be grouped together in a particular structure. It is also often convenient to include a brief description of the image. The \texttt{<figure>} element provides a means of wrapping one or more such elements together as a kind of graphic ‘block’: \par\bgroup\exampleFont \begin{shaded}\noindent\mbox{}{<\textbf{figure}>}\mbox{}\newline 
\hspace*{6pt}{<\textbf{graphic}\hspace*{6pt}{url}="{fessipic.png}"/>}\mbox{}\newline 
\hspace*{6pt}{<\textbf{head}>}Mr Fezziwig's Ball{</\textbf{head}>}\mbox{}\newline 
\hspace*{6pt}{<\textbf{figDesc}>}A Cruikshank engraving showing Mr Fezziwig leading\mbox{}\newline 
\hspace*{6pt}\hspace*{6pt} a group of revellers.{</\textbf{figDesc}>}\mbox{}\newline 
{</\textbf{figure}>}\end{shaded}\egroup\par \par
When a digitized version of the graphic concerned is available, it may be embedded at the appropriate point within the document in this way.
\section[{Interpretation and  Analysis}]{Interpretation and Analysis}\label{U5-anal}\par
It is often said that \textit{all} markup is a form of interpretation or analysis. While it is certainly difficult, and may be impossible, to distinguish firmly between ‘objective’ and ‘subjective’ information in any universal way, it remains true that judgments concerning the latter are typically regarded as more likely to provide controversy than those concerning the former. Many scholars therefore prefer to record such interpretations only if it is possible to alert the reader that they are considered more open to dispute, than the rest of the markup. This section describes some of the elements provided by the TEI scheme to meet this need.
\subsection[{Orthographic Sentences}]{Orthographic Sentences}\par
Interpretation typically ranges across the whole of a text, with no particular respect to other structural units. A useful preliminary to intensive interpretation is therefore to segment the text into discrete and identifiable units, each of which can then bear a label for use as a sort of ‘canonical reference’. To facilitate such uses, these units may not cross each other, nor nest within each other. They may conveniently be represented using the following element: \par
As the name suggests, the \texttt{<s>} element is most commonly used (in linguistic applications at least) for marking \textit{orthographic sentences}, that is, units defined by orthographic features such as punctuation. For example, the passage from \textit{Jane Eyre} discussed earlier might be divided into s-units as follows:\par\bgroup\exampleFont \begin{shaded}\noindent\mbox{}{<\textbf{pb}\hspace*{6pt}{n}="{474}"/>}\mbox{}\newline 
{<\textbf{div}\hspace*{6pt}{n}="{38}"\hspace*{6pt}{type}="{chapter}">}\mbox{}\newline 
\hspace*{6pt}{<\textbf{p}>}\mbox{}\newline 
\hspace*{6pt}\hspace*{6pt}{<\textbf{s}\hspace*{6pt}{n}="{001}">}Reader, I married him.{</\textbf{s}>}\mbox{}\newline 
\hspace*{6pt}\hspace*{6pt}{<\textbf{s}\hspace*{6pt}{n}="{002}">}A quiet wedding we had:{</\textbf{s}>}\mbox{}\newline 
\hspace*{6pt}\hspace*{6pt}{<\textbf{s}\hspace*{6pt}{n}="{003}">}he and I, the parson and clerk, were alone present.{</\textbf{s}>}\mbox{}\newline 
\hspace*{6pt}\hspace*{6pt}{<\textbf{s}\hspace*{6pt}{n}="{004}">}When we got back from church, I went\mbox{}\newline 
\hspace*{6pt}\hspace*{6pt}\hspace*{6pt}\hspace*{6pt} into the kitchen of the manor-house, where Mary was cooking\mbox{}\newline 
\hspace*{6pt}\hspace*{6pt}\hspace*{6pt}\hspace*{6pt} the dinner, and John cleaning the knives,\mbox{}\newline 
\hspace*{6pt}\hspace*{6pt}\hspace*{6pt}\hspace*{6pt} and I said —{</\textbf{s}>}\mbox{}\newline 
\hspace*{6pt}{</\textbf{p}>}\mbox{}\newline 
\hspace*{6pt}{<\textbf{p}>}\mbox{}\newline 
\hspace*{6pt}\hspace*{6pt}{<\textbf{q}>}\mbox{}\newline 
\hspace*{6pt}\hspace*{6pt}\hspace*{6pt}{<\textbf{s}\hspace*{6pt}{n}="{005}">}Mary, I have been married to Mr Rochester\mbox{}\newline 
\hspace*{6pt}\hspace*{6pt}\hspace*{6pt}\hspace*{6pt}\hspace*{6pt}\hspace*{6pt} this morning.{</\textbf{s}>}\mbox{}\newline 
\hspace*{6pt}\hspace*{6pt}{</\textbf{q}>} ... {</\textbf{p}>}\mbox{}\newline 
{</\textbf{div}>}\end{shaded}\egroup\par \noindent  Note that \texttt{<s>} elements cannot nest: the beginning of one \texttt{<s>} element implies that the previous one has finished. When s-units are tagged as shown above, it is advisable to tag the entire text end-to-end, so that every word in the text being analysed will be contained by exactly one \texttt{<s>} element, whose identifier can then be used to specify a unique reference for it. If the identifiers used are unique within the document, then the \textit{@xml:id} attribute might be used in preference to the \textit{@n} used in the above example.
\subsection[{General-Purpose Interpretation Elements}]{General-Purpose Interpretation Elements}\par
A more general purpose segmentation element, the \texttt{<seg>} has already been introduced for use in identifying otherwise unmarked targets of cross references and hypertext links (see section \textit{\hyperref[U5-ptrs]{8. Cross References and Links}}); it identifies some phrase-level portion of text to which the encoder may assign a user-specified \textit{@type}, as well as a unique identifier; it may thus be used to tag textual features for which there is no provision in the published TEI Guidelines.\par
For example, the Guidelines provide no ‘apostrophe’ element to mark parts of a literary text in which the narrator addresses the reader (or hearer) directly. One approach might be to regard these as instances of the \texttt{<q>} element, distinguished from others by an appropriate value for the \textit{@who} attribute. A possibly simpler, and certainly more general, solution would however be to use the \texttt{<seg>} element as follows: \par\bgroup\exampleFont \begin{shaded}\noindent\mbox{}{<\textbf{div}\hspace*{6pt}{n}="{38}"\hspace*{6pt}{type}="{chapter}">}\mbox{}\newline 
\hspace*{6pt}{<\textbf{p}>}\mbox{}\newline 
\hspace*{6pt}\hspace*{6pt}{<\textbf{seg}\hspace*{6pt}{type}="{apostrophe}">}Reader, I married him.{</\textbf{seg}>}\mbox{}\newline 
\hspace*{6pt}\hspace*{6pt} A quiet wedding we had: ...{</\textbf{p}>}\mbox{}\newline 
{</\textbf{div}>}\end{shaded}\egroup\par \noindent  The \textit{@type} attribute on the \texttt{<seg>} element can take any value, and so can be used to record phrase-level phenomena of any kind; it is good practice to record the values used and their significance in the header.\par
A \texttt{<seg>} element of one type (unlike the \texttt{<s>} element which it superficially resembles) can be nested within a \texttt{<seg>} element of the same or another type. This enables quite complex structures to be represented; some examples were given in section \textit{\hyperref[xatts]{8.3. Special kinds of Linking}} above. However, because it must respect the requirement that elements be properly nested, and may not cut across each other, it cannot cope with the common requirement to associate an interpretation with arbitrary segments of a text which may completely ignore the document hierarchy. It also requires that the interpretation itself be represented by a single coded value in the \textit{@type} attribute.\par
Neither restriction applies to the \texttt{<interp>} element, which provides powerful features for the encoding of quite complex interpretive information in a relatively straightforward manner.  These elements allows the encoder to specify both the class of an interpretation, and the particular instance of that class which the interpretation involves. Thus, whereas with \texttt{<seg>} one can say simply that something is an apostrophe, with \texttt{<interp>} one can say that it is an instance (apostrophe) of a larger class (rhetorical figures).\par
Moreover, \texttt{<interp>} is an empty element, which must be linked to the passage to which it applies either by means of the \textit{@ana} attribute discussed in section \textit{\hyperref[xatts]{8.3. Special kinds of Linking}} above, or by means of its own \textit{@inst} attribute. This means that any kind of analysis can be represented, with no need to respect the document hierarchy, and also facilitates the grouping of analyses of a particular type together. A special purpose \texttt{<interpGrp>} element is provided for the latter purpose.\par
For example, suppose that you wish to mark such diverse aspects of a text as themes or subject matter, rhetorical figures, and the locations of individual scenes of the narrative. Different portions of our sample passage from \textit{Jane Eyre} for example, might be associated with the rhetorical figures of apostrophe, hyperbole, and metaphor; with subject-matter references to churches, servants, cooking, postal service, and honeymoons; and with scenes located in the church, in the kitchen, and in an unspecified location (drawing room?).\par
These interpretations could be placed anywhere within the \texttt{<text>} element; it is however good practice to put them all in the same place (e.g. a separate section of the front or back matter), as in the following example: \par\bgroup\exampleFont \begin{shaded}\noindent\mbox{}{<\textbf{back}>}\mbox{}\newline 
\hspace*{6pt}{<\textbf{div}\hspace*{6pt}{type}="{Interpretations}">}\mbox{}\newline 
\hspace*{6pt}\hspace*{6pt}{<\textbf{p}>}\mbox{}\newline 
\hspace*{6pt}\hspace*{6pt}\hspace*{6pt}{<\textbf{interp}\hspace*{6pt}{resp}="{\#LB-MSM}"\mbox{}\newline 
\hspace*{6pt}\hspace*{6pt}\hspace*{6pt}\hspace*{6pt}{type}="{figureOfSpeech}"\hspace*{6pt}{xml:id}="{fig-apos-1}">}apostrophe{</\textbf{interp}>}\mbox{}\newline 
\hspace*{6pt}\hspace*{6pt}\hspace*{6pt}{<\textbf{interp}\hspace*{6pt}{resp}="{\#LB-MSM}"\mbox{}\newline 
\hspace*{6pt}\hspace*{6pt}\hspace*{6pt}\hspace*{6pt}{type}="{figureOfSpeech}"\hspace*{6pt}{xml:id}="{fig-hyp-1}">}hyperbole{</\textbf{interp}>}\mbox{}\newline 
\hspace*{6pt}\hspace*{6pt}\hspace*{6pt}{<\textbf{interp}\hspace*{6pt}{resp}="{\#LB-MSM}"\hspace*{6pt}{type}="{setting}"\mbox{}\newline 
\hspace*{6pt}\hspace*{6pt}\hspace*{6pt}\hspace*{6pt}{xml:id}="{set-church-1}">}church{</\textbf{interp}>}\mbox{}\newline 
\hspace*{6pt}\hspace*{6pt}\hspace*{6pt}{<\textbf{interp}\hspace*{6pt}{resp}="{\#LB-MSM}"\hspace*{6pt}{type}="{reference}"\mbox{}\newline 
\hspace*{6pt}\hspace*{6pt}\hspace*{6pt}\hspace*{6pt}{xml:id}="{ref-church-1}">}church{</\textbf{interp}>}\mbox{}\newline 
\hspace*{6pt}\hspace*{6pt}\hspace*{6pt}{<\textbf{interp}\hspace*{6pt}{resp}="{\#LB-MSM}"\hspace*{6pt}{type}="{reference}"\mbox{}\newline 
\hspace*{6pt}\hspace*{6pt}\hspace*{6pt}\hspace*{6pt}{xml:id}="{ref-serv-1}">}servants{</\textbf{interp}>}\mbox{}\newline 
\hspace*{6pt}\hspace*{6pt}{</\textbf{p}>}\mbox{}\newline 
\hspace*{6pt}{</\textbf{div}>}\mbox{}\newline 
{</\textbf{back}>}\end{shaded}\egroup\par \par
The evident redundancy of this encoding can be considerably reduced by using the \texttt{<interpGrp>} element to group together all those \texttt{<interp>} elements which share common attribute values, as follows: \par\bgroup\exampleFont \begin{shaded}\noindent\mbox{}{<\textbf{back}>}\mbox{}\newline 
\hspace*{6pt}{<\textbf{div}\hspace*{6pt}{type}="{Interpretations}">}\mbox{}\newline 
\hspace*{6pt}\hspace*{6pt}{<\textbf{p}>}\mbox{}\newline 
\hspace*{6pt}\hspace*{6pt}\hspace*{6pt}{<\textbf{interpGrp}\hspace*{6pt}{resp}="{\#LB-MSM}"\mbox{}\newline 
\hspace*{6pt}\hspace*{6pt}\hspace*{6pt}\hspace*{6pt}{type}="{figureOfSpeech}">}\mbox{}\newline 
\hspace*{6pt}\hspace*{6pt}\hspace*{6pt}\hspace*{6pt}{<\textbf{interp}\hspace*{6pt}{xml:id}="{fig-apos}">}apostrophe{</\textbf{interp}>}\mbox{}\newline 
\hspace*{6pt}\hspace*{6pt}\hspace*{6pt}\hspace*{6pt}{<\textbf{interp}\hspace*{6pt}{xml:id}="{fig-hyp}">}hyperbole{</\textbf{interp}>}\mbox{}\newline 
\hspace*{6pt}\hspace*{6pt}\hspace*{6pt}\hspace*{6pt}{<\textbf{interp}\hspace*{6pt}{xml:id}="{fig-meta}">}metaphor{</\textbf{interp}>}\mbox{}\newline 
\hspace*{6pt}\hspace*{6pt}\hspace*{6pt}{</\textbf{interpGrp}>}\mbox{}\newline 
\hspace*{6pt}\hspace*{6pt}\hspace*{6pt}{<\textbf{interpGrp}\hspace*{6pt}{resp}="{\#LB-MSM}"\mbox{}\newline 
\hspace*{6pt}\hspace*{6pt}\hspace*{6pt}\hspace*{6pt}{type}="{scene-setting}">}\mbox{}\newline 
\hspace*{6pt}\hspace*{6pt}\hspace*{6pt}\hspace*{6pt}{<\textbf{interp}\hspace*{6pt}{xml:id}="{set-church}">}church{</\textbf{interp}>}\mbox{}\newline 
\hspace*{6pt}\hspace*{6pt}\hspace*{6pt}\hspace*{6pt}{<\textbf{interp}\hspace*{6pt}{xml:id}="{set-kitch}">}kitchen{</\textbf{interp}>}\mbox{}\newline 
\hspace*{6pt}\hspace*{6pt}\hspace*{6pt}\hspace*{6pt}{<\textbf{interp}\hspace*{6pt}{xml:id}="{set-unspec}">}unspecified{</\textbf{interp}>}\mbox{}\newline 
\hspace*{6pt}\hspace*{6pt}\hspace*{6pt}{</\textbf{interpGrp}>}\mbox{}\newline 
\hspace*{6pt}\hspace*{6pt}\hspace*{6pt}{<\textbf{interpGrp}\hspace*{6pt}{resp}="{\#LB-MSM}"\mbox{}\newline 
\hspace*{6pt}\hspace*{6pt}\hspace*{6pt}\hspace*{6pt}{type}="{reference}">}\mbox{}\newline 
\hspace*{6pt}\hspace*{6pt}\hspace*{6pt}\hspace*{6pt}{<\textbf{interp}\hspace*{6pt}{xml:id}="{ref-church}">}church{</\textbf{interp}>}\mbox{}\newline 
\hspace*{6pt}\hspace*{6pt}\hspace*{6pt}\hspace*{6pt}{<\textbf{interp}\hspace*{6pt}{xml:id}="{ref-serv}">}servants{</\textbf{interp}>}\mbox{}\newline 
\hspace*{6pt}\hspace*{6pt}\hspace*{6pt}\hspace*{6pt}{<\textbf{interp}\hspace*{6pt}{xml:id}="{ref-cook}">}cooking{</\textbf{interp}>}\mbox{}\newline 
\hspace*{6pt}\hspace*{6pt}\hspace*{6pt}{</\textbf{interpGrp}>}\mbox{}\newline 
\hspace*{6pt}\hspace*{6pt}{</\textbf{p}>}\mbox{}\newline 
\hspace*{6pt}{</\textbf{div}>}\mbox{}\newline 
{</\textbf{back}>}\end{shaded}\egroup\par \par
Once these interpretation elements have been defined, they can be linked with the parts of the text to which they apply in either or both of two ways. The \textit{@ana} attribute can be used on whichever element is appropriate: \par\bgroup\exampleFont \begin{shaded}\noindent\mbox{}{<\textbf{div}\hspace*{6pt}{n}="{38}"\hspace*{6pt}{type}="{chapter}">}\mbox{}\newline 
\hspace*{6pt}{<\textbf{p}\hspace*{6pt}{ana}="{\#set-church \#set-kitch}"\mbox{}\newline 
\hspace*{6pt}\hspace*{6pt}{xml:id}="{P38.1}">}\mbox{}\newline 
\hspace*{6pt}\hspace*{6pt}{<\textbf{s}\hspace*{6pt}{ana}="{\#fig-apos}"\hspace*{6pt}{xml:id}="{P38.1.1}">}Reader, I married him.{</\textbf{s}>}\mbox{}\newline 
\hspace*{6pt}{</\textbf{p}>}\mbox{}\newline 
{</\textbf{div}>}\end{shaded}\egroup\par \noindent  Note in this example that since the paragraph has two settings (in the church and in the kitchen), the identifiers of both have been supplied.\par
Alternatively, the \texttt{<interp>} elements can point to all the parts of the text to which they apply, using their \textit{@inst} attribute: \par\bgroup\exampleFont \begin{shaded}\noindent\mbox{}{<\textbf{interp}\hspace*{6pt}{inst}="{\#P38.1.1}"\hspace*{6pt}{resp}="{\#LB-MSM}"\mbox{}\newline 
\hspace*{6pt}{type}="{figureOfSpeech}"\hspace*{6pt}{xml:id}="{fig-apos-2}">}apostrophe{</\textbf{interp}>}\mbox{}\newline 
{<\textbf{interp}\hspace*{6pt}{inst}="{\#P38.1}"\hspace*{6pt}{resp}="{\#LB-MSM}"\mbox{}\newline 
\hspace*{6pt}{type}="{scene-setting}"\hspace*{6pt}{xml:id}="{set-church-2}">}church{</\textbf{interp}>}\mbox{}\newline 
{<\textbf{interp}\hspace*{6pt}{inst}="{\#P38.1}"\hspace*{6pt}{resp}="{\#LB-MSM}"\mbox{}\newline 
\hspace*{6pt}{type}="{scene-setting}"\hspace*{6pt}{xml:id}="{set-kitchen-2}">}kitchen{</\textbf{interp}>}\end{shaded}\egroup\par \par
The \texttt{<interp>} is not limited to any particular type of analysis, The literary analysis shown above is but one possibility; one could equally well use \texttt{<interp>} to capture a linguistic part-of-speech analysis. For example, the example sentence given in section \textit{\hyperref[xatts]{8.3. Special kinds of Linking}} assumes a linguistic analysis which might be represented as follows: \par\bgroup\exampleFont \begin{shaded}\noindent\mbox{}{<\textbf{interp}\hspace*{6pt}{type}="{pos}"\hspace*{6pt}{xml:id}="{NP1}">}noun phrase, singular{</\textbf{interp}>}\mbox{}\newline 
{<\textbf{interp}\hspace*{6pt}{type}="{pos}"\hspace*{6pt}{xml:id}="{VV1}">}inflected verb, present-tense singular{</\textbf{interp}>}\mbox{}\newline 
 ...\mbox{}\newline 
\end{shaded}\egroup\par 
\section[{Technical Documentation}]{Technical Documentation}\label{U5-techdoc}\par
Although the focus of this document is on the use of the TEI scheme for the encoding of existing ‘pre-electronic’ documents, the same scheme may also be used for the encoding of new documents. In the preparation of new documents (such as this one), XML has much to recommend it: the document's structure can be clearly represented, and the same electronic text can be re-used for many purposes — to provide both online hypertext or browsable versions and well-formatted typeset versions from a common source for example.\par
To facilitate this, the TEI Lite schema includes some elements for marking features of technical documents in general, and of XML-related documents in particular.
\subsection[{Additional Elements for Technical Documents}]{Additional Elements for Technical Documents}\par
The following elements may be used to mark particular features of technical documents: \par
The following example shows how these elements might be used to encode a passage from a tutorial introducing the Fortran programming language: \par\bgroup\exampleFont \begin{shaded}\noindent\mbox{}{<\textbf{p}>}It is traditional to introduce a language with a program like the\mbox{}\newline 
 following:\mbox{}\newline 
{<\textbf{eg}>} CHAR*12 GRTG\mbox{}\newline 
\hspace*{6pt}\hspace*{6pt} GRTG = 'HELLO WORLD'\mbox{}\newline 
\hspace*{6pt}\hspace*{6pt} PRINT *, GRTG\mbox{}\newline 
\hspace*{6pt}\hspace*{6pt} END\mbox{}\newline 
\hspace*{6pt}{</\textbf{eg}>}\mbox{}\newline 
{</\textbf{p}>}\mbox{}\newline 
{<\textbf{p}>}This simple example first declares a variable {<\textbf{ident}>}GRTG{</\textbf{ident}>}, in\mbox{}\newline 
 the line {<\textbf{code}>}CHAR*12 GRTG{</\textbf{code}>}, which identifies {<\textbf{ident}>}GRTG{</\textbf{ident}>}\mbox{}\newline 
 as consisting of 12 bytes of type {<\textbf{ident}>}CHAR{</\textbf{ident}>}. To this variable,\mbox{}\newline 
 the value {<\textbf{val}>}HELLO WORLD{</\textbf{val}>}\mbox{}\newline 
 is then assigned.{</\textbf{p}>}\end{shaded}\egroup\par \par
A formatting application, given a text like that above, can be instructed to format examples appropriately (e.g. to preserve line breaks, or to use a distinctive font). Similarly, the use of tags such as \texttt{<ident>} greatly facilitates the construction of a useful index.\par
The \texttt{<formula>} element should be used to enclose a mathematical or chemical formula presented within the text as a distinct item. Since formulae generally include a large variety of special typographic features not otherwise present in ordinary text, it will usually be necessary to present the body of the formula in a specialized notation. The notation used should be specified by the \textit{@notation} attribute, as in the following example: \par\bgroup\exampleFont \begin{shaded}\noindent\mbox{}{<\textbf{formula}\hspace*{6pt}{notation}="{tex}">} ⃥begin❴math❵E = mc\textasciicircum ❴2❵⃥end❴math❵\mbox{}\newline 
{</\textbf{formula}>}\end{shaded}\egroup\par \par
A particular problem arises when XML encoding is the subject of discussion within a technical document, itself encoded in XML. In such a document, it is clearly essential to distinguish clearly the markup occurring within examples from that marking up the document itself, and end-tags are highly likely to occur. One simple solution is to use the predefined entity reference \texttt{\&lt;} to represent each < character which marks the start of an XML tag within the examples. A more general solution is to mark off the whole body of each example as containing data which is not to be scanned for XML mark-up by the parser. This is achieved by enclosing it within a special XML construct called a \textit{CDATA marked section}, as in the following example: \par\hfill\bgroup\exampleFont\vskip 10pt\begin{shaded}
\obeyspaces <p>A list should be encoded as follows:\newline
<eg><![ CDATA [\newline
   <list>\newline
   <item>First item in the list</item>\newline
   <item>Second item</item>\newline
   </list>\newline
]]>\newline
</eg>\newline
The <gi>list</gi> element consists of a series of <gi>item</gi>\newline
elements.\end{shaded}
\par\egroup 
\par
The \texttt{<list>} element used within the example above will not be regarded as forming part of the document proper, because it is embedded within a marked section (beginning with the special markup declaration ‘<![CDATA[ ’, and ending with ‘]]>’).\par
Note also the use of the \texttt{<gi>} element to tag references to element names (or \textit{generic identifiers}) within the body of the text.
\subsection[{Generated Divisions}]{Generated Divisions}\par
Most modern document production systems have the ability to generate automatically whole sections such as a table of contents or an index. The TEI Lite scheme provides an element to mark the location at which such a generated section should be placed. \par
The \texttt{<divGen>} element can be placed anywhere that a division element would be legal, as in the following example: \par\bgroup\exampleFont \begin{shaded}\noindent\mbox{}{<\textbf{front}>}\mbox{}\newline 
\hspace*{6pt}{<\textbf{titlePage}>}\mbox{}\newline 
\textit{<!-- ... -->}\mbox{}\newline 
\hspace*{6pt}{</\textbf{titlePage}>}\mbox{}\newline 
\hspace*{6pt}{<\textbf{divGen}\hspace*{6pt}{type}="{toc}"/>}\mbox{}\newline 
\hspace*{6pt}{<\textbf{div}>}\mbox{}\newline 
\hspace*{6pt}\hspace*{6pt}{<\textbf{head}>}Preface{</\textbf{head}>}\mbox{}\newline 
\textit{<!-- ... -->}\mbox{}\newline 
\hspace*{6pt}{</\textbf{div}>}\mbox{}\newline 
{</\textbf{front}>}\mbox{}\newline 
{<\textbf{body}>}\mbox{}\newline 
\textit{<!-- ... -->}\mbox{}\newline 
{</\textbf{body}>}\mbox{}\newline 
{<\textbf{back}>}\mbox{}\newline 
\hspace*{6pt}{<\textbf{div}>}\mbox{}\newline 
\hspace*{6pt}\hspace*{6pt}{<\textbf{head}>}Appendix{</\textbf{head}>}\mbox{}\newline 
\textit{<!-- ... -->}\mbox{}\newline 
\hspace*{6pt}{</\textbf{div}>}\mbox{}\newline 
\hspace*{6pt}{<\textbf{divGen}\hspace*{6pt}{n}="{Index}"\hspace*{6pt}{type}="{index}"/>}\mbox{}\newline 
{</\textbf{back}>}\end{shaded}\egroup\par \par
This example also demonstrates the use of the \textit{@type} attribute to distinguish the different kinds of division to be generated: in the first case a table of contents (a \textit{toc}) and in the second an index.\par
When an existing index or table of contents is to be encoded (rather than one being generated) for some reason, the \texttt{<list>} element discussed in section \textit{\hyperref[U5-lists]{11. Lists}} should be used.
\subsection[{Index Generation}]{Index Generation}\label{indexing}\par
While production of a table of contents from a properly tagged document is generally unproblematic for an automatic processor, the production of a good quality index will often require more careful tagging. It may not be enough simply to produce a list of all parts tagged in some particular way, although extracting (for example) all occurrences of elements such as \texttt{<term>} or \texttt{<name>} will often be a good departure point for an index.\par
The TEI schema provides a special purpose \texttt{<index>} tag which may be used to mark both the parts of the document which should be indexed, and how the indexing should be done. \par
For example, the second paragraph of this section might include the following:\par\bgroup\exampleFont \begin{shaded}\noindent\mbox{}...\mbox{}\newline 
 TEI lite also provides a special purpose {<\textbf{gi}>}index{</\textbf{gi}>} tag\mbox{}\newline 
\mbox{}\newline 
{<\textbf{index}>}\mbox{}\newline 
\hspace*{6pt}{<\textbf{term}>}indexing{</\textbf{term}>}\mbox{}\newline 
{</\textbf{index}>}\mbox{}\newline 
{<\textbf{index}>}\mbox{}\newline 
\hspace*{6pt}{<\textbf{term}>}index (tag){</\textbf{term}>}\mbox{}\newline 
\hspace*{6pt}{<\textbf{index}>}\mbox{}\newline 
\hspace*{6pt}\hspace*{6pt}{<\textbf{term}>}use in index generation{</\textbf{term}>}\mbox{}\newline 
\hspace*{6pt}{</\textbf{index}>}\mbox{}\newline 
{</\textbf{index}>}\mbox{}\newline 
 which may be used ...\end{shaded}\egroup\par \par
The \texttt{<index>} element can also be used to provide a form of interpretive or analytic information. For example, in a study of Ovid, it might be desired to record all the poet's references to different figures, for comparative stylistic study. In the following lines of the \textit{Metamorphoses}, such a study would record the poet's references to Jupiter (as \textit{deus}, \textit{se}, and as the subject of \textit{confiteor} [in inflectional form number 227]), to Jupiter-in-the-guise-of-a-bull (as \textit{imago tauri fallacis} and the subject of \textit{teneo}), and so on.\footnote{The analysis is taken, with permission, from Willard McCarty and Burton Wright, \textit{An Analytical Onomasticon to the Metamorphoses of Ovid} (Princeton: Princeton University Press, forthcoming). Some simplifications have been undertaken.} \par\bgroup\exampleFont \begin{shaded}\noindent\mbox{}{<\textbf{l}\hspace*{6pt}{n}="{3.001}">}iamque deus posita fallacis imagine tauri{</\textbf{l}>}\mbox{}\newline 
{<\textbf{l}\hspace*{6pt}{n}="{3.002}">}se confessus erat Dictaeaque rura tenebat{</\textbf{l}>}\end{shaded}\egroup\par \noindent  This need might be met using the \texttt{<note>} element discussed in section in \textit{\hyperref[U5-notes]{7. Notes}}, or with the \texttt{<interp>} element discussed in section \textit{\hyperref[U5-anal]{15. Interpretation and Analysis}}. Here we demonstrate how it might also be satisfied by using the \texttt{<index>} element.\par
We assume that the object is to generate more than one index: one for names of deities (called \textit{@dn}), another for onomastic references (called \textit{@on}), a third for pronominal references (called \textit{@pr}) and so forth. One way of achieving this might be as follows: \par\bgroup\exampleFont \begin{shaded}\noindent\mbox{}{<\textbf{l}\hspace*{6pt}{n}="{3.001}">}iamque deus posita fallacis imagine tauri\mbox{}\newline 
{<\textbf{index}\hspace*{6pt}{indexName}="{dn}">}\mbox{}\newline 
\hspace*{6pt}\hspace*{6pt}{<\textbf{term}>}Iuppiter{</\textbf{term}>}\mbox{}\newline 
\hspace*{6pt}\hspace*{6pt}{<\textbf{index}>}\mbox{}\newline 
\hspace*{6pt}\hspace*{6pt}\hspace*{6pt}{<\textbf{term}>}deus{</\textbf{term}>}\mbox{}\newline 
\hspace*{6pt}\hspace*{6pt}{</\textbf{index}>}\mbox{}\newline 
\hspace*{6pt}{</\textbf{index}>}\mbox{}\newline 
\hspace*{6pt}{<\textbf{index}\hspace*{6pt}{indexName}="{on}">}\mbox{}\newline 
\hspace*{6pt}\hspace*{6pt}{<\textbf{term}>}Iuppiter (taurus){</\textbf{term}>}\mbox{}\newline 
\hspace*{6pt}\hspace*{6pt}{<\textbf{index}>}\mbox{}\newline 
\hspace*{6pt}\hspace*{6pt}\hspace*{6pt}{<\textbf{term}>}imago tauri fallacis{</\textbf{term}>}\mbox{}\newline 
\hspace*{6pt}\hspace*{6pt}{</\textbf{index}>}\mbox{}\newline 
\hspace*{6pt}{</\textbf{index}>}\mbox{}\newline 
{</\textbf{l}>}\mbox{}\newline 
{<\textbf{l}\hspace*{6pt}{n}="{3.002}">}se confessus erat Dictaeaque rura tenebat\mbox{}\newline 
{<\textbf{index}\hspace*{6pt}{indexName}="{pr}">}\mbox{}\newline 
\hspace*{6pt}\hspace*{6pt}{<\textbf{term}>}Iuppiter{</\textbf{term}>}\mbox{}\newline 
\hspace*{6pt}\hspace*{6pt}{<\textbf{index}>}\mbox{}\newline 
\hspace*{6pt}\hspace*{6pt}\hspace*{6pt}{<\textbf{term}>}se{</\textbf{term}>}\mbox{}\newline 
\hspace*{6pt}\hspace*{6pt}{</\textbf{index}>}\mbox{}\newline 
\hspace*{6pt}{</\textbf{index}>}\mbox{}\newline 
\hspace*{6pt}{<\textbf{index}\hspace*{6pt}{indexName}="{v}">}\mbox{}\newline 
\hspace*{6pt}\hspace*{6pt}{<\textbf{term}>}Iuppiter{</\textbf{term}>}\mbox{}\newline 
\hspace*{6pt}\hspace*{6pt}{<\textbf{index}>}\mbox{}\newline 
\hspace*{6pt}\hspace*{6pt}\hspace*{6pt}{<\textbf{term}>}confiteor (v227){</\textbf{term}>}\mbox{}\newline 
\hspace*{6pt}\hspace*{6pt}{</\textbf{index}>}\mbox{}\newline 
\hspace*{6pt}{</\textbf{index}>}\mbox{}\newline 
{</\textbf{l}>}\end{shaded}\egroup\par \noindent  For each \texttt{<index>} element above, an entry will be generated in the appropriate index, using as headword the content of the \texttt{<term>} element it contains; the \texttt{<term>} elements nested within the secondary \texttt{<index>} element in each case provide a secondary keyword. The actual reference will be taken from the context in which the \texttt{<index>} element appears, i.e. in this case the identifier of the \texttt{<l>} element containing it.
\subsection[{Addresses}]{Addresses}\par
The \texttt{<address>} element is used to mark a postal address of any kind. It contains one or more \texttt{<addrLine>} elements, one for each line of the address. \par
Here is a simple example: \par\bgroup\exampleFont \begin{shaded}\noindent\mbox{}{<\textbf{address}>}\mbox{}\newline 
\hspace*{6pt}{<\textbf{addrLine}>}Computer Center (M/C 135){</\textbf{addrLine}>}\mbox{}\newline 
\hspace*{6pt}{<\textbf{addrLine}>}1940 W. Taylor, Room 124{</\textbf{addrLine}>}\mbox{}\newline 
\hspace*{6pt}{<\textbf{addrLine}>}Chicago, IL 60612-7352{</\textbf{addrLine}>}\mbox{}\newline 
\hspace*{6pt}{<\textbf{addrLine}>}U.S.A.{</\textbf{addrLine}>}\mbox{}\newline 
{</\textbf{address}>}\end{shaded}\egroup\par \par
The individual parts of an address may be further distinguished by using the \texttt{<name>} element discussed above (section \textit{\hyperref[nomen]{10.1. Names and Referring Strings}}). \par\bgroup\exampleFont \begin{shaded}\noindent\mbox{}{<\textbf{address}>}\mbox{}\newline 
\hspace*{6pt}{<\textbf{addrLine}>}Computer Center (M/C 135){</\textbf{addrLine}>}\mbox{}\newline 
\hspace*{6pt}{<\textbf{addrLine}>}1940 W. Taylor, Room 124{</\textbf{addrLine}>}\mbox{}\newline 
\hspace*{6pt}{<\textbf{addrLine}>}\mbox{}\newline 
\hspace*{6pt}\hspace*{6pt}{<\textbf{name}\hspace*{6pt}{type}="{city}">}Chicago{</\textbf{name}>}, IL 60612-7352{</\textbf{addrLine}>}\mbox{}\newline 
\hspace*{6pt}{<\textbf{addrLine}>}\mbox{}\newline 
\hspace*{6pt}\hspace*{6pt}{<\textbf{name}\hspace*{6pt}{type}="{country}">}USA{</\textbf{name}>}\mbox{}\newline 
\hspace*{6pt}{</\textbf{addrLine}>}\mbox{}\newline 
{</\textbf{address}>}\end{shaded}\egroup\par 
\section[{Character Sets, Diacritics, etc.}]{Character Sets, Diacritics, etc.}\label{U5-chars}\par
With the advent of XML and its adoption of Unicode as the required character set for all documents, most problems previously associated with the representation of the divers languages and writing systems of the world are greatly reduced. For those working with standard forms of the European languages in particular, almost no special action is needed: any XML editor should enable you to input accented letters or other ‘non-ASCII’ characters directly, and they should be stored in the resulting file in a way which is transferable directly between different systems.\par
There are two important exceptions: the characters \& and < may not be entered directly in an XML document, since they have a special significance as initiating markup. They must always be represented as \textit{entity references}, like this: \texttt{\&amp;} or \texttt{\&lt;}. Other characters may also be represented by means of entity reference where necessary, for example to retain compatibility with a pre-Unicode processing system.
\section[{Front and Back Matter}]{Front and Back Matter}\label{U5-fronbac}
\subsection[{Front Matter}]{Front Matter}\par
For many purposes, particularly in older texts, the preliminary material such as title pages, prefatory epistles, etc., may provide very useful additional linguistic or social information. P5 provides a set of recommendations for distinguishing the textual elements most commonly encountered in front matter, which are summarized here.
\subsubsection[{Title Page}]{Title Page}\label{h51}\par
The start of a title page should be marked with the element \texttt{<titlePage>}. All text contained on the page should be transcribed and tagged with the appropriate element from the following list: \par
Typeface distinctions should be marked with the \textit{@rend} attribute when necessary, as described above. Very detailed description of the letter spacing and sizing used in ornamental titles is not as yet provided for by the Guidelines. Changes of language should be marked by appropriate use of the \textit{@lang} attribute or the \texttt{<foreign>} element, as necessary. Names, wherever they appear, should be tagged using the \texttt{<name>}, as elsewhere.\par
Two example title pages follow: \par\bgroup\exampleFont \begin{shaded}\noindent\mbox{}{<\textbf{titlePage}\hspace*{6pt}{rend}="{Roman}">}\mbox{}\newline 
\hspace*{6pt}{<\textbf{docTitle}>}\mbox{}\newline 
\hspace*{6pt}\hspace*{6pt}{<\textbf{titlePart}\hspace*{6pt}{type}="{main}">} PARADISE REGAIN'D. A POEM In IV {<\textbf{hi}>}BOOKS{</\textbf{hi}>}.\mbox{}\newline 
\hspace*{6pt}\hspace*{6pt}{</\textbf{titlePart}>}\mbox{}\newline 
\hspace*{6pt}\hspace*{6pt}{<\textbf{titlePart}>} To which is added {<\textbf{title}>}SAMSON AGONISTES{</\textbf{title}>}.\mbox{}\newline 
\hspace*{6pt}\hspace*{6pt}{</\textbf{titlePart}>}\mbox{}\newline 
\hspace*{6pt}{</\textbf{docTitle}>}\mbox{}\newline 
\hspace*{6pt}{<\textbf{byline}>}The Author {<\textbf{docAuthor}>}JOHN MILTON{</\textbf{docAuthor}>}\mbox{}\newline 
\hspace*{6pt}{</\textbf{byline}>}\mbox{}\newline 
\hspace*{6pt}{<\textbf{docImprint}>}\mbox{}\newline 
\hspace*{6pt}\hspace*{6pt}{<\textbf{name}>}LONDON{</\textbf{name}>},\mbox{}\newline 
\hspace*{6pt}\hspace*{6pt} Printed by {<\textbf{name}>}J.M.{</\textbf{name}>}\mbox{}\newline 
\hspace*{6pt}\hspace*{6pt} for {<\textbf{name}>}John Starkey{</\textbf{name}>}\mbox{}\newline 
\hspace*{6pt}\hspace*{6pt} at the {<\textbf{name}>}Mitre{</\textbf{name}>}\mbox{}\newline 
\hspace*{6pt}\hspace*{6pt} in {<\textbf{name}>}Fleetstreet{</\textbf{name}>},\mbox{}\newline 
\hspace*{6pt}\hspace*{6pt} near {<\textbf{name}>}Temple-Bar.{</\textbf{name}>}\mbox{}\newline 
\hspace*{6pt}{</\textbf{docImprint}>}\mbox{}\newline 
\hspace*{6pt}{<\textbf{docDate}>}MDCLXXI{</\textbf{docDate}>}\mbox{}\newline 
{</\textbf{titlePage}>}\end{shaded}\egroup\par \noindent \par\bgroup\exampleFont \begin{shaded}\noindent\mbox{}{<\textbf{titlePage}>}\mbox{}\newline 
\hspace*{6pt}{<\textbf{docTitle}>}\mbox{}\newline 
\hspace*{6pt}\hspace*{6pt}{<\textbf{titlePart}\hspace*{6pt}{type}="{main}">} Lives of the Queens of England, from the Norman\mbox{}\newline 
\hspace*{6pt}\hspace*{6pt}\hspace*{6pt}\hspace*{6pt} Conquest;{</\textbf{titlePart}>}\mbox{}\newline 
\hspace*{6pt}\hspace*{6pt}{<\textbf{titlePart}\hspace*{6pt}{type}="{sub}">}with anecdotes of their courts.\mbox{}\newline 
\hspace*{6pt}\hspace*{6pt}{</\textbf{titlePart}>}\mbox{}\newline 
\hspace*{6pt}{</\textbf{docTitle}>}\mbox{}\newline 
\hspace*{6pt}{<\textbf{titlePart}>}Now first published from Official Records\mbox{}\newline 
\hspace*{6pt}\hspace*{6pt} and other authentic documents private as well as\mbox{}\newline 
\hspace*{6pt}\hspace*{6pt} public.{</\textbf{titlePart}>}\mbox{}\newline 
\hspace*{6pt}{<\textbf{docEdition}>}New edition, with corrections and\mbox{}\newline 
\hspace*{6pt}\hspace*{6pt} additions{</\textbf{docEdition}>}\mbox{}\newline 
\hspace*{6pt}{<\textbf{byline}>}By {<\textbf{docAuthor}>}Agnes Strickland{</\textbf{docAuthor}>}\mbox{}\newline 
\hspace*{6pt}{</\textbf{byline}>}\mbox{}\newline 
\hspace*{6pt}{<\textbf{epigraph}>}\mbox{}\newline 
\hspace*{6pt}\hspace*{6pt}{<\textbf{q}>}The treasures of antiquity laid up in old\mbox{}\newline 
\hspace*{6pt}\hspace*{6pt}\hspace*{6pt}\hspace*{6pt} historic rolls, I opened.{</\textbf{q}>}\mbox{}\newline 
\hspace*{6pt}\hspace*{6pt}{<\textbf{bibl}>}BEAUMONT{</\textbf{bibl}>}\mbox{}\newline 
\hspace*{6pt}{</\textbf{epigraph}>}\mbox{}\newline 
\hspace*{6pt}{<\textbf{docImprint}>}Philadelphia: Blanchard and Lea{</\textbf{docImprint}>}\mbox{}\newline 
\hspace*{6pt}{<\textbf{docDate}>}1860.{</\textbf{docDate}>}\mbox{}\newline 
{</\textbf{titlePage}>}\end{shaded}\egroup\par 
\subsubsection[{Prefatory Matter}]{Prefatory Matter}\label{h52}\par
Major blocks of text within the front matter should be marked as \texttt{<div>} or \texttt{<div>} elements; the following suggested values for the \textit{@type} attribute may be used to distinguish various common types of prefatory matter: \begin{description}

\item[{foreword}]a text addressed to the reader, by the author, editor or publisher, possibly in the form of a letter.
\item[{preface}]a text addressed to the reader, by the author, editor or publisher, possibly in the form of a letter.
\item[{dedication}]a text (often a letter) addressed to someone other than the reader in which the author typically commends the work in hand to the attention of the person concerned.
\item[{abstract}]a prose argument summarizing the content of the work.
\item[{ack}]Acknowledgements.
\item[{contents}]a table of contents (typically this should be tagged as a \texttt{<list>}).
\item[{frontispiece}]a pictorial frontispiece, possibly including some text.
\end{description} \par
Like any text division, those in front matter may contain low level structural or non-structural elements as described elsewhere. They will generally begin with a heading or title of some kind which should be tagged using the \texttt{<head>} element. Epistles will contain the following additional elements: \par
Epistles which appear elsewhere in a text will, of course, contain these same elements.\par
As an example, the dedication at the start of Milton's \textit{Comus} should be marked up as follows: \par\bgroup\exampleFont \begin{shaded}\noindent\mbox{}{<\textbf{div}\hspace*{6pt}{type}="{dedication}">}\mbox{}\newline 
\hspace*{6pt}{<\textbf{head}>}To the Right Honourable {<\textbf{name}>}JOHN Lord Viscount\mbox{}\newline 
\hspace*{6pt}\hspace*{6pt}\hspace*{6pt}\hspace*{6pt} BRACLY{</\textbf{name}>}, Son and Heir apparent to the Earl of\mbox{}\newline 
\hspace*{6pt}\hspace*{6pt} Bridgewater, \&c.{</\textbf{head}>}\mbox{}\newline 
\hspace*{6pt}{<\textbf{salute}>}MY LORD,{</\textbf{salute}>}\mbox{}\newline 
\hspace*{6pt}{<\textbf{p}>}THis {<\textbf{hi}>}Poem{</\textbf{hi}>}, which receiv'd its first occasion of\mbox{}\newline 
\hspace*{6pt}\hspace*{6pt} Birth from your Self, and others of your Noble Family ....\mbox{}\newline 
\hspace*{6pt}\hspace*{6pt} and as in this representation your attendant\mbox{}\newline 
\hspace*{6pt}{<\textbf{name}>}Thyrsis{</\textbf{name}>}, so now in all reall expression{</\textbf{p}>}\mbox{}\newline 
\hspace*{6pt}{<\textbf{closer}>}\mbox{}\newline 
\hspace*{6pt}\hspace*{6pt}{<\textbf{salute}>}Your faithfull, and most humble servant{</\textbf{salute}>}\mbox{}\newline 
\hspace*{6pt}\hspace*{6pt}{<\textbf{signed}>}\mbox{}\newline 
\hspace*{6pt}\hspace*{6pt}\hspace*{6pt}{<\textbf{name}>}H. LAWES.{</\textbf{name}>}\mbox{}\newline 
\hspace*{6pt}\hspace*{6pt}{</\textbf{signed}>}\mbox{}\newline 
\hspace*{6pt}{</\textbf{closer}>}\mbox{}\newline 
{</\textbf{div}>}\end{shaded}\egroup\par 
\subsection[{Back Matter}]{Back Matter}
\subsubsection[{Structural Divisions of Back Matter}]{Structural Divisions of Back Matter}\par
Because of variations in publishing practice, back matter can contain virtually any of the elements listed above for front matter, and the same elements should be used where this is so. Additionally, back matter may contain the following types of matter within the \texttt{<back>} element. Like the structural divisions of the body, these should be marked as \texttt{<div>} elements, and distinguished by the following suggested values of the \textit{@type} attribute: \begin{description}

\item[{appendix}]an appendix.
\item[{glossary}]a list of words and definitions, typically marked up as a \texttt{<list type="gloss">} element .
\item[{notes}]a series of \texttt{<note>} elements.
\item[{bibliography}]a series of bibliographic references, typically in the form of a special bibliographic-list element \texttt{<listBibl>}, whose items are individual \texttt{<bibl>} elements.
\item[{index}]a set of index entries, possibly represented as a structured list or glossary list, with optional leading \texttt{<head>} and perhaps some paragraphs of introductory or closing text (An index may also be generated for a document by using the \texttt{<index>} element, described above in section index).
\item[{colophon}]a description at the back of the book describing where, when, and by whom it was printed; in modern books it also often gives production details and identifies the type faces used.
\end{description} 
\section[{The Electronic Title Page}]{The Electronic Title Page}\label{U5-header}\par
Every TEI text has a header which provides information analogous to that provided by the title page of printed text. The header is introduced by the element \texttt{<teiHeader>} and has four major parts: \par
A corpus or collection of texts, which share many characteristics, may have one header for the corpus and individual headers for each component of the corpus. In this case the \textit{@type} attribute indicates the type of header. \texttt{<teiHeader type="corpus">} introduces the header for corpus-level information.\par
Some of the header elements contain running prose which consists of one or more \texttt{<p>}s. Others are grouped: \begin{itemize}
\item Elements whose names end in \textit{Stmt}(for statement) usually enclose a group of elements recording some structured information.
\item Elements whose names end in \textit{Decl} (for declaration) enclose information about specific encoding practices.
\item Elements whose names end in \textit{Desc} (for description) contain a prose description.
\end{itemize} 
\subsection[{The File Description}]{The File Description}\par
The \texttt{<fileDesc>} element is mandatory. It contains a full bibliographic description of the file with the following elements: \par
A minimal header has the following structure: \par\bgroup\exampleFont \begin{shaded}\noindent\mbox{}{<\textbf{teiHeader}>}\mbox{}\newline 
\hspace*{6pt}{<\textbf{fileDesc}>}\mbox{}\newline 
\hspace*{6pt}\hspace*{6pt}{<\textbf{titleStmt}>}\mbox{}\newline 
\textit{<!-- ... -->}\mbox{}\newline 
\hspace*{6pt}\hspace*{6pt}{</\textbf{titleStmt}>}\mbox{}\newline 
\hspace*{6pt}\hspace*{6pt}{<\textbf{publicationStmt}>}\mbox{}\newline 
\textit{<!-- ... -->}\mbox{}\newline 
\hspace*{6pt}\hspace*{6pt}{</\textbf{publicationStmt}>}\mbox{}\newline 
\hspace*{6pt}\hspace*{6pt}{<\textbf{sourceDesc}>}\mbox{}\newline 
\textit{<!-- ... -->}\mbox{}\newline 
\hspace*{6pt}\hspace*{6pt}{</\textbf{sourceDesc}>}\mbox{}\newline 
\hspace*{6pt}{</\textbf{fileDesc}>}\mbox{}\newline 
{</\textbf{teiHeader}>}\end{shaded}\egroup\par 
\subsubsection[{The Title Statement}]{The Title Statement}\par
The following elements can be used in the \texttt{<titleStmt>}: \par
It is recommended that the title should distinguish the computer file from the source text, for example: \par\hfill\bgroup\exampleFont\vskip 10pt\begin{shaded}
\obeyspaces [title of source]: a machine readable transcription\newline
[title of source]: electronic edition\newline
A machine readable version of: [title of source]\end{shaded}
\par\egroup 
 The \texttt{<respStmt>} element contains the following subcomponents:  Example: \par\bgroup\exampleFont \begin{shaded}\noindent\mbox{}{<\textbf{titleStmt}>}\mbox{}\newline 
\hspace*{6pt}{<\textbf{title}>}Two stories by Edgar Allen Poe: a machine readable\mbox{}\newline 
\hspace*{6pt}\hspace*{6pt} transcription{</\textbf{title}>}\mbox{}\newline 
\hspace*{6pt}{<\textbf{author}>}Poe, Edgar Allen (1809-1849){</\textbf{author}>}\mbox{}\newline 
\hspace*{6pt}{<\textbf{respStmt}>}\mbox{}\newline 
\hspace*{6pt}\hspace*{6pt}{<\textbf{resp}>}compiled by{</\textbf{resp}>}\mbox{}\newline 
\hspace*{6pt}\hspace*{6pt}{<\textbf{name}>}James D. Benson{</\textbf{name}>}\mbox{}\newline 
\hspace*{6pt}{</\textbf{respStmt}>}\mbox{}\newline 
{</\textbf{titleStmt}>}\end{shaded}\egroup\par 
\subsubsection[{The Edition Statement}]{The Edition Statement}\par
The \texttt{<editionStmt>} groups information relating to one edition of a text (where \textit{edition} is used as elsewhere in bibliography), and may include the following elements: \par
Example: \par\bgroup\exampleFont \begin{shaded}\noindent\mbox{}{<\textbf{editionStmt}>}\mbox{}\newline 
\hspace*{6pt}{<\textbf{edition}\hspace*{6pt}{n}="{U2}">}Third draft, substantially revised\mbox{}\newline 
\hspace*{6pt}{<\textbf{date}>}1987{</\textbf{date}>}\mbox{}\newline 
\hspace*{6pt}{</\textbf{edition}>}\mbox{}\newline 
{</\textbf{editionStmt}>}\end{shaded}\egroup\par \par
Determining exactly what constitutes a new edition of an electronic text is left to the encoder.
\subsubsection[{The Extent Statement}]{The Extent Statement}\par
The \texttt{<extent>} statement describe the approximate size of a file.\par
Example: \par\bgroup\exampleFont \begin{shaded}\noindent\mbox{}{<\textbf{extent}>}4532 bytes{</\textbf{extent}>}\end{shaded}\egroup\par 
\subsubsection[{The Publication Statement}]{The Publication Statement}\par
The \texttt{<publicationStmt>} is mandatory. It may contain a simple prose description or groups of the elements described below: \par
At least one of these three elements must be present, unless the entire publication statement is in prose. The following elements may occur within them: \par
Example: \par\bgroup\exampleFont \begin{shaded}\noindent\mbox{}{<\textbf{publicationStmt}>}\mbox{}\newline 
\hspace*{6pt}{<\textbf{publisher}>}Oxford University Press{</\textbf{publisher}>}\mbox{}\newline 
\hspace*{6pt}{<\textbf{pubPlace}>}Oxford{</\textbf{pubPlace}>}\mbox{}\newline 
\hspace*{6pt}{<\textbf{date}>}1989{</\textbf{date}>}\mbox{}\newline 
\hspace*{6pt}{<\textbf{idno}\hspace*{6pt}{type}="{ISBN}">} 0-19-254705-5{</\textbf{idno}>}\mbox{}\newline 
\hspace*{6pt}{<\textbf{availability}>}\mbox{}\newline 
\hspace*{6pt}\hspace*{6pt}{<\textbf{p}>}Copyright 1989, Oxford University\mbox{}\newline 
\hspace*{6pt}\hspace*{6pt}\hspace*{6pt}\hspace*{6pt} Press{</\textbf{p}>}\mbox{}\newline 
\hspace*{6pt}{</\textbf{availability}>}\mbox{}\newline 
{</\textbf{publicationStmt}>}\end{shaded}\egroup\par 
\subsubsection[{Series and Notes Statements}]{Series and Notes Statements}\par
The \texttt{<seriesStmt>} element groups information about the series, if any, to which a publication belongs. It may contain \texttt{<title>}, \texttt{<idno>}, or \texttt{<respStmt>} elements.\par
The \texttt{<notesStmt>}, if used, contains one or more \texttt{<note>} elements which contain a note or annotation. Some information found in the notes area in conventional bibliography has been assigned specific elements in the TEI scheme.
\subsubsection[{The Source Description}]{The Source Description}\par
The \texttt{<sourceDesc>} is a mandatory element which records details of the source or sources from which the computer file is derived. It may contain simple prose or a bibliographic citation, using one or more of the following elements: \par
Examples: \par\bgroup\exampleFont \begin{shaded}\noindent\mbox{}{<\textbf{sourceDesc}>}\mbox{}\newline 
\hspace*{6pt}{<\textbf{bibl}>}The first folio of Shakespeare, prepared by Charlton\mbox{}\newline 
\hspace*{6pt}\hspace*{6pt} Hinman (The Norton Facsimile, 1968){</\textbf{bibl}>}\mbox{}\newline 
{</\textbf{sourceDesc}>}\end{shaded}\egroup\par \noindent  \par\bgroup\exampleFont \begin{shaded}\noindent\mbox{}{<\textbf{sourceDesc}>}\mbox{}\newline 
\hspace*{6pt}{<\textbf{bibl}>}\mbox{}\newline 
\hspace*{6pt}\hspace*{6pt}{<\textbf{author}>}CNN Network News{</\textbf{author}>}\mbox{}\newline 
\hspace*{6pt}\hspace*{6pt}{<\textbf{title}>}News headlines{</\textbf{title}>}\mbox{}\newline 
\hspace*{6pt}\hspace*{6pt}{<\textbf{date}>}12 Jun 1989{</\textbf{date}>}\mbox{}\newline 
\hspace*{6pt}{</\textbf{bibl}>}\mbox{}\newline 
{</\textbf{sourceDesc}>}\end{shaded}\egroup\par 
\subsection[{The Encoding Description}]{The Encoding Description}\par
The \texttt{<encodingDesc>} element specifies the methods and editorial principles which governed the transcription of the text. Its use is highly recommended. It may be prose description or may contain elements from the following list: 
\subsubsection[{Project and Sampling Descriptions}]{Project and Sampling Descriptions}\par
Examples of \texttt{<projectDesc>} and \texttt{<samplingDesc>}: \par\bgroup\exampleFont \begin{shaded}\noindent\mbox{}{<\textbf{encodingDesc}>}\mbox{}\newline 
\hspace*{6pt}{<\textbf{projectDesc}>}\mbox{}\newline 
\hspace*{6pt}\hspace*{6pt}{<\textbf{p}>}Texts collected for use in the Claremont\mbox{}\newline 
\hspace*{6pt}\hspace*{6pt}\hspace*{6pt}\hspace*{6pt} Shakespeare Clinic, June 1990.\mbox{}\newline 
\hspace*{6pt}\hspace*{6pt}{</\textbf{p}>}\mbox{}\newline 
\hspace*{6pt}{</\textbf{projectDesc}>}\mbox{}\newline 
{</\textbf{encodingDesc}>}\end{shaded}\egroup\par \noindent  \par\bgroup\exampleFont \begin{shaded}\noindent\mbox{}{<\textbf{encodingDesc}>}\mbox{}\newline 
\hspace*{6pt}{<\textbf{samplingDecl}>}\mbox{}\newline 
\hspace*{6pt}\hspace*{6pt}{<\textbf{p}>}Samples of 2000 words taken from the beginning\mbox{}\newline 
\hspace*{6pt}\hspace*{6pt}\hspace*{6pt}\hspace*{6pt} of the text{</\textbf{p}>}\mbox{}\newline 
\hspace*{6pt}{</\textbf{samplingDecl}>}\mbox{}\newline 
{</\textbf{encodingDesc}>}\end{shaded}\egroup\par 
\subsubsection[{Editorial Declarations}]{Editorial Declarations}\par
The \texttt{<editorialDecl>} contains a prose description of the practices used when encoding the text. Typically this description should cover such topics as the following, each of which may conveniently be given as a separate paragraph. \begin{description}

\item[{correction }]how and under what circumstances corrections have been made in the text.
\item[{normalization}]the extent to which the original source has been regularized or normalized.
\item[{quotation}]what has been done with quotation marks in the original -- have they been retained or replaced by entity references, are opening and closing quotes distinguished, etc. 
\item[{hyphenation}]what has been done with hyphens (especially end-of-line hyphens) in the original -- have they been retained, replaced by entity references, etc.
\item[{segmentation}]how has the text has been segmented, for example into sentences, tone-units, graphemic strata, etc.
\item[{interpretation}]what analytic or interpretive information has been added to the text. 
\end{description} \par
Example: \par\bgroup\exampleFont \begin{shaded}\noindent\mbox{}{<\textbf{editorialDecl}>}\mbox{}\newline 
\hspace*{6pt}{<\textbf{p}>}The part of speech analysis applied throughout\mbox{}\newline 
\hspace*{6pt}\hspace*{6pt} section 4 was added by hand and has not been\mbox{}\newline 
\hspace*{6pt}\hspace*{6pt} checked.{</\textbf{p}>}\mbox{}\newline 
\hspace*{6pt}{<\textbf{p}>}Errors in transcription controlled by using the\mbox{}\newline 
\hspace*{6pt}\hspace*{6pt} WordPerfect spelling checker.{</\textbf{p}>}\mbox{}\newline 
\hspace*{6pt}{<\textbf{p}>}All words converted to Modern American spelling\mbox{}\newline 
\hspace*{6pt}\hspace*{6pt} using Webster's 9th Collegiate dictionary.{</\textbf{p}>}\mbox{}\newline 
\hspace*{6pt}{<\textbf{p}>}All quotation marks converted to entity\mbox{}\newline 
\hspace*{6pt}\hspace*{6pt} references odq and cdq.{</\textbf{p}>}\mbox{}\newline 
{</\textbf{editorialDecl}>}\end{shaded}\egroup\par 
\subsubsection[{Reference and Classification Declarations}]{Reference and Classification Declarations}\par
The \texttt{<refsDecl>} element is used to document the way in which any standard referencing scheme built into the encoding works. In its simplest form, it consists of prose description.\par
Example: \par\bgroup\exampleFont \begin{shaded}\noindent\mbox{}{<\textbf{refsDecl}>}\mbox{}\newline 
\hspace*{6pt}{<\textbf{p}>}The {<\textbf{att}>}n{</\textbf{att}>} attribute on each {<\textbf{gi}>}div{</\textbf{gi}>} contains the\mbox{}\newline 
\hspace*{6pt}\hspace*{6pt} canonical reference for each such division in the form\mbox{}\newline 
\hspace*{6pt}\hspace*{6pt} XX.yyy where XX is the book number in roman numeral and\mbox{}\newline 
\hspace*{6pt}\hspace*{6pt} yyy is the section number in arabic.{</\textbf{p}>}\mbox{}\newline 
{</\textbf{refsDecl}>}\end{shaded}\egroup\par \par
The \texttt{<classDecl>} element groups together definitions or sources for any descriptive classification schemes used by other parts of the header. At least one such scheme must be provided, encoded using the following elements: \par
In the simplest case, the taxonomy may be defined by a bibliographic reference, as in the following example: \par\bgroup\exampleFont \begin{shaded}\noindent\mbox{}{<\textbf{classDecl}>}\mbox{}\newline 
\hspace*{6pt}{<\textbf{taxonomy}\hspace*{6pt}{xml:id}="{LC-SH}">}\mbox{}\newline 
\hspace*{6pt}\hspace*{6pt}{<\textbf{bibl}>}Library of Congress Subject Headings\mbox{}\newline 
\hspace*{6pt}\hspace*{6pt}{</\textbf{bibl}>}\mbox{}\newline 
\hspace*{6pt}{</\textbf{taxonomy}>}\mbox{}\newline 
{</\textbf{classDecl}>}\end{shaded}\egroup\par \par
Alternatively, or in addition, the encoder may define a special purpose classification scheme, as in the following example: \par\bgroup\exampleFont \begin{shaded}\noindent\mbox{}{<\textbf{taxonomy}\hspace*{6pt}{xml:id}="{B}">}\mbox{}\newline 
\hspace*{6pt}{<\textbf{bibl}>}Brown Corpus{</\textbf{bibl}>}\mbox{}\newline 
\hspace*{6pt}{<\textbf{category}\hspace*{6pt}{xml:id}="{B.A}">}\mbox{}\newline 
\hspace*{6pt}\hspace*{6pt}{<\textbf{catDesc}>}Press Reportage{</\textbf{catDesc}>}\mbox{}\newline 
\hspace*{6pt}\hspace*{6pt}{<\textbf{category}\hspace*{6pt}{xml:id}="{B.A1}">}\mbox{}\newline 
\hspace*{6pt}\hspace*{6pt}\hspace*{6pt}{<\textbf{catDesc}>}Daily{</\textbf{catDesc}>}\mbox{}\newline 
\hspace*{6pt}\hspace*{6pt}{</\textbf{category}>}\mbox{}\newline 
\hspace*{6pt}\hspace*{6pt}{<\textbf{category}\hspace*{6pt}{xml:id}="{B.A2}">}\mbox{}\newline 
\hspace*{6pt}\hspace*{6pt}\hspace*{6pt}{<\textbf{catDesc}>}Sunday{</\textbf{catDesc}>}\mbox{}\newline 
\hspace*{6pt}\hspace*{6pt}{</\textbf{category}>}\mbox{}\newline 
\hspace*{6pt}\hspace*{6pt}{<\textbf{category}\hspace*{6pt}{xml:id}="{B.A3}">}\mbox{}\newline 
\hspace*{6pt}\hspace*{6pt}\hspace*{6pt}{<\textbf{catDesc}>}National{</\textbf{catDesc}>}\mbox{}\newline 
\hspace*{6pt}\hspace*{6pt}{</\textbf{category}>}\mbox{}\newline 
\hspace*{6pt}\hspace*{6pt}{<\textbf{category}\hspace*{6pt}{xml:id}="{B.A4}">}\mbox{}\newline 
\hspace*{6pt}\hspace*{6pt}\hspace*{6pt}{<\textbf{catDesc}>}Provincial{</\textbf{catDesc}>}\mbox{}\newline 
\hspace*{6pt}\hspace*{6pt}{</\textbf{category}>}\mbox{}\newline 
\hspace*{6pt}\hspace*{6pt}{<\textbf{category}\hspace*{6pt}{xml:id}="{B.A5}">}\mbox{}\newline 
\hspace*{6pt}\hspace*{6pt}\hspace*{6pt}{<\textbf{catDesc}>}Political{</\textbf{catDesc}>}\mbox{}\newline 
\hspace*{6pt}\hspace*{6pt}{</\textbf{category}>}\mbox{}\newline 
\hspace*{6pt}\hspace*{6pt}{<\textbf{category}\hspace*{6pt}{xml:id}="{B.A6}">}\mbox{}\newline 
\hspace*{6pt}\hspace*{6pt}\hspace*{6pt}{<\textbf{catDesc}>}Sports{</\textbf{catDesc}>}\mbox{}\newline 
\hspace*{6pt}\hspace*{6pt}{</\textbf{category}>}\mbox{}\newline 
\hspace*{6pt}{</\textbf{category}>}\mbox{}\newline 
\hspace*{6pt}{<\textbf{category}\hspace*{6pt}{xml:id}="{B.D}">}\mbox{}\newline 
\hspace*{6pt}\hspace*{6pt}{<\textbf{catDesc}>}Religion{</\textbf{catDesc}>}\mbox{}\newline 
\hspace*{6pt}\hspace*{6pt}{<\textbf{category}\hspace*{6pt}{xml:id}="{B.D1}">}\mbox{}\newline 
\hspace*{6pt}\hspace*{6pt}\hspace*{6pt}{<\textbf{catDesc}>}Books{</\textbf{catDesc}>}\mbox{}\newline 
\hspace*{6pt}\hspace*{6pt}{</\textbf{category}>}\mbox{}\newline 
\hspace*{6pt}\hspace*{6pt}{<\textbf{category}\hspace*{6pt}{xml:id}="{B.D2}">}\mbox{}\newline 
\hspace*{6pt}\hspace*{6pt}\hspace*{6pt}{<\textbf{catDesc}>}Periodicals and tracts{</\textbf{catDesc}>}\mbox{}\newline 
\hspace*{6pt}\hspace*{6pt}{</\textbf{category}>}\mbox{}\newline 
\hspace*{6pt}{</\textbf{category}>}\mbox{}\newline 
{</\textbf{taxonomy}>}\end{shaded}\egroup\par \par
Linkage between a particular text and a category within such a taxonomy is made by means of the \texttt{<catRef>} element within the \texttt{<textClass>} element, as further described below.
\subsection[{The Profile Description}]{The Profile Description}\par
The \texttt{<profileDesc>} element enables information characterizing various descriptive aspects of a text to be recorded within a single framework. It has three optional components: \par
The \texttt{<creation>} element is useful for documenting where a work was created, even though it may not have been published or recorded there.\par
Example: \par\bgroup\exampleFont \begin{shaded}\noindent\mbox{}{<\textbf{creation}>}\mbox{}\newline 
\hspace*{6pt}{<\textbf{date}\hspace*{6pt}{when}="{1992-08}">}August 1992{</\textbf{date}>}\mbox{}\newline 
\hspace*{6pt}{<\textbf{name}\hspace*{6pt}{type}="{place}">}Taos, New Mexico{</\textbf{name}>}\mbox{}\newline 
{</\textbf{creation}>}\end{shaded}\egroup\par \par
The \texttt{<langUsage>} element is useful where a text contains many different languages. It may contain \texttt{<language>} elements to document each particular language used:  an example is needed.\par
The \texttt{<textClass>} element classifies a text by reference to the system or systems defined by the \texttt{<classDecl>} element, and contains one or more of the following elements: \par
The element \texttt{<keywords>} contains a list of keywords or phrases identifying the topic or nature of a text. The attribute \textit{@scheme} links these to the classification system defined in \texttt{<taxonomy>}. \par\bgroup\exampleFont \begin{shaded}\noindent\mbox{}{<\textbf{textClass}>}\mbox{}\newline 
\hspace*{6pt}{<\textbf{keywords}\hspace*{6pt}{scheme}="{LCSH}">}\mbox{}\newline 
\hspace*{6pt}\hspace*{6pt}{<\textbf{list}>}\mbox{}\newline 
\hspace*{6pt}\hspace*{6pt}\hspace*{6pt}{<\textbf{item}>}English literature -- History and criticism --\mbox{}\newline 
\hspace*{6pt}\hspace*{6pt}\hspace*{6pt}\hspace*{6pt}\hspace*{6pt}\hspace*{6pt} Data processing.{</\textbf{item}>}\mbox{}\newline 
\hspace*{6pt}\hspace*{6pt}\hspace*{6pt}{<\textbf{item}>}English literature -- History and criticism --\mbox{}\newline 
\hspace*{6pt}\hspace*{6pt}\hspace*{6pt}\hspace*{6pt}\hspace*{6pt}\hspace*{6pt} Theory etc.{</\textbf{item}>}\mbox{}\newline 
\hspace*{6pt}\hspace*{6pt}\hspace*{6pt}{<\textbf{item}>}English language -- Style -- Data\mbox{}\newline 
\hspace*{6pt}\hspace*{6pt}\hspace*{6pt}\hspace*{6pt}\hspace*{6pt}\hspace*{6pt} processing.{</\textbf{item}>}\mbox{}\newline 
\hspace*{6pt}\hspace*{6pt}{</\textbf{list}>}\mbox{}\newline 
\hspace*{6pt}{</\textbf{keywords}>}\mbox{}\newline 
{</\textbf{textClass}>}\end{shaded}\egroup\par 
\subsection[{The Revision Description}]{The Revision Description}\par
The \texttt{<revisionDesc>} element provides a change log in which each change made to a text may be recorded. The log may be recorded as a sequence of \texttt{<change>} elements each of which contains a brief description of the change. The attributes \textit{@date} and \textit{@who} may be used to identify when the change was carried out and the agency responsible for it.\par
Example: \par\bgroup\exampleFont \begin{shaded}\noindent\mbox{}{<\textbf{revisionDesc}>}\mbox{}\newline 
\hspace*{6pt}{<\textbf{change}\hspace*{6pt}{when}="{1991-03-06}"\hspace*{6pt}{who}="{EMB}">}File format updated{</\textbf{change}>}\mbox{}\newline 
\hspace*{6pt}{<\textbf{change}\hspace*{6pt}{when}="{1990-05-25}"\hspace*{6pt}{who}="{EMB}">}Stuart's corrections entered{</\textbf{change}>}\mbox{}\newline 
{</\textbf{revisionDesc}>}\end{shaded}\egroup\par 
\section[{List of Elements Described}]{List of Elements Described}\label{U5-taglist}\par
The following list shows all the elements defined for the TEI Lite schema, with a brief description of each, and a link to its full specification in the Appendix. 
\end{document}
